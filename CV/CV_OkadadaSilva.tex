\documentclass[letterpaper,11pt]{article}

\usepackage{latexsym}
\usepackage[empty]{fullpage}
\usepackage{titlesec}
\usepackage{marvosym}
\usepackage[usenames,dvipsnames]{color}
\usepackage{verbatim}
\usepackage{enumitem}
\usepackage[hidelinks, colorlinks = true,linkcolor = blue, urlcolor  = blue]{hyperref}
\usepackage{fancyhdr}
\usepackage[english]{babel}
\usepackage{tabularx}
\usepackage{multicol}
\usepackage{ragged2e}
\usepackage{changepage}
\usepackage{enumitem}
\usepackage{multirow}
\input{glyphtounicode}

\usepackage{baskervillef}
\usepackage[T1]{fontenc}

\pagestyle{fancy}
\fancyhf{} 
\fancyfoot{}
\setlength{\footskip}{10pt}
\renewcommand{\headrulewidth}{0pt}
\renewcommand{\footrulewidth}{0pt}

\addtolength{\oddsidemargin}{0.0in}
\addtolength{\evensidemargin}{0.0in}
\addtolength{\textwidth}{0.0in}
\addtolength{\topmargin}{0.2in}
\addtolength{\textheight}{0.0in}


\urlstyle{same}

%\raggedbottom
\raggedright
\setlength{\tabcolsep}{0in}

\titleformat{\section}{
  \it\vspace{0pt}
}{}{0em}{}[\color{black}\titlerule\vspace{-5pt}]

\pdfgentounicode=1

\newcommand{\resumeItem}[1]{
  \item{
    {#1 \vspace{-4pt}}
  }
}

\newcommand{\resumeSubheading}[4]{
  \vspace{-2pt}\item
    \begin{tabular*}{0.97\textwidth}[t]{l@{\extracolsep{\fill}}r}
      \textbf{#1} & #2 \\
      \textit{\small #3} & \textit{\small #4} \\
    \end{tabular*}\vspace{-10pt}
}


\newcommand{\resumeSubItem}[1]{\resumeItem{#1}\vspace{-3pt}}
\renewcommand\labelitemii{$\vcenter{\hbox{\tiny$\bullet$}}$}
\newcommand{\resumeSubHeadingListStart}{\begin{itemize}[leftmargin=0.15in, label={}]}
\newcommand{\resumeSubHeadingListEnd}{\end{itemize}}
\newcommand{\resumeItemListStart}{\begin{itemize}}
\newcommand{\resumeItemListEnd}{\end{itemize}\vspace{-2pt}}

\begin{document}

  \begin{tabular*}{\textwidth}{l@{\extracolsep{\fill}}r}
  \textbf{}                                
  & \scriptsize Email: vo10@illinois.edu \\
  \multirow{2}{*}{\huge \textbf{Vinicius Okada da Silva}} 
  & \scriptsize Phone: +1 (301) 502-6313 \\
  & \scriptsize \textcolor{blue}{\url{https://sites.google.com/view/viniciusokadadasilva/}}
  \end{tabular*}

\rule{\textwidth}{0.4pt}

\vspace{-3mm}

\rule{\textwidth}{0.4pt}


\section{Education}
\resumeSubHeadingListStart
    \resumeSubheading
        {University of Illinois Urbana-Champaign}{2019 -- Present}
        {Ph.D. in Economics}{}
    \resumeSubheading
        {Case Western Reserve University}{2015 -- 2019}
        {BA Economics, BS Math and Physics}{}
\resumeSubHeadingListEnd

\section{Research Interests}
\begin{itemize}[leftmargin=0.15in, label={}]
    \normalsize{\item{
     \textbf{Economic History}, \textbf{Development Economics}
    }}
 \end{itemize}

 \section{Working Papers}
    \hspace{3mm}
    \textbf{"Land Grants in Colonial Brazil: Long-Term Effects on
    Development"} (\textbf{Job Market Paper})
    \vspace{3mm}

    \hspace{3mm}
    \textbf{"Jesuit Missionaries in the Colonial Amazon: Long-term Effects on Human Capital."} (\textit{Submitted})
    \vspace{3mm}
    %\justify{
    %\begin{adjustwidth}{3mm}{0mm}
    %\textit{Abstract:} This paper aims to identify Jesuit missions' long-term impact on human capital and development in the Brazilian Amazon. Using Brazilian census data from 1872 and 2010 combined with a novel dataset on the location of Jesuit missions in the Brazilian Amazon, I find that places closer to the former missions had higher literacy rates in both periods. To estimate the causal effect, I compare the impacts of the Jesuits against other missionary orders, the nearby Jesuit missions in the state of Maranhão, and I use an instrumental variable approach that considers the locations of Tupi-speaking tribes in the region. Further, demographic differences, the number of schools, or school attendance do not explain the results. This paper extends and reinforces the literature analyzing the positive effects of missionaries' transmission of human capital in the colonial period.
    %\end{adjustwidth}}

    %\vspace{5mm}
    \hspace{3mm}
    \textbf{"Long-Lasting Effects of Bible Translations on Literacy:  Evidence from Sub-Saharan Africa"} 

    \hspace{3mm}
    \textit{with Noelia Romero, Abigail Stocker, and Rebecca Thornton}
    \vspace{3mm}
    %\justify{
    %\begin{adjustwidth}{3mm}{0mm}
    %\textit{Abstract:} This paper evaluates the impact of early life exposure to Bible translations on education. To estimate causal effects and avoid issues with selection into translation and mission location, we compare educational outcomes across cohorts of individuals within language groups with and without exposure to a Bible translation in their mother-tongue language. We analyze data from a representative sample of approximately 75,000 adults in 13 sub-Saharan African countries using the Demographic and Health Surveys. Our difference-in-differences strategy accounts for the differential timing of Bible translations and the trends in educational outcomes over time within each language group. Individuals born ten years after the first Bible translation are 12 percentage points more likely to be literate later in life and attain 1.3 additional years of education than those born before the translation. Effects do not vary by proximity to missions (either Catholic or Protestant), distance to a printing press, urban area, or religious faith. We provide the first causal evidence of the impact of Bible translations on education
    %\end{adjustwidth}}

    %\vspace{5mm}
    \hspace{3mm}
    \textbf{"Staple Crop Pest Damage and Natural Resources Exploitation: Fall Army Worm Infestation and Charcoal Production in Zambia"} 
    
    \hspace{3mm}
    \textit{with Protensia Hadunka and Kathy Baylis}
    %\justify{
    %\begin{adjustwidth}{3mm}{0mm}
    %\textit{Abstract:} Sub-Saharan Africa (SSA) is home to some of the highest rates of deforestation in the world. One driver may be negative agricultural shocks that drive households to consume natural resources as a coping mechanism. This paper uses primary household panel data from Zambia to estimate the effect of the introduction of an agricultural pest, fall armyworms (FAW), on charcoal production. We exploit exogenous variation in the intensity of exposure to FAW across households and years to identify their effect. We find a positive and significant effect of FAW on charcoal production and deforestation. The estimates indicate that the FAW in a village increases the probability of a farmer producing charcoal by 3.48 percentage points, from 22 percent to 25 percent, leading to an increase in deforestation of 13.6 percent. The results also indicate that when methods to mitigate FAW damage are available, farmers are less likely to resort to charcoal production as a coping strategy. Having the ability to reduce the share of maize, use pesticides, or migrate for off-farm employment are associated to be successful ways to mitigate the use of charcoal in the face of agricultural production shocks.
    %\end{adjustwidth}}

\section{Work in Progress}

\begin{adjustwidth}{3mm}{0mm}
\textbf{"Effect of the 2011 Libyan Crisis on Insecurity in the Sahel Region"}  \textit{with Togbedji Gansey and Mahounan Yedomiffi}
\end{adjustwidth}

\section{Teaching Experience}
\begin{adjustwidth}{3mm}{0mm}
\textbf{Department of Economics - University of Illinois Urbana-Champaign}
\end{adjustwidth}

\begin{adjustwidth}{6mm}{0mm}
  \textbf{\textit{Lecturer:}}
\end{adjustwidth}

\begin{adjustwidth}{9mm}{0mm}
  \textit{Spring 2023}: ECON 303 - Intermediate Macroeconomics
\end{adjustwidth}

\begin{adjustwidth}{9mm}{0mm}
  \textit{Fall 2022}: ECON 303 - Intermediate Macroeconomics$^*$
\end{adjustwidth}

\begin{adjustwidth}{6mm}{0mm}
  \textbf{\textit{Head Teaching Assistant:}}
\end{adjustwidth}

\begin{adjustwidth}{9mm}{0mm}
  \textit{Fall 2023}: ECON 303 - Intermediate Macroeconomics
\end{adjustwidth}

\begin{adjustwidth}{9mm}{0mm}
  \textit{Fall 2022}: ECON 303 - Intermediate Macroeconomics$^*$
\end{adjustwidth}

\begin{adjustwidth}{9mm}{0mm}
  \textit{Spring 2022}: ECON 303 - Intermediate Macroeconomics$^*$
\end{adjustwidth}

\begin{adjustwidth}{9mm}{0mm}
  \textit{Fall 2021}: ECON 303 - Intermediate Macroeconomics$^*$
\end{adjustwidth}

\begin{adjustwidth}{6mm}{0mm}
  \textbf{\textit{Teaching Assistant:}}
\end{adjustwidth}

\begin{adjustwidth}{9mm}{0mm}
  \textit{Spring 2021}: ECON 303 - Intermediate Macroeconomics
\end{adjustwidth}

\begin{adjustwidth}{9mm}{0mm}
  \textit{Fall 2020}: ECON 303 - Intermediate Macroeconomics$^*$
\end{adjustwidth}

\begin{adjustwidth}{3mm}{0mm}
  \scriptsize * indicates ranked in the "List of Teachers Ranked as Excellent by their Students".
\end{adjustwidth}

\begin{adjustwidth}{3mm}{0mm}
  \textbf{Case Western Reserve University}
\end{adjustwidth}

\begin{adjustwidth}{6mm}{0mm}
  \textbf{\textit{Supplemental Instructor:}}
\end{adjustwidth}

\begin{adjustwidth}{9mm}{0mm}
  \textit{Fall 2018 - Spring 2019}: PHYS 121 - General Physics I - Mechanics
\end{adjustwidth}

\section{Awards}

\begin{adjustwidth}{6mm}{0mm}
  Lemann Fellowship - Summer 2023
\end{adjustwidth}

\section{Presentations}
\begin{adjustwidth}{3mm}{0mm}
  \begin{itemize}
    \item \href{https://www.southerneconomic.org/event/7662b305-ad92-474d-8f2c-bce1240b9858/websitePage:e0a9d079-18e0-4413-a0b6-32f8f6b51e64}{Southern Economic Association} (2023) 
    
    \textit{"Long-Lasting Effects of Exposure to Bible Translations: Evidence from Sub-Saharan Africa."}
  \end{itemize}

  \begin{itemize}
    \item \href{http://www.asrec.org/conferences/}{Association for the Study of Religion, Economics, and Culture Conference} (2023) 
    
    \textit{"Jesuit Missionaries in the Colonial Amazon: Long-term Effects on Human Capital."}
  \end{itemize}

  \begin{itemize}
    \item \href{https://cies2023.org/}{Comparative \& International Education Society Conference} (2023) 
    
    \textit{"Long-Lasting Effects of Bible Translation on Literacy: Evidence from Uganda"}
  \end{itemize}

  \begin{itemize}
    \item \href{https://kellogg.nd.edu/development-day-2021#tab-2998}{Development Day at Notre Dame} (2021) 
    
    \textit{"Jesuit Missionaries in the Colonial Amazon: Long-term Effects on Human Capital."}
  \end{itemize}
\end{adjustwidth}

\section{Referee Services}
\begin{adjustwidth}{3mm}{0mm}
\textit{Economics of Education Review}
\end{adjustwidth}

\end{document}



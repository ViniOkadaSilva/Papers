\documentclass[letterpaper,11pt]{article}

\usepackage{latexsym}
\usepackage[empty]{fullpage}
\usepackage{titlesec}
\usepackage{marvosym}
\usepackage[usenames,dvipsnames]{color}
\usepackage{verbatim}
\usepackage{enumitem}
\usepackage[hidelinks, colorlinks = true,linkcolor = blue, urlcolor  = blue]{hyperref}
\usepackage{fancyhdr}
\usepackage[english]{babel}
\usepackage{tabularx}
\usepackage{multicol}
\usepackage{ragged2e}
\usepackage{changepage}
\usepackage{enumitem}
\usepackage{multirow}
\input{glyphtounicode}

\usepackage{baskervillef}
\usepackage[T1]{fontenc}

\pagestyle{fancy}
\fancyhf{} 
\fancyfoot{}
\fancyfoot[L]{\textit{Updated \today}}
\setlength{\footskip}{10pt}
\renewcommand{\headrulewidth}{0pt}
\renewcommand{\footrulewidth}{0pt}

\addtolength{\oddsidemargin}{0.0in}
\addtolength{\evensidemargin}{0.0in}
\addtolength{\textwidth}{0.0in}
\addtolength{\topmargin}{0.2in}
\addtolength{\textheight}{0.0in}


\urlstyle{same}

%\raggedbottom
\raggedright
\setlength{\tabcolsep}{0in}

\titleformat{\section}{
  \it\vspace{0pt}
}{}{0em}{}[\color{black}\titlerule\vspace{-5pt}]

\pdfgentounicode=1

\newcommand{\resumeItem}[1]{
  \item{
    {#1 \vspace{-4pt}}
  }
}

\newcommand{\resumeSubheading}[4]{
  \vspace{-2pt}\item
    \begin{tabular*}{0.97\textwidth}[t]{l@{\extracolsep{\fill}}r}
      \textbf{#1} & #2 \\
      \textit{\small #3} & \textit{\small #4} \\
    \end{tabular*}\vspace{-10pt}
}


\newcommand{\resumeSubItem}[1]{\resumeItem{#1}\vspace{-3pt}}
\renewcommand\labelitemii{$\vcenter{\hbox{\tiny$\bullet$}}$}
\newcommand{\resumeSubHeadingListStart}{\begin{itemize}[leftmargin=0.15in, label={}]}
\newcommand{\resumeSubHeadingListEnd}{\end{itemize}}
\newcommand{\resumeItemListStart}{\begin{itemize}}
\newcommand{\resumeItemListEnd}{\end{itemize}\vspace{-2pt}}

\begin{document}

  \begin{tabular*}{\textwidth}{l@{\extracolsep{\fill}}r}
  \textbf{}                                
  & \scriptsize Email: vo10@illinois.edu \\
  \multirow{2}{*}{\huge \textbf{Vinicius Okada da Silva}} 
  & \scriptsize Phone: +1 (301) 502-6313 \\
  & \scriptsize \textcolor{blue}{\url{https://sites.google.com/view/viniciusokadadasilva/}}
  \end{tabular*}

\rule{\textwidth}{0.4pt}


\section{Education}
\resumeSubHeadingListStart
    \resumeSubheading
        {University of Illinois}{2019 -- Expected 2025}
        {Ph.D. in Economics}{}
    \resumeSubheading
        {Case Western Reserve University}{2015 -- 2019}
        {BA Economics, BS Math and Physics}{}
\resumeSubHeadingListEnd

\section{Research Interests}
\begin{itemize}[leftmargin=0.15in, label={}]
    \normalsize{\item{
     \textbf{Economic History}, \textbf{Development Economics}
    }}
 \end{itemize}

\section{Working Papers}
\hspace{3mm}
\textbf{"Land Grants in Colonial Brazil: Long-Term Effects on
Development"} (\textbf{Job Market Paper})
\vspace{.5mm}
\justify{
\begin{adjustwidth}{3mm}{0mm}
\end{adjustwidth}}

\vspace{-5mm}
\hspace{-3mm}
\textbf{"Jesuit Missionaries in the Colonial Amazon: Long-term Effects on Human Capital"} 
\vspace{.5mm}

\vspace{3mm}
\hspace{-3mm}
\textbf{"Long-Lasting Effects of Bible Translations on Literacy:  Evidence from Sub-Saharan Africa"} 

\hspace{0.5mm}
\textit{with Noelia Romero, Abigail Stocker, and Rebecca Thornton}
\vspace{0.5mm}
 

%\hspace{0.5mm}
\vspace{3mm}
\hspace{-3mm}
\textbf{"Staple Crop Pest Damage and Natural Resources Exploitation: Fall Army Worm Infestation and}

\hspace{-2mm}
\textbf{Charcoal Production in Zambia"} 
\hspace{0.5mm}
\textit{with Protensia Hadunka and Kathy Baylis}
%\justify{
%\begin{adjustwidth}{3mm}{0mm}
%\textit{Abstract:} Sub-Saharan Africa (SSA) is home to some of the highest rates of deforestation in the world. One driver may be negative agricultural shocks that drive households to consume natural resources as a coping mechanism. This paper uses primary household panel data from Zambia to estimate the effect of the introduction of an agricultural pest, fall armyworms (FAW), on charcoal production. We exploit exogenous variation in the intensity of exposure to FAW across households and years to identify their effect. We find a positive and significant effect of FAW on charcoal production and deforestation. The estimates indicate that the FAW in a village increases the probability of a farmer producing charcoal by 3.48 percentage points, from 22 percent to 25 percent, leading to an increase in deforestation of 13.6 percent. The results also indicate that when methods to mitigate FAW damage are available, farmers are less likely to resort to charcoal production as a coping strategy. Having the ability to reduce the share of maize, use pesticides, or migrate for off-farm employment are associated to be successful ways to mitigate the use of charcoal in the face of agricultural production shocks.
%\end{adjustwidth}}

\section{Work in Progress}

\begin{adjustwidth}{3mm}{0mm}
\textbf{"Terrorism Activities and Children Outcomes"} 

\hspace{0.25mm}
\textit{with Togbedji Gansey and Mahounan Yedomiffi}
\end{adjustwidth}

\section{Teaching Experience}
\begin{adjustwidth}{3mm}{0mm}
\textbf{Department of Economics - University of Illinois}
\end{adjustwidth}

\begin{adjustwidth}{6mm}{0mm}
  \textbf{\textit{Lecturer:}}
\end{adjustwidth}

\begin{adjustwidth}{9mm}{0mm}
  \textit{Fall 2022 - Spring 2023}: ECON 303 - Intermediate Macroeconomics
\end{adjustwidth}

\begin{adjustwidth}{6mm}{0mm}
  \textbf{\textit{Head Teaching Assistant:}}
\end{adjustwidth}

\begin{adjustwidth}{9mm}{0mm}
  \textit{Fall 2023 - Spring 2024}: ECON 303 - Intermediate Macroeconomics
\end{adjustwidth}

\begin{adjustwidth}{9mm}{0mm}
  \textit{Fall 2021 - Fall 2023}: ECON 303 - Intermediate Macroeconomics
\end{adjustwidth}

\begin{adjustwidth}{6mm}{0mm}
  \textbf{\textit{Teaching Assistant:}}
\end{adjustwidth}

\begin{adjustwidth}{9mm}{0mm}
  \textit{Fall 2020 - Spring 2021}: ECON 303 - Intermediate Macroeconomics
\end{adjustwidth}
\vspace{2mm}
\begin{adjustwidth}{3mm}{0mm}
  \textbf{\textit{List of Teachers Ranked as Excellent:}}
  \\
  Fall 2020, Fall 2021, Spring 2022, Fall 2022, Spring 2024
\end{adjustwidth}

\vspace{2mm}

\begin{adjustwidth}{3mm}{0mm}
  \textbf{Case Western Reserve University}
\end{adjustwidth}

\begin{adjustwidth}{6mm}{0mm}
  \textbf{\textit{Supplemental Instructor:}}
\end{adjustwidth}

\begin{adjustwidth}{9mm}{0mm}
  \textit{Fall 2018 - Spring 2019}: PHYS 121 - General Physics I - Mechanics
\end{adjustwidth}

\section{Awards}

\begin{adjustwidth}{6mm}{0mm}
  Lemann Fellowship - Summer 2023
  \\
  Conference Travel Award - Fall 2023
\end{adjustwidth}

\section{Presentations}
\begin{adjustwidth}{3mm}{0mm}
  \begin{itemize}
    \item \href{https://cssh.northeastern.edu/gap/neudc-2024/}{North East Universities Development Consortium Conference} (2024 - Upcoming) 
    
    \textit{"Land Grants in Colonial Brazil and Long-Term Effects on Development."}
  \end{itemize}

  \begin{itemize}
    \item \href{https://www.southerneconomic.org/event/7662b305-ad92-474d-8f2c-bce1240b9858/websitePage:e0a9d079-18e0-4413-a0b6-32f8f6b51e64}{Southern Economic Association} (2023) 
    
    \textit{"Long-Lasting Effects of Exposure to Bible Translations: Evidence from Sub-Saharan Africa."}
  \end{itemize}

  \begin{itemize}
    \item \href{http://www.asrec.org/conferences/}{Association for the Study of Religion, Economics, and Culture Conference} (2023) 
    
    \textit{"Jesuit Missionaries in the Colonial Amazon: Long-term Effects on Human Capital."}
  \end{itemize}

  \begin{itemize}
    \item \href{https://cies2023.org/}{Comparative \& International Education Society Conference} (2023) 
    
    \textit{"Long-Lasting Effects of Bible Translation on Literacy: Evidence from Uganda"}
  \end{itemize}

  \begin{itemize}
    \item \href{https://kellogg.nd.edu/development-day-2021#tab-2998}{Development Day at Notre Dame} (2021) 
    
    \textit{"Jesuit Missionaries in the Colonial Amazon: Long-term Effects on Human Capital."}
  \end{itemize}

  \begin{itemize}
    \item University of Illinois, Economics Department - Applied Micro Seminars (2020-2024)
  \end{itemize}
\end{adjustwidth}

\section{Referee Services}
\begin{adjustwidth}{3mm}{0mm}
\textit{Economics of Education Review}
\end{adjustwidth}

\section*{References}
\begin{multicols}{3}
    \noindent
    Prof. Richard Akresh \\
    University of Illinois \\
    1407 W. Gregory Drive \\
    214 David Kinley Hall \\
    Urbana IL 61801 \\
    (217)-377-5783 \\
    \href{mailto:akresh@illinois.edu}{akresh@illinois.edu}
    \columnbreak

    \noindent
    Prof. Felipe Valencia Caicedo \\
    University of British Columbia \\
    6000 Iona Drivee \\
    Office \# 114 \\
    Vancouver, BC V6T 1L4  \\
    (604)-827-0004 \\
    \href{mailto:felipe.valencia@ubc.ca}{felipe.valencia@ubc.ca}
    \columnbreak

    \noindent
    Prof. Mary Arends-Kuenning \\
    University of Illinois \\
    1301 Gregory Dr \\
    408 Mumford Hall \\
    Urbana IL 61801 \\
    (217) 333-0753 \\
    \href{mailto:marends@illinois.edu}{marends@illinois.edu}
\end{multicols}

\clearpage

\section{Abstracts}
\hspace{3mm}
    \textbf{"Land Grants in Colonial Brazil: Long-Term Effects on
    Development"} (\textbf{Job Market Paper})
    \vspace{.5mm}
    \justify{
    \begin{adjustwidth}{3mm}{0mm}
    \textit{Abstract:} Legal access to land in Brazil has been a key political issue for the past century. The concentration of land in large estates that are often unproductive is argued to be a factor in the rural population's low social mobility and inequality. However, restricted land access in Brazil has its roots in colonial times. Large plots of land were granted from 1530-1822 through land grants called \textit{sesmarias}. 
    Through a novel georeferenced dataset on the location of the grants in eight Brazilian states, I estimate the long-term effects of the grants on Brazil's land distribution. 
    Using propensity score matching, an instrumental variable, and exploiting colonial policy variation on where the grants could be assigned, I find consistent positive effects of land concentration in 1995 for municipalities with a \textit{sesmaria}.
    Land grants are also associated with increased land conflicts and urbanization in modern Brazil.
    \end{adjustwidth}}

    \vspace{3mm}
    \hspace{-5mm}
    \textbf{"Jesuit Missionaries in the Colonial Amazon: Long-term Effects on Human Capital"} 
    \vspace{.5mm}
    \justify{
    \begin{adjustwidth}{3mm}{0mm}
    \textit{Abstract:} This paper aims to identify Jesuit missions' long-term impact on human capital and development in the Brazilian Amazon. Using Brazilian census data from 1872 and 2010 combined with a novel dataset on the location of Jesuit missions in the Brazilian Amazon, I find that places closer to the former missions had higher literacy rates in both periods. To estimate the causal effect, I use an instrumental variable approach that considers the locations of Tupi-speaking tribes in the region. I also compare the impacts of the Jesuits against other missionary orders. 
    Using microcensus data, the results indicate that the effects have persisted through 1970-2010. Further, demographic differences, the number of schools, or school attendance do not explain the results. This paper extends and reinforces the literature analyzing the positive effects of missionaries' transmission of human capital in the colonial period.
    \end{adjustwidth}}

    \vspace{3mm}
    \hspace{-5mm}
    \textbf{"Long-Lasting Effects of Bible Translations on Literacy:  Evidence from Sub-Saharan Africa"} 
    
    \hspace{0.5mm}
    \textit{with Noelia Romero, Abigail Stocker, and Rebecca Thornton} (\textit{Submitted})
    \vspace{0.5mm}
    \justify{
    \begin{adjustwidth}{3mm}{0mm}
    \textit{Abstract:} This paper evaluates the impact of early life exposure to mother-tongue Bible translations in the 1980s on adult educational outcomes. We analyze data from a sample of approximately 75,000 adults in the Demographic and Health Surveys in 13 sub-Saharan African countries. Our difference-in-differences strategy compares educational outcomes within and across ethno-linguistic groups and accounts for the differential timing of Bible translations and trends in outcomes over time. Individuals born ten years after the first-known Bible translation for their ethno-linguistic group have an 11 percentage point increase in the likelihood of being literate as adults, a gain of 1.2 years of education, and a 17 percentage point gain in the likelihood of completing primary school.
    Effects do not vary greatly by gender or by regional prevalence of Muslim faith.
    We discuss possible mechanisms underlying our results, finding evidence of the potential importance of complementarities with inputs concentrated around historical missions.
    \end{adjustwidth}}
\end{document}



\documentclass{article}

\usepackage[utf8]{inputenc}
\usepackage{subfig}
\usepackage{setspace}

\usepackage{amsmath}
\doublespacing

\usepackage{indentfirst}
\usepackage{pdflscape}
\usepackage[left=1in,right=1in,top=1in,bottom=1in]{geometry}

\DeclareUnicodeCharacter{0301}{\'{e}}

\usepackage[font=small,skip=0pt]{caption}
\usepackage[section]{placeins}
\usepackage{titlesec}
\titlelabel{\thetitle.\quad}
\usepackage{authblk}
\usepackage{csquotes}
\usepackage{booktabs}
\usepackage{siunitx}
\usepackage{amssymb}
\newcolumntype{d}{S[input-symbols = ()]}
\usepackage{float}
\usepackage{dcolumn}
\usepackage[bottom]{footmisc} 
\usepackage[textsize=tiny]{todonotes}
\usepackage{longtable}
\usepackage{array}
\usepackage{multirow}
\usepackage{wrapfig}
\usepackage{float}
\usepackage{colortbl}
\usepackage{pdflscape}
\usepackage{tabu}
\usepackage{threeparttable}
\usepackage{threeparttablex}
\usepackage[normalem]{ulem}
\usepackage{makecell}
\usepackage{xcolor}
\usepackage{afterpage}
\usepackage{graphicx}
\usepackage{bbding}
\usepackage{xargs}
\usepackage[citecolor = blue]{hyperref}
\hypersetup{colorlinks = true, linkcolor = red, urlcolor = teal}

\usepackage{bookmark}
\usepackage{todonotes}
\setlength{\marginparwidth}{2cm}

\usepackage[backend = biber, style=authoryear, sorting = nty]{biblatex}

\addbibresource{citationsv3.bib}

\DeclareFieldFormat{citehyperref}{%
  \DeclareFieldAlias{bibhyperref}{noformat}% Avoid nested links
  \bibhyperref{#1}}

\DeclareFieldFormat{textcitehyperref}{%
  \DeclareFieldAlias{bibhyperref}{noformat}% Avoid nested links
  \bibhyperref{%
    #1%
    \ifbool{cbx:parens}
      {\bibcloseparen\global\boolfalse{cbx:parens}}
      {}}}

\savebibmacro{cite}
\savebibmacro{textcite}

\renewbibmacro*{cite}{%
  \printtext[citehyperref]{%
    \restorebibmacro{cite}%
    \usebibmacro{cite}}}

\renewbibmacro*{textcite}{%
  \ifboolexpr{
    ( not test {\iffieldundef{prenote}} and
      test {\ifnumequal{\value{citecount}}{1}} )
    or
    ( not test {\iffieldundef{postnote}} and
      test {\ifnumequal{\value{citecount}}{\value{citetotal}}} )
  }
    {\DeclareFieldAlias{textcitehyperref}{noformat}}
    {}%
  \printtext[textcitehyperref]{%
    \restorebibmacro{textcite}%
    \usebibmacro{textcite}}}

\renewcommand*{\nameyeardelim}{\addcomma\space}

% Makes the authors name bold in the References section:

\def\sectionautorefname{Section}

\graphicspath{{~/OneDrive - University of Illinois - Urbana/Research/Projects/Third Year Paper/Pictures/}}

\title{Jesuit Missionaries in the Colonial Amazon: Long-term effects on Human Capital\thanks{I would like to thank Rebecca Thornton, Marieke Kleemans, Richard Akresh, Adam Osman, Andy Garin, Ben Marx, and David Albouy, the seminar participants at UIUC's Applied Micro Research lunch, the seminar participants at the 2021 Development Day at Notre Dame.}}

\author{Vinicius Okada da Silva\thanks{Contact information: University of Illinois at Urbana-Champaign. Department of Economics, 1407 W. Gregory Drive, David Kinley Hall: Room 126, Urbana, Illinois 61801. E-mail: vo10@illinois.edu}}

\affil{Department of Economics, University of Illinois at Urbana-Champaign}

\date{}

\DeclareUnicodeCharacter{2212}{-}

\begin{document}

\maketitle
\thispagestyle{empty} 

\begin{abstract}
This paper aims to identify Jesuit missions' long-term impact on human capital and development in the Brazilian Amazon. Using Brazilian census data from 1872 and 2010 combined with a novel dataset on the location of Jesuit missions in the Brazilian Amazon, I find that places closer to the former missions had higher literacy rates in both periods. To estimate a causal effect estimate, I compare the impacts of the Jesuits against other missionary orders, the nearby Jesuit missions in the state of Maranhão, and I use an instrumental variable approach that considers the locations of Tupi-speaking tribes in the region. Further, demographic differences, the number of schools, or school attendance do not explain the results. This paper extends and reinforces the literature analyzing the positive effects of missionaries' transmission of human capital in the colonial period.
\end{abstract}


\clearpage
\pagenumbering{arabic} 

\section{Introduction}

\begin{displayquote}
``Everything [of political and economic nature in the Amazon is] related to the deeds, ideas, and goals of the Society of Jesus."
\\ 
\smallskip
- The Jesuits in Grao-Pará, \textcite[p.~9]{Azevedo1930-ir}
\end{displayquote}

During the Catholic Reformation in Europe, the Catholic church created several missionary orders to proselytize Christianity to the indigenous people of the recently found New World. 
In Brazil, the most prominent of these orders was the Society of Jesus, known as the Jesuits. 
While they were present in the entirety of Brazil, two areas of importance for their missionary work were the Southern region with the Guarani and the Amazon with the Tupi. 
The Jesuits were the first to codify local languages into a written form and would often be the first Europeans to establish mission settlements that eventually became cities.
Despite their roles as colonizers, serving the King of Portugal, the Jesuits often went against the local governments and settlers to protect the natives from enslavement or forced labor.

Given the colonization role that the Jesuits had in the Amazon, this paper aims to quantify the effects of Jesuit missions on the persistence of human capital development in the Brazilian Amazon by extending the analysis of \textcite{Valencia_Caicedo2018-gp} to the region\footnote{
  \textcite{Valencia_Caicedo2018-gp} analyses the effects of the Guarani Jesuit missions in Southern Brazil, Argentina, and Paraguay. A more recent paper by \textcite{Gomez-i-Aznar2022-pt} further corroborates the evidence, also analyzing the effect of Jesuit missions on the Gurani people with mission censuses and finds significant effects on the natives' numeracy.}. 
The colonization of the Amazon provides an ideal setting to study persistence. 
First, Amazon's large area and geographical isolation allowed the Jesuits to be relatively undisturbed until their expulsion. 
Second, colonization was slow in the Amazon compared to the rest of Brazil. 
By 1720, there were only one thousand Portuguese speakers in the region \parencite{Mufwene2014-gx}, with indigenous people making up a high share of the population. 
Third, to the present day, its inhabitants are both culturally and genetically connected to the former natives \parencites[p.~50]{Arenz2012-cc}{Parker1989-ul}{Souza2019-lr}. 
Lastly, the Amazon had an intensive missionary presence throughout its colonial period receiving a large proportion of the Jesuit missionaries in Brazil. 

To measure the impact of the Jesuits in the Brazilian Amazon, I combine Brazilian censuses with a novel dataset on the location of Jesuit missions in the Amazon.
The Jesuit locations are obtained from a map of Serafim Leite, a Jesuit priest.
Leite compiled the Jesuits' history in Brazil in ten volumes, two of which focused on the Amazonian region of Brazil.
The dataset contains the location of 60 former Jesuit missions in the present-day Brazilian states of Pará, Amazonas, Roraima, and Amapa. 
The geographical location of the Jesuit missions is then combined with the 1872 and 2010 Brazilian censuses at the parish and municipality levels, respectively, to show the effects of proximity to a Jesuit mission on human capital measures.

I exploit the location of the Jesuit missions similarly to \textcite{Valencia_Caicedo2018-gp} to estimate the effect of the Jesuits on human development measures. 
I initially address the endogeneity of the Jesuit missions' locations, all the key specifications control for a rich set of controls which include: mean elevation, mean slope, distance to the nearest river, distance to the coast, and potential calories from agriculture pre- and post- the Columbian exchange. 
The main estimates indicate that localities further away from Jesuit missions had significantly lower literacy rates in 1872 and 2010. Being 100km farther away from a Jesuit mission decreased the literacy rate by 3.7\% in 1872 and 2.5\% in 2010. 
In 2010, localities near Jesuit missions also had higher GDP per capita, with the services and industrial sectors causing this difference. 

I further address endogeneity with three other specifications. 
First, I compare the effects of the Jesuit missions in the neighboring state of Maranhão, where the Jesuits were unsuccessful in establishing themselves due to opposition from Portuguese settlers and the invasion of the region by the Dutch. 
I find positive and significant effects on the literacy rate for municipalities near Jesuit missions in the Amazon; however, proximity to missions in Maranhão negatively affects literacy. 
The results indicate that while successful in the Brazilian Amazon, the Jesuits did not have lasting effects on the neighboring state. 
Second, I compare the effects of proximity to a Jesuit mission with the proximity to other religious orders, the Carmelites and Franciscans. 
The results indicate no significant impact on the proximity to missions by other religious orders in 1872 and 2010.  
Lastly, I instrument the distance to a Jesuit mission with the distance to the nearest Tupi-speaking area obtained from \textcite{Clement2015-rf, Eriksen2011-cv}. 
The Jesuits developed a ``lingua-geral" based on Tupi-Guarani to communicate with the indigenous population. 
Indigenous tribes that spoke Tupi were easier to communicate with and convert, facilitating the establishment of missions in their land. 
The instrumental variable estimates indicate that in 1872 and 2010, being 100km farther from a Jesuit mission decreased literacy by 7.3\% and 2.6\%. 

The results also support historical evidence of how the Jesuits taught the indigenous people in their villages. 
Due to a royal decree, the \textit{Regimento das Missões}, indigenous men living in the missions spent six months away every year to work with either the government or Portuguese settlers \parencite{Hemming1987-vj}. 
Consequently, women and children were the ones who were more likely to benefit from the Jesuits' instruction. 
% To test whether the Jesuits' indeed had a differential impact on the literacy of men and women, I estimate heterogeneous effects 
When comparing the literacy rate between men and women, I find that in 1872 women had a higher literacy rate than men. 
The point estimates indicate that living in a parish 100km further to a Jesuit mission decreased women's literacy by 4.3\% for women while only 3.3\% for men.
Even though the results are not statistically different, they indicate a possible differential impact on literacy by gender, which would agree with the colonial system in place. 
However, in 2010 there were no more significant differences between men's and women's literacy. 

I also check for possible alternative explanations that could drive the effects on literacy. 
In 1872 the results were not driven by demographic differences. 
In 2010, neither population density nor an increased number of schools explain the results.
I further conduct several robustness checks, including using the distance through the rivers, representing the distance the early settlers would have to traverse; treatment assigned to localities within varying radii from a Jesuit mission; excluding localities too far away from a mission; excluding localities surrounding state capitals; and excluding localities too far away from rivers. 
The results remain significant and indicate the Jesuits' effect in the region. 
I also analyze to what extent the coefficients are stable to unobservables using the procedure specified in \textcite{Oster2019-qd}. 
The estimates indicate that controlling for additional unobservables would increase the magnitude of the coefficients making them more significant.
Following \textcite{Kelly2019-ln}, I address the potential for spatial autocorrelation by conducting a ``randomization inference" experiment. 
I generate random placebo missions and reestimate the coefficients on the proximity to them. 
If the Jesuit missions affected nearby localities,  randomly generated missions should not outperform the results. 
The 1872 and 2010 results are robust to the generation of the placebo missions, with the proportion of coefficients that are both more significant and bigger in magnitude being lower than 10\%. 

This paper contributes to the literature by extending the analysis of missionary orders in South America first conducted by \cite{Valencia_Caicedo2018-gp}. 
The author finds significant positive effects of the Jesuits' missions on literacy rate and income. 
The results in \textcite{Valencia_Caicedo2018-gp} are similar in magnitude to the ones in this paper\footnote{
  While the estimates between this paper and \textcite{Valencia_Caicedo2018-gp} are similar in magnitude, the context of the Jesuit settlement and expulsion were different. 
  In the Amazon, there was continuity in the missions' administration, as they were transferred from Jesuits to Portuguese officials. 
  In contrast, the Guarani missions had an uprising against the new Portuguese rule as the Spanish Guarani were afraid of being enslaved by Portuguese \textit{Bandeirantes}. 
  The historical difference here points to other mechanisms that allowed the Guarani missions to preserve their human capital after the Jesuits' expulsion and possibly the resilience of the introduction of human capital.}. 
By analyzing the Jesuits' effect on the Brazilian Amazon, I conduct, to the best of my knowledge, the first historical and economic analysis of missionary activity in the Brazilian Amazon.  
By extending the geographical coverage of missionary activity, the results of this paper provide additional evidence of the missionaries' impact during colonial times. Additionally, this paper builds on the literature that shows how historical events can have long-term persistent effects.

This paper also adds to the literature on the colonization of the Portuguese and Spanish Americas \parencite{Alston2022-fp, Barsanetti2021-hp, Franco2021-vn, Waldinger2017-rz, Dell2010-qt}.
These papers find how different colonial institutions, such as the mita in Peru or different types of missionaries in Mexico, caused various persistent effects. 
The paper also adds to the literature that explores the historical causes of human capital accumulation in Brazil, which analyses trade shocks and the development of elementary education \parencite{Musacchio2014-pq}.

This paper is also related to the broad literature on institutions and historical persistence through human capital accumulation that has been summarized in \textcite{Nunn2010-ls}. This paper also contributes to the literature on the persistent impact of missionaries during colonial times. Related literature studying missionary effects in Africa finds significant effects of missionary presence in either human capital, health, and democratic institutions \parencite{Cage2020-wc, Cage2016-kk, Wantchekon2015-ry, Nunn2014-tj, Woodberry2012-es, Nunn2010-ls, Gallego2010-bn}. These papers often distinguish between the impacts of Protestant and Catholic missionaries, finding significantly higher effects for Protestant missions.  Also related is the smaller literature of Christian missionaries in China, which find that both Jesuits and protestant missionaries led to technological and economic development \parencite{Ma2021-sb, Bai2015-kq}. 

The remainder of this paper is organized as follows. 
Section 2 provides some historical background on the Amazon's colonization and the Jesuits' role. 
In section 3, the data used is described. Section 4 describes the main empirical strategy used. 
In section 5, I show the results from the main specification.
In section 6, results for different specifications are described. 
In section 7, I provide several robustness checks. 
Section 8 concludes the paper. 

\section{Historical Background}

\subsection{Colonization of the Amazon and the Natives:}

The colonization of the Amazon began relatively late compared to the rest of Brazil. 
The central city in the region, Belém, was founded by the Portuguese only in 1616 \parencite{Chambouleyron2019-tm}. Several factors explain the relative lack of interest by the Portuguese in the region. 
First, settlements near the coast in the Northeast and Southeast provided easy transportation of goods from and to Portugal and its colonies in Africa. 
Second, settlement for traditional European-style agriculture and husbandry was difficult in the Amazon\footnote{The Jesuits tried introducing animals in the region; however, it was unsuccessful \parencite[p.]{Hoornaert1992-vx}.}. 
The dense tropical forest of the region hindered exploration, required effort to clear the land, and provided poor soil for plantations\footnote{
  Agriculture was viable near the rivers in the area called \textit{varzea}; however, that required knowledge of the river cycles which only the indigenous people knew \parencite[p.15-16]{Hoornaert1992-vx}.}. 
Lastly, the Amazon lacked precious metals such as gold or silver, which were the focus of the Portuguese Crown in their New World colonies. 
As a result, most settlements by Portuguese colonizers were located on the coast and alongside the fertile Tocantins valley \parencites[p.~44]{De_Assis_Costa2018-nt}{Chambouleyron2019-tr}.

While the Amazon lacked precious metals or agriculture, it offered one attractive commodity to willing Portuguese settlers: the indigenous labor force. 
Unlike the rest of Brazil, a key characteristic of the region was the importance of free and enslaved natives as part of the labor force \parencite{Chambouleyron2019-tm}. 
The Amazon was home to a large number of natives, who could be enslaved to work as labor for the colony\footnote{
  The Brazilian Amazon's indigenous population in pre-colonial times is estimated to be over two million, with a higher density of them located along the rivers of the region \parencites{Melatti2007-yn} {Denevan1992-qt}[p.~119]{Bethell1984-eo}}. 
Native labor was cheaper when compared to African slaves, which had high transportation costs.
A final benefit of indigenous people was that since the local economy was based on the extraction of resources and spices, they were the only ones who knew the region's intricacies \parencite{Parker1989-ul}.
Therefore, Portuguese colonists were willing to settle in the Portuguese Amazon despite its poor economic prospects by relying on native workers. 
Given the necessity of the Portuguese crown to settle in the region, it tried to organize and pacify the natives to make them more welcoming to future Portuguese colonizers.

%The Portuguese Crown, to attract settlers, needed to organize and settle the indigenous population. 

With the support of the Portuguese Crown, the Jesuits were invited to the Amazon to pacify the local indigenous people and fortify Portuguese borders. 
The Jesuits, formally known as the Society of Jesus, were established in 1534 by Saint Ignatius of Loyola during the Catholic Reformation in Europe. 
John II, King of Portugal, viewed the Jesuits highly, making them the pioneering missionaries in Portuguese colonies. 
The Jesuits arrived in Brazil in 1549 and quickly established several missions throughout the territory. 
The Jesuits were most successful in Southern Brazil with the Guarani tribes and in the North along the Amazon River and its tributaries \parencite[p.~3]{Hemming1987-vj}. 
The Jesuits' first arrival in the Amazon would be in Belém in 1653. 
The Amazon quickly became a focus of Jesuit missionaries due to its large native population.

Initially, the Jesuits established missions near the major cities of the Amazon. Logistics and accommodation impeded further Jesuit expansion into the dense tropical forest. However, conflict with settlers over the rights of indigenous labor led to the establishment of missions farther from the main colonial cities. The new missions would often be located along the colony's frontier along the Amazon river but isolated from European presence. Other than priests, non-indigenous people could not reside in the missions \parencite[p.~100]{Cardoso1984-ic}. 

As Portuguese dominance over the region increased, the settlers would eventually push the frontier and come in contact with the isolated Jesuit missions. 
Effectively, this increased conflicts between the Jesuits and Portuguese settlers\footnote{Fr. António Vieira was the most prominent Jesuit who lobbied the Portuguese Crown for the rights of the indigenous people in Brazil. By the pressure of Portuguese settlers, Vieira would be exiled from Brazil from 1661-1681. \parencite{Zeron2015-of}.}. 
To solve the conflict, the Portuguese Crown established the \textit{Regimento das Missões} in 1686. 
Its main goal delineated the power, responsibilities, rights, and jurisdiction of missionary orders and the colonizers. 
The \textit{Regimento das Missões} benefited the Jesuits by granting all religious orders complete political and spiritual control of their missions' natives. 
However, it also delineated specific areas under Jesuit jurisdiction, which included the southern bank of the Amazon river \parencite{Chambouleyron2019-tr}.

While the Jesuits were colonizers themselves, and the natives often were suspicious of their actions, they offered refuge from the harsher oppression from Portuguese settlers. The Jesuits primarily focused on preaching to the natives by reading and writing the Bible\footnote{The teachings would ideally happen daily in the local parish, present at every mission.}. 
The Jesuits also defended the rights of the natives, unlike the Portuguese settlers, who wanted to use them as slave labor.
While the regimented life under the missions was as physically demanding as working under Portuguese settlers, economically, the Jesuits were able to protect the natives by reinvesting any of their profits in their missions \parencite[p.~235]{Azevedo1930-ir}.

\subsection{Expulsion and Aftermath}

The Jesuits' presence in South America would end in 1759 with the rise of Sebastião José de Carvalho e Melo, also known as the Marquis of Pombal. Pombal rose to prominence on the Portuguese throne serving as a critical advisor to the king.  
Pombal was an avid anti-Jesuit who blamed them for the economic stagnation in the Portuguese colonies, especially in the Amazon. 
Pombal heard the settlers' complaints that the Jesuits were monopolizing the native labor, which led to the underdevelopment of the land in the region \parencite{Parker1989-ul}.
To overtake the power of the Jesuits, Pombal assigned his brother Paulo António de Carvalho e Mendonça to become the governor of the province of Grao-Pará and Maranhão. 
Pombal and his brother started a slandering campaign against the Jesuits, riling up the already agitated colonists against the Jesuits. 
In 1759, Mendonça would end the \textit{Regimento das Missões} and establish the \textit{Directorate of the Indians}, which lasted until 1798.

The Directorate effectively ended any Jesuit presence in Brazil during its colonial era.
All the religious orders with missions in the Amazon were expelled, while any farms and goods left over were confiscated. 
The expulsion of the religious order effectively removed direct Jesuit influence on the region's indigenous people. 
The Directorate gave the administration of the missions to local Portuguese government officials. 
Sixty former missions became villages under a new laic administration of Portuguese officials \parencite{Chambouleyron2019-tm}. 


The significant effect of the Jesuits on the Amazon cannot be understated. 
First, the Amazon was a place of intense missionary activity, with a quarter of the total Jesuit missionaries in Brazil living in the region before their expulsion \parencite{Bethell1984-eo}. 
Second, the Jesuits were able to create 60 missions along the rivers and develop a thriving economic base of spices from the forest \parencite{Hemming1987-vj}. 
Thirdly, the Jesuits economically developed the Amazon by introducing new technology, reinvesting capital into their missions, and teaching the indigenous people of their missions the basics of literacy. 
Lastly, the Jesuits created the first European-style settlements, which led to the development of the first urban centers that would eventually become some of the main cities in the region\footnote{
  At its height, the number of indigenous people living in Jesuit missions numbered 200,000 \parencite{Bethell1984-eo, Alden1996-id}.In comparison, the total population of the two states comprising the Amazon in 1872 was only slightly over 300,000.}. 


\section{Data}

The primary source of data for the Jesuit missions comes from \textcite{Leite1943-dy}. Serafim Leite was a Jesuit priest who compiled the history of the Jesuits in Brazil. 
Leite, based on official records, includes a detailed map of the Jesuit missions in Northern Brazil\footnote{
  The original map is available in \autoref{fig:SerafimLeite}}. 
The map is georeferenced using QGIS, giving a sample of 60 Jesuit missions in four states: Pará, Amazonas, Roraima, and Amapa\footnote{
  All four states were part of Pará or Amazonas during the colonial period. Acre is not in the sample since it was not a part of Brazil until 1903. There were no missions located in the states of Rondonia and Tocantins}.

Brazil's data for 2010 are obtained from IBGE, including literacy rate, GDP per capita, Gini coefficient, and urban population. 
Census data for 1872 is obtained from the Nucleus of Research in Economic and Geographic History from the Federal University of Minas Gerais\footnote{
  Available at \url{http://www.nphed.cedeplar.ufmg.br/}}. 
The 1872 Imperial Census contains demographic data at the municipality and parish level\footnote{
  For the 1872 census, a parish represents the smallest geographical unit available. Given the importance of the Catholic Church in Brazil, parishes were often the base for collecting data such as birth (baptism) and marriages.}.
Parishes are geo-located to present-day locations for a total of 90 observations for 1872. 

Shapefiles for the  Brazilian coast, navigable rivers, and municipality seats are obtained from IBGE\footnote{
  Historical municipality boundaries are from \textit{geobr} on R. The package is available at \url{https://cran.r-project.org/web/packages/geobr/index.html}}. 
Slope data comes from the European Environment Agency\footnote{
  Available at \url{https://www.eea.europa.eu/data-and-maps/data/world-digital-elevation-model-etopo5}}, and elevation comes from \textcite{Amatulli2018-gl}. 
Data on the maximum amount of calories based on pre-Columbian and post-Columbian crops are obtained from \textcite{Galor2016-ba}. 
The location of Franciscan and Carmelite missions in the region is obtained from \textcite{Bombardi2014-jf}. 
The location of pre-colonial Tupi-speaking areas is obtained from a map used in \textcite{Clement2015-rf, Eriksen2011-cv}.

Since the distance to a Jesuit mission is a continuous variable, I provide summary statistics considering a location treated if it is within 50km of a Jesuit mission in \autoref{tab:1872_summary} and \autoref{tab:2010_summary}. 
For 1872 and 2010, it is already possible to observe how places near the Jesuit missions have higher literacy than places farther away. 
I further plot unconditional literacy rates on distance to a Jesuit mission in \autoref{fig:Unconditional1872} and \autoref{fig:Unconditional2010}. 
Both graphs provide further evidence of how the proximity to the Jesuit mission plays a role in the literacy rate, as there is a negative trend of the distance to the nearest Jesuit mission on the literacy rate.

\section{Methodology}

\subsection{Main Specification}

I estimate the effect of proximity to a Jesuit mission on the outcome variables with the following specification, which follows from \textcite{Valencia_Caicedo2018-gp}:

\begin{equation}
\label{eqn:mainreg}
	Y_{i,s} = \beta \textit{DistanceToJesuitMission}_{i,s} + \gamma \textit{GEO}_{i,s} +  \alpha \textit{X}_{i,s} + \mu_s + \epsilon_{i,s}
\end{equation}

Where $DistanceToJesuitMisson_{i,s}$ is the main dependent variable of interest. It measures the Euclidean distance in kilometers from a unit of observation $i$ in a state $s$ to the nearest Jesuit mission. The coefficient of interest is $\beta$ which indicates the effect of being 1km away from a Jesuit mission on the outcome variable. If the Jesuits had a positive effect on the outcome variable,  $\beta$ would be negative, indicating that localities farther away from their missions have worse outcomes than localities near them. $GEO_{i,s}$ is a set of geographical control variables that include the area of a municipality, average slope, average elevation, distance to the coast, distance to the nearest navigable river, potential calories from agriculture, longitude, and latitude. $X_{i,s}$ is a set of control variables that include the century of creation of a municipality and a dummy indicating if the municipality is the capital of the state\footnote{
  The area and century of creation of the locations are only used as controls for the 2010 sample. Since parishes are only points, I am unable to calculate their areas. Similarly, since the parishes are not municipalities, I cannot track down their year of creation}. 
$\mu_s$ represents state fixed-effects. In all specifications, robust standard errors are in parentheses, while Conley standard errors are in brackets \parencite{Conley1999-gq}\footnote{
  Conley standard errors use a Bartlett kernel, and a distance cutoff of 400km are reported in brackets when possible}.

The identification of the regression assumes that conditional on the set of geographical variables, the location of the Jesuit mission was exogenous. While most of the controls included in the regression would likely be considered by the Jesuits when choosing their mission location, it might not fully capture their decision. Therefore, later I analyze several other specifications to address the missions' possible endogeneity. 


\section{Results}

\subsection{1872 Results:}

To capture persistence, I use the first available Brazilian census, the Brazilian Imperial census of 1872. 
Due to the small number of municipalities in the four states in 1872, I expand the sample size by georeferencing 90 parishes. 
If there are any persistent effects of the Jesuits on human capital, they would have to exist on the earliest possible date. 
Using the 1872 census also eliminates any historical events from 1872-to 2010 as possible explanations for the result\footnote{
  A possible concern addressed with the 1872 census was the Rubber boom and the drought in the Northeast, which led to massive immigration to the region \parencite{Parker1989-ul}. While the rubber boom began in the 1870s, it was not fully developed until later; additionally, the first wave of migration to the region due to the drought began only in 1877. Therefore, it is unlikely that either affected the 1872 results.}. 
The results of this section indicate that the Jesuit missions already affected literacy rates in 1872. 

\autoref{tab:1872_results} shows the estimates of \autoref{eqn:mainreg} using the 1872 census. 
A century after the Jesuits' expulsion, the results indicate a significant effect of proximity to a Jesuit mission on the literacy rate. 
Column 1 indicates that conditional only on state fixed effects and state capital dummies, there are no effects on the proximity. 
Once controlling for the parish's geographical characteristics, the results suggest that being 100 km further away from a Jesuit mission drops the literacy rate by 3.5\%. 
This result is economically significant since the mean literacy rate across parishes was only 20.9\%. 
Proportionally, the results indicate a 16\% decrease relative to the baseline mean literacy rate for a parish located 100km away from a former Jesuit mission.
Compared to the estimates on the illiteracy of \cite{Valencia_Caicedo2018-gp} for the 1920 census in Southern Brazil, the results are smaller in magnitude. 

Columns 3 and 6 explore alternative explanations, such as higher school attendance or higher presence of teachers. 
Parishes near Jesuit missions did not have a higher percentage of school-age children attending school but had a higher number of teachers per 10,000 population. 
The point estimate of column 6 indicates that being 100km away from a Jesuit mission would cause the number of teachers per 10,000 population to decrease by 3.3. 
Therefore, higher demand for education cannot explain the results, but it is not possible to rule out any effects caused by a higher supply of teachers. 

% The results of this section indicate that the Jesuits already had an effect on literacy in 1872

\subsection{2010 Results:}

The previous section provides evidence that the Jesuit missions were associated with higher literacy in 1872, over a century since their expulsion. 
This section aims to analyze whether their influence persisted for another century.

\autoref{tab:main_results} shows the results of the main specification using the 2010 census. 
The results indicate that municipalities further away from a Jesuit mission have lower literacy, with and without geographical controls.
The preferred specification in column 2, which includes geographical controls, indicates that having the municipality seat 100 km farther away from a former Jesuit Mission reduces the literacy rate by 2.5\%.
The estimated effects are similar to those found in \textcite{Valencia_Caicedo2018-gp} in which the estimated impact of the Gurani Jesuit missions in Southern Brazil increased illiteracy by 3.1\% per 100km. 
When comparing the 2010 results to 1872, the point estimate is lower than the 3.5\% found using the 1872 Census, indicating that the gap in literacy rate has decreased. 
Additionally, the estimated coefficient in 2010 is not as economically significant as the mean literacy rate across municipalities was 82.7\%. 
Therefore, while municipalities closer to the former Jesuit missions still exhibited higher literacy rates in 2010, its effect has diminished in magnitude and economic significance since 1872. 
Columns 3 to 6 compare the effect of Jesuit missions on inequality, measured by the Gini coefficient, and GDP per capita. 
The estimate in column 4 indicates that places farther away from Jesuit missions were more unequal in 2010. 
The results are unsurprising given that the Jesuits preached equality among its inhabitants, and profits were often reinvested into the community. 
In contrast, municipalities created by Portuguese settlers were more likely to have a concentration of wealth from plantation-style agriculture. Column 6 also indicates how municipalities closer to the former Jesuit missions are more prosperous than municipalities farther away, as measured by the GDP per capita. 
These results are similar to \textcite{Valencia_Caicedo2018-gp}, who finds increased income for municipalities in Southern Brazil near the Guarani Jesuit missions. 

Both the 1872 and 2010 censuses indicate that the Jesuit missionaries had a lasting effect on human capital development in the Amazon.
However, there are valid endogeneity concerns with the location of the missions, even after controlling for geographical characteristics. 
It is possible that such as the Jesuits chose locations based on a set of unobservables not considered in this section.
The following sections try to address endogeneity concerns that are not addressed in this section. 

\section{Other Specifications}

\subsection{Comparison with the Missions in Maranhão}

In addition to the Amazon, The Jesuits were interested in the neighboring state of Maranhão, where they built an important college in the capital of Sao Luis; however, they were not as successful as in the Amazon. 
Several reasons can explain the Jesuits' lack of success in the region. 
First, Maranhão was a region of contention between the Portuguese and the Dutch. 
While part of Portuguese Brazil, the Dutch conquered the capital Sao Luis from 1630 to 1654. 
As a result, the Jesuits in the region had to use the indigenous people in their missions as militias for several years to repel the Dutch. 
Second, the proximity of Maranhão to the coast made the arrival of settlers easier, which increased the state's population\footnote{
  Based on the 1872 Brazilian census, the state of Maranhão had a population of 359,040 in 1872, while Amazonas and Pará combined had 332,847.}. 
A higher density of settlers often led to conflict with the Jesuits, forcing them to create missions farther from the main colonies. 
The leading Jesuit priest in the region, Antonio Vieira, was arrested and removed from Brazil in 1655 at the request of colonial settlers \parencite{Leite1943-dy}.
Lastly, in the Amazon, the main economic activity was the extraction of the \textit{drogas-of-sertao}, which required the cooperation of the natives, while in Maranhão, cattle ranching was the basis of the economy \parencite{Chambouleyron2019-tm}. 

Given the historical differences in the Jesuit presence in the Amazon and Maranhão, the following equation estimates the differential impact of the Jesuits using the 2010 census:

\begin{equation}
\label{eqn:Maranhão}
\begin{split}
	Y_{i,s} = & \beta_1 \textit{DistaceToJesuitMission}_{i,s} + \beta_2 \textit{MA}_{s} + \beta_3 \textit{DistaceToJesuitMission}_{i,s} \cdot \textit{MA}_{s} \\ & + \gamma \textit{GEO}_{i,s} +  \alpha \textit{X}_{i,s} + \mu_s +  \epsilon_{i,s}
\end{split}
\end{equation}

The coefficient $\beta_1$ indicates the effect on the outcome variable from being 1 km farther away from a Jesuit mission. 
The coefficient $\beta_3$ indicates the effect of being 1 km farther away from a Jesuit mission for the municipalities located in Maranhão. 
If the Jesuits had any effect on the outcome variables but were unsuccessful in Maranhão,  $\beta_1$ would be negative while $\beta_3$ would be positive or insignificant.

The results of the previous hypothesis are found in \autoref{tab:main_results_MA}. 
In column 2, the point estimate of $\beta_1$ is negative and significant, indicating that being 100km away from a former Jesuit mission is associated with a drop of 1.6\% in literacy. 
The point estimate is lower than the 2.5\% estimated from \autoref{tab:main_results}; however, they remain statistically significant.
The estimate for $\beta_3$ is positive and significant, indicating that being farther away from a Jesuit mission increases literacy for municipalities in Maranhão. 
Both coefficients, when combined, indicate that while being closer to a Jesuit mission is associated with a higher literacy rate per municipality in 2010, these results are only present in municipalities part of the Brazilian Amazon.


Similar to \autoref{tab:main_results}, being closer to a former Jesuit mission is associated with a decrease in inequality with no differences between regions. 
Comparing the effects on GDP per capita on \autoref{tab:main_results_MA} with  \autoref{tab:main_results}, the effect of the Jesuit missions on GDP per capita becomes insignificant. 
However, the interaction term indicates that being farther away from a Jesuit mission in Maranhão increases GDP per capita. 
Therefore, the estimates in column 6 indicate that the Jesuits did not cause an increase in GDP per capita in places close to it; however, in  Maranhão, municipalities near the former Jesuit missions are, on average poorer. 

The results of this section indicate that while the Jesuits were successful in the Amazon in generating long-lasting human capital development, they were not as successful in the neighboring state of Maranhão. 
The statistical results align with the historical record of the different conditions the Jesuits had to work under in both regions.
Therefore, the mere Jesuit presence was not enough to establish human capital, but only under necessary conditions were the Jesuits able to develop their missions. 
The Jesuit missionaries only found those conditions in the Amazon, but not in Maranhão.

% In Maranhão, the Jesuits were successful for two main reasons: the need to reinvest resources in a war against the Dutch; and its proximity to the coast, which allowed more interaction between the Jesuits and the local settlers. Unlike in the Amazon, the Jesuits did not have the necessary isolation to develop their missions and instruct the indigenous people. Therefore, while the Jesuit presence was necessary to develop human capital in the Amazon, it was also necessary that they were under the right conditions to proselytize and teach. 

\subsection{Comparison with Carmelites and Franciscans:}

In 1686, under the decree of the King of Portugal, the \textit{Regimento das Missões} was established to determine the role of different missionary orders in the Amazon. 
A vital aspect of the \textit{Regimento das Missões} was the assignment of missionary activity to non-Jesuit religious orders. 
Since the Jesuits were being overstretched and running out of supplies for their missions, the \textit{Regimento das Missões} allowed different missionary orders to create missions and even overtake some of the Jesuits.

Two of the most successful religious orders introduced were the Franciscans and the Carmelites. 
The Franciscans arrived in the Amazon earlier than the Jesuits; however, they never created as many missions in their early years.
After the establishment of the \textit{Regimento das Missões}, the Franciscans took over some successful Jesuits' missions and the profitable cattle farms on the Island of Marajo\footnote{
  While I refer to them as Franciscans, the name of the order was the Capuchins of St. Anthony, a branch of the main Franciscan order.}. 
They also established their own missions north of the Amazon river.
The Carmelites gained exclusive rights to establish missions along the Negro and Branco rivers in Western Amazon. 
The Carmelites eventually built successful missions such as Barcellos, which became the capital of the Captaincy of Rio Negro \parencite{Perdigao2020-fk}. 
While successful, Franciscans and Carmelites were more traditional monastic orders; therefore, they preferred to live in poverty, in sharp contrast to the Jesuits, who focused on developing their missions. 
As a result, Franciscans and Carmelites never invested much in the human capital formation of the indigenous people living in their missions \parencite[p.~288]{Azevedo1943-gs}.

Following \textcite{Valencia_Caicedo2018-gp}, I compare the effect of the Jesuit missions with missions established by different orders.
Specifically, I compare the impact of the distances to the nearest Jesuit mission with the distance to the nearest Carmelite/Franciscan mission on the literacy rate. 
Other missionary orders would want to establish their missions in suitable locations.
Therefore, when comparing within missionary orders, this specification controls for possible unobservables not captured in the previous regressions.
The locations of the 21 Carmelites and Franciscans missions are in \autoref{fig:NonJesuitMissions2010Map}, obtained from \textcite{Bombardi2014-jf}. 

First, I estimate the impact on the distance of the non-Jesuit missions by themselves with the following equation:

\begin{equation}
\label{eqn:otherreg}
	Y_{i,s} = \theta \textit{DistanceToNonJesuitMission}_{i,s} + \gamma \textit{GEO}_{i,s} +  \alpha \textit{X}_{i,s} + \mu_s + \epsilon_{i,s}
\end{equation}

Where $\theta$ indicates the effect of being 1km farther away from a non-Jesuit mission on the dependent variable, if the Carmelites and Franciscans built their missions in areas that gave long-run economic benefits, the coefficient $\theta$ should be negative. 
In order to compare the differential impact of Jesuit and non-Jesuit missions, I also estimate the following specification: 

\begin{equation}
\label{eqn:bothreg}
\begin{split}
	Y_{i,s} = & \beta \textit{DistanceToJesuitMission}_{i,s} + \theta' \textit{DistanceToNonJesuitMission}_{i,s} \\ & +  
	\gamma \textit{GEO}_{i,s} +  \alpha \textit{X}_{i,s} + \mu_s + \epsilon_{i,s}
\end{split}	
\end{equation}

$\beta$ estimates the effect of being 1km farther from a Jesuit mission on the dependent variable. 
$\theta'$ measures the effect of being 1km farther from a Carmelite or a Franciscan mission on the literacy rate. 
Suppose the proximity to a Jesuit mission had a different positive effect on the literacy rate than the Franciscans and Carmelites. 
In that case, $\theta' > \beta$, as a more negative coefficient, indicates a stronger decrease in literacy as a locality becomes farther away from a mission.
I further compare if $\beta$ and $\theta'$ are different through an F-test of equality of coefficients.

In 1872, as shown in \autoref{tab:robustness_placebo1872_results}, column 2 indicates that the Franciscans and Carmelite missions had no significant effects on the literacy rate. In column 4, we still see an insignificant effect of the non-Jesuit missions. However, the coefficient on the distance to the Jesuit mission remains significant, with the point estimate of -0.037 being identical to the one found in column 2 of \autoref{tab:1872_results}. However, it is not possible to reject that the coefficients are different from each other as the F-statistic p-value is 0.176. 

The estimate for 2010 are in \autoref{tab:robustness_placebo_results}. 
In 2010, the results for the Carmelites and Franciscans are insignificant in column 2, once again indicating that their presence did not affect literacy. 
The coefficient remains insignificant in column 4, while the point estimate for the effect of the Jesuit missions of -0.025 is negative, significant, and almost identical to the one in \autoref{tab:main_results}. 
Given the point estimates, it is possible to reject the equality of the coefficients between the distance to the nearest Jesuit mission to the distance to the nearest non-Jesuit mission.

The results of this section indicate that even though other religious orders established missions in the Amazon, only the ones established by the Jesuits have long-term evidence of impacting literacy rates. The results corroborate the historical evidence that during the early colonial period of Brazil, the Jesuits were the primary teachers to both Portuguese settlers and the indigenous people.

The results of this section can be contrasted to \textcite{Waldinger2017-rz}, who finds that in Mexico, municipalities that had Mendicant missionaries have higher literacy. 
In contrast, municipalities with Jesuit missionaries had lower or insignificant effects on literacy. 
The non-existent effect of Jesuits in Mexico can be attributed to the Jesuits' focus on teaching the Spanish youth to control the local elites \parencite{Waldinger2017-rz}. 
In the Brazilian Amazon, the Jesuits focused on teaching the indigenous people in their missions. 
The results are also related to \textcite{Valencia_Caicedo2018-gp} where the author finds that illiteracy decreased farther from Franciscan missions. At the same time, illiteracy increased farther from the Jesuit missions. 

This section's results measure the Jesuits' unique impact on the literacy rate. 
Unlike the previous literature, there are no adverse effects from the other missionary orders. 
Franciscans and Carmelites successfully established missions in the Amazon and even took over some of the Jesuits in the Amazon. 
However, since their approach to evangelization was different from the Jesuits, the results do not indicate any long-lasting effect on human capital in localities near their former missions.
Following \textcite{Valencia_Caicedo2018-gp}, all the following regressions, unless specified, will include the distance to the nearest non-Jesuit mission as a control. 

\subsection{Instrumental Variable:}

Brazil had a large variety of local languages by the time of colonization\footnote{For example, during his travel up the Amazon River, a Spanish Jesuit missionary came in contact with over 150 different languages \parencite{Mufwene2014-gx}.}. 
Given the variety of local languages, the Jesuits needed a language to communicate with the indigenous people. 
The natural choice was the Tupi-Guarani language family, which the Jesuits and Portuguese settlers came in early contact with as it was the most common language spoken by the natives who lived on Brazil's coast \parencite{McGinness2018-ig}. 
As a result, in 1595, the Jesuit Fr. Jose de Anchieta created the \textit{lingua-geral} based on Tupi-Guarani to communicate with the indigenous people. 
The \textit{lingua-geral} was the first effort to create an orthography for Tupi-Guarani, and it provided a written and standardized form that would be usable for future Jesuits to communicate with the indigenous people of South America \parencites[p.~192]{Newson2020-vg}{McGinness2018-ig}\footnote{
  \textit{General Language}, references how it was a catch-all term for the variety of languages it would be used to communicate}\textsuperscript{,}\footnote{
    The \textit{lingua-geral} was so successful, especially in the Amazonian region, that it eventually became the \textit{lingua-franca} of the region and remained so until the Portuguese government pushed Portuguese to become the official language \parencite{Chambouleyron2019-tm}}.

I exploit the language barrier at the time of colonization between the Jesuits and the indigenous people as an instrument to estimate the Jesuits' causal effect on the literacy rate. 
The exclusion restriction assumes that the language spoken in an area is only associated with the outcome variable through the proximity to a Jesuit mission. 
I first show in \autoref{tab:tupi_economic} that proximity to Tupi-speaking areas does not affect GDP per capita; therefore, the former Tupi areas are not better off economically.
Second, while there could be language spillovers, the Tupi was solely a spoken language in which the Jesuits introduced the written form. 
As a result, any effects on literacy have to come from the Jesuits' teaching in their missions.

Of importance is that the \textit{lingua-geral} was created before the arrival of the Jesuits in the Amazon\footnote{
  As previously mentioned, the \textit{lingua-geral} was finalized in 1595, and the Jesuit arrival to the Amazon only began in 1653.}. 
Instead, the Jesuits developed the \textit{lingua-geral} based on their interactions with the tribes on the Brazilian coast. 
Therefore, it was not the case that the Jesuits constructed this language to help them create their missions in the Amazon. 
Also noteworthy is that missionaries faced difficulties communicating with indigenous people who did not speak Tupi. 
For example, when non-Tupi speaking natives decided to settle in the missions, they “could not understand the catechisms, nor could those schooled in Tupinamba grammar understand the indigenous speakers” \parencite{Mufwene2014-gx}. 
The Jesuits also needed the cooperation of the indigenous people to establish missions since the number of priests was limited. Most established missions arose from cooperation between the Jesuits and the indigenous people.
Therefore initial communication was essential when deciding where to establish a mission. 
Finally, given the importance of interrelation between tribes, befriending a Tupian tribe allowed the Jesuits to be better received by their allies, making contact with a new tribe safer and easier \parencite{Reeve1993-wy}.

An estimation of the location of the Tupi-speaking tribes' pre-colonization is obtained from \textcite{Clement2015-rf, Eriksen2011-cv} who provide archeological estimates of the distribution of the main languages in the Amazon\footnote{The other main language families other than Tupi are: Arawakan, Carib, Macro-Ge, Panoan, and Tucanoan. The most prominent ones are the former three. The Arawakan were located along the Negro River and Western Amazon, the Carib were located along Northern Amazon, and Macro-Ge were near the Tupi areas of Eastern Amazon. For more information on their locations, the entire map of their estimated geographic distribution is available in \textcite{Clement2015-rf}.}.
I then estimate the Euclidean distance from each location to the nearest Tupi-speaking area\footnote{
  The areas that were inhabited by Tupi-speaking tribes can be found in \autoref{fig:TupiSpeaking}}.

The first stage regression is:
\begin{equation}
\label{eqn:firststage}
	\textit{DistanceToJesuitMission}_{i,s} = \beta \textit{DistanceToTupiArea}_{i,s} + \gamma \textit{GEO}_{i,s} +  \alpha \textit{X}_{i,s} + \mu_s + \epsilon_{i,s}
\end{equation}

The second state regression is:

\begin{equation}
\label{eqn:ivequation}
	Y_{i,s} = \beta \widehat{\textit{DistanceToJesuitMission}_{i,s}} + \gamma \textit{GEO}_{i,s} +  \alpha \textit{X}_{i,s} + \mu_s + \epsilon_{i,s}
\end{equation}

Results using the distance to a Tupi-speaking area as an instrument can be found in \autoref{tab:1872_iv_results} and \autoref{tab:2010_iv_results}. Confirming the historical records that the Jesuits created \textit{aldeias} in Tupi-speaking areas, there is a strong correlation between the distance to a Tupi-speaking area to the distance to a Jesuit mission. 
The first-stage results indicate that the Jesuits placed their missions closer to areas where the indigenous people spoke Tupi. 
The F-statistic is above 20 in 1872 and 2010, indicating that the instrument does not lack explanatory power. 

For 1872, the IV estimator from \autoref{tab:1872_iv_results} indicates that being 100 km farther from a Jesuit mission results in a drop of 7\% in the literacy rate. 
Compared to \autoref{tab:1872_results}, the point estimate is almost doubled. In 2010, the IV estimator found in \autoref{tab:2010_iv_results} indicates that being 100km away from a mission is associated with a drop of 2.5\% in literacy. 
The point estimate is almost the same compared to \autoref{tab:main_results}. 
The 1872 and 2010 results still indicate a stronger effect in 1872 that has faded away in 2010, both being significant. 

This section strengthens the previous OLS results found in \autoref{tab:1872_results} and \autoref{tab:main_results} by arguing for a more causal estimate of the Jesuit presence in the region. 

\subsection{Gender Effects:}

Another consequence of the \textit{Regimento das Missões} was the division of indigenous labor between the Jesuits and the Portuguese Crown. 
The assignment of natives to work outside the missions broke down the Jesuits' monopoly over indigenous labor. 
The \textit{Regimento das Missões} established that two-thirds of the male natives in the \textit{aldeias} had to work for the Portuguese Crown and the Portuguese colonists. 
Given that a large part of the male population of the missions would be away for significant periods, the Jesuits were then only able to teach women and children\footnote{
  Estimates indicate that during the Directorate, the indigenous men part of the former missions would spend 64\% of the time working in expeditions for spices, either for the government or the settlers \parencite{Hemming1987-vj}. 
These estimates also represent the time spent away during the Jesuits' presence in the \textit{Regimento das Missões}.}. 
With women being exposed the most to the Jesuits' teaching, it would be expected that the effects on literacy would be more significant among women, at least shortly after their expulsion. 

Results for the effects of literacy by gender in 1872 are in \autoref{tab:1872_gender_results}. 
The regression estimates corroborate that while men and women near Jesuit missions had higher literacy, women had more substantial effects. 
For women, the estimated effect of living in a parish 100km away from a Jesuit mission is associated with a 4.4\% drop in literacy; for men, the effect is 3.2\%
The gender differences are more pronounced when compared to the means by each group.
Relative to the baseline literacy of each gender, 27.2\% for women and 14.6\% for men, the results would indicate being 100km away from a Jesuit mission had an increase of 30\% in the literacy rate for women. In contrast, for men, the effect is 12\% of the mean.
This result is similar to \textcite{Valencia_Caicedo2018-gp}, who finds that in the 1895 Argentinian census, the Guarani Jesuit missions had a more substantial effect on female literacy than male literacy. 

In 2010, however, the difference between the point estimates is less pronounced, with larger effects for men. The point estimates indicate that the effect of being 100km farther from a Jesuit mission is associated with a decrease of 2.8\% and 2.2\% in literacy rates for men and women. The estimates in \autoref{tab:2010_gender_results} indicate that the Jesuit's differential effect on literacy has faded from 1872 to 2010. However, it should be no surprise to observe the convergence in literacy between men and women. After the Jesuit expulsion and the Directorate period, both men and women would have access to the same basic levels of education. 

\subsection{Alternative Explanations:}
\subsubsection{1872:}

In \autoref{tab:1872_demo_results}, I check whether there are any demographic differences between the parishes in 1872 that could explain the results. 
The point estimates indicate that the demographics of places closer to Jesuit missions are not statistically different from places farther away. 
There are no significant effects on the proportion of Brazilians from other states, immigrants, caboclos, whites, or pardos\footnote{
  Caboclos means a mix between an indigenous person and a white person. Pardos are a mix between Africans and white people.}. 
The only observed difference is that places farther away from missions have fewer slaves relative to the free population. The point estimate indicates that being 100km away from a Jesuit mission would decrease the slave population of an 1872 parish by 1.8\%. 
Given that there are no demographic differences among non-slaves, I can discard the possibility that the effects on the literacy rate found previously are being driven by more educated Portuguese migrants inhabiting the parishes near Jesuit missions. 
The significant results on the proportion of enslaved people could indicate that places near Jesuit missions were already more economically active and required a higher share of African slave labor in their economy.

\subsubsection{2010:}
For the 2010 census, I consider whether urbanization, population density, or the supply of schools could be possible explanations for the effects on human capital from \autoref{tab:main_results}. 
Urbanization and population density are evidence of agglomeration and cities' development, which could lead to higher human capital by channels other than the Jesuit presence. 
If municipalities near the Jesuit missions also have more schools, that could explain higher literacy effects.

The results on the effect of proximity to a Jesuit mission on urban population, density, and school supply are available in \autoref{tab:alternative_exp_2010}. 
The results indicate that municipality seats closer to former Jesuit missions are more urbanized, with a decrease of 4.1\% in the urban population for every 100km. However, there is no impact on population density. 
The results would then indicate that while municipalities closer to former Jesuit missions had a higher proportion of people living in cities, it is not enough to affect the population density. 
The population density results of this section are similar to \textcite{Valencia_Caicedo2018-gp}, which finds no impact on population density in Southern Brazil. 
Similarly to the context of \textcite{Valencia_Caicedo2018-gp}, the Jesuits built some of their missions in the Amazon's interior to seek isolation and avoid settler conflict and interference.

When measuring the supply of schools, the results indicate that places farther away from a Jesuit mission have more schools per 10,000 inhabitants and more schools per 10,000 for people under 15. 
The point estimates show that being 100km away from a former Jesuit mission increases the municipality's number of schools per 10,000 people by 0.2 while the number per 10,000 people under 15 by 0.41. 
Considering that the mean population across the municipalities in 2010 was 51,000, the effects would indicate that, on average, being 100km farther from a Jesuit mission decreases the number of schools by 1 per municipality. 
Therefore, the difference in the literacy rate in 2010 cannot be explained by a higher supply of schooling available.
If anything, municipalities near the former Jesuit missions have increased literacy despite having fewer schools.

I further analyze what sectors drive the GDP-per capita results from \autoref{tab:main_results}. 
I find no differences in the agricultural and government spending sector; however, similar to \textcite{Valencia_Caicedo2018-gp}, I also find that the effects are present only in the industrial and services sector. 
These results indicate how human capital accumulation also leads to specialization in sectors that require significant investments in human capital.


\section{Robustness Checks}

\subsection{Distance by River:}

Given the importance of the rivers during the early colonial period of the Brazilian Amazon, the choice of the Euclidean distance might not be indicative of the proper distance the Jesuits would have to traverse when creating their missions. In this section, I consider the effects of the distance from a location to the nearest Jesuit mission through the main navigable rivers of the region. 

%\autoref{fig:RiverDist1872} and \autoref{fig:RiverDist2010} show the relationship between the Euclidean distance and the distance through rivers for both 1872 and 2010. Overall, there is a strong positive correlation between both measurements of distance. Places near the Jesuit missions measured by the Euclidean distance remain close as measured by the distance through the rivers. However, the distance through the river increases significantly for observations with a high Euclidean distance.

Results using the distance by a river as the main independent variable are in \autoref{tab:river_network}. For 1872, the point estimate decreases from -0.037 to -0.027 while remaining statistically significant. The results indicate that increasing the distance from a parish to a Jesuit mission through the river network decreases the parish's literacy rate by 2.7\% for every 100km. Therefore, using the distance to a mission through the river decreases the estimated effect of a Jesuit mission by 1\% per 100km, with the result statistically and economically significant. For 2010, there is a three-quarters drop in the estimated coefficient from -0.025 to -0.007. The coefficient indicates that the effect of being 100km farther away from a Jesuit mission through the river network is associated with a drop of 0.7\% in a municipality's literacy rate. The result remains statistically significant, however. This section indicates that using a different type of distance measurement still yields significant results for the Jesuits on literacy even though the point estimates decreased.

\subsubsection{Excluding Municipalities too far away from Missions}

Municipalities too far away from Jesuit missions are not likely the best control group. Municipality seats located too far away from any Jesuit missions might be naturally more isolated and less developed, leading to lower literacy rates. Another concern is that those municipality seats are too far away are too different from the places close to the Jesuit missions. I address this issue by estimating \autoref{eqn:mainreg} for 2010, selecting only observations within a cutoff difference from a Jesuit mission.

\autoref{fig:JesuitMissions2010Map} shows the point estimates along with the 95\% confidence intervals with robust standard errors for different cutoff distances. First, it is important to note how all the point estimates are negative and centered around the main estimate from \autoref{tab:main_results}, indicated by the red dashed line. The coefficients are insignificant for the subsample of observations between 50 and 100 kilometers away from a Jesuit mission. However, the coefficients become significant once I consider observations within 150 kilometers and remain significant throughout. Therefore, it is not the case that extreme outliers are not driving the results. Neither is the case that considering a subsample more representative of the treated changes the results.

\subsection{Distance from a Capital}

Another possible concern for the 2010 results is that many smaller municipalities are located around the capitals of each state. Therefore, they would make a greater share of the observations closer to the capital. Therefore, municipalities near the capitals could cause biased estimates as the capitals of each state are historically more developed. I address this issue by estimating \autoref{eqn:mainreg} but excluding any municipalities within a cutoff distance of any state capital. 

Results are found in \autoref{fig:CapitalDistances}. All the point estimates are negative for all cutoffs. 
Overall, the results remain significant at the 95\% confidence intervals until we exclude all observations within 300km of a state capital. 
Therefore, it is not the case that the increased number of municipalities surrounding the capitals of each state is driving the results from the main specification. 
Neither is the effect of proximity to a capital being associated with an increase in the development of nearby municipalities explaining the results.

\subsection{Distance from a River}

Even though all specifications directly control for the distance to the nearest river, a possible concern is that municipalities too far away from any rivers are biasing the results. 
Given the importance of rivers during the early colonial period, localities too far away from rivers would not be accessible for Jesuits and settlers alike. 
Therefore, a better control group to compare would be all municipalities nearby rivers, as the Jesuits could have accessed them. 
I estimate \autoref{eqn:mainreg} by excluding municipalities that are too far from rivers. 

Results are found in \autoref{fig:RiverDistances}. Even when considering municipalities within 10km of a river, the point estimate remains statistically significant and close to the estimated value from \autoref{tab:main_results} of -0.025.
The point estimates are also consistent throughout the distances from 10-200km. 
The results indicate that selecting municipalities near the rivers, which would all have been possible to be settled by the Jesuits, is not a significant factor in explaining the results. 

\subsection{Redefining Treatment}

I consider the following specification for 2010:

\begin{equation}
\label{eqn:treated}
	Y_{i,s} = \beta \textit{Treated}_{i,s} + \gamma \textit{GEO}_{i,s} +  \alpha \textit{X}_{i,s} + \mu_s + \epsilon_{i,s}
\end{equation}

Where $Treated_{i,s}$ is a dummy variable that equals one if the municipality seat is within a certain distance of a Jesuit mission. 
The coefficient $\beta$ then estimates how much higher the literacy is for municipalities within the cutoff distance to the Jesuit mission than those outside the cutoff distance.

Results for this specification are found in \autoref{fig:JesuitDistances}. 
The point estimates indicate that being within 10km of a former Jesuit mission is associated with an increase in literacy rate of 5\%. 
The coefficient decreases but remains significant and settles around 3\% when considering treatment being within 20-100 kilometers of a former Jesuit mission. 

\subsection{Other Censuses}

While the results between the earliest census, 1872, and the most recent one, 2010, provide evidence of the persistent effect of the Jesuit missions, I further provide evidence that these effects existed in the periods between them.
I add literacy rate estimations for the censuses of 1970, 1980, and 1991.

The results can be found in  \autoref{tab:all_years}. 
All point estimates are negative for all three censuses; however, for 1970, the coefficient is not significant. 
Also interesting is how the coefficients for 1991 are bigger than the estimates of 1872, possibly indicating other historical factors at play at this time, even though, based on the standard errors, it is not sufficient to reject that the coefficients are that different. 

A possible explanation for the results is that during both 1970 and 1980, Brazil was under a military dictatorship that heavily invested in expanding the highway network system in the Amazon. 
In the 1970s, the first main highway, the Transamazonica, was built, connecting the Northeast to the center of the Amazon.
The construction of highways in the region led to a large wave of migration to the region. 
Therefore, during the 1970s, there was a period of structural reformation in the region, in which new migrants settled in new areas expanding the population and possibly diffusing the results of 1970.

\subsection{Oster Bounds:}

Following \textcite{Oster2019-qd}, I analyze the possible impact of unobservables in the results\footnote{
  \textcite{Oster2019-qd} develops a procedure to analyze how stable the point estimates are from omitted variables. 
  First, the regression with only Y and X is estimated, resulting in an estimate of $\mathring{\beta}$ and $\mathring{R^2}$. 
  The regression is then estimated with controls to get $\tilde{\beta}$ and $\tilde{R^2}$. Given these results and the maximum $R^2$ and $\delta$, the bias-adjusted treatment is estimated as: $ \beta = \tilde{\beta} - \delta[\mathring{\beta} - \tilde{\beta}] \frac{R_{max} - \tilde{R}}{\tilde{R} - \mathring{R}}$}.
Since several omitted variables could explain why the Jesuits chose their locations, using \cite{Oster2019-qd} allows me to understand how stable the coefficients are to unobservables.

The coefficients for the regressions including all the geographical controls for 1872 and 2010, become $-0.089$ and $-0.053$, respectively\footnote{using a $R^2_{max}$ of $1.3 \cdot R^2$}. 
Through this methodology, unobservables are likely to push the coefficients away from zero, making the results more significant. 
The coefficient for the 1872 regression is consistent with \textcite{Oster2019-qd}, being within $2.8 \cdot SD$ of the original coefficient found in \autoref{tab:1872_results}; however, that is not the case for the 2010 results.

\subsection{Randomization Inference:}

I also conduct three “randomization inference” exercises by generating several placebo missions and comparing them to the literacy rate results in 1872 and 2010. 
This section addresses the concerns presented with spatial autocorrelation found in \textcite{Kelly2019-ln}. 
\textcite{Kelly2019-ln} argues that spatial autocorrelation decreases standard errors making coefficients more significant.


Following \textcite{Kelly2019-ln}, I generate random spatial noise that preserves the spatial autocorrelation by generating placebo missions. 
I create 60 random points within the limits of the study area and treat them as placebo missions. 
The distance from each locality to the nearest placebo mission is calculated and used as the new independent variable in \autoref{eqn:mainreg}. 
Given that the distances are now random, with no spatial autocorrelation, the probability I find results as significant should match the probability given by the t-statistic in the regression. 
However, by construction, the distance to the nearest placebo mission will be spatially correlated as two locations near each other will have similar values for the distance to the nearest mission.

These random placebo missions should not outperform the main regression results; otherwise, the significance of the results is caused by places near each other having similar values. 
I try three different methods of generating random placebo missions:

\begin{itemize}
\item Generating placebo missions randomly within the area of study
\item Generating missions given the latitude and longitude of the actual missions in the region
\item Generating missions within a 10km boundary of the rivers in the region since the Jesuits could only setup missions close to rivers
\end{itemize}

A set of placebo missions is considered to outperform the main specification if the coefficient on the distance is both stronger in magnitude and more significant. Maps of the region showing how the placebo missions were distributed across each specification are in \autoref{fig:PlaceboMapv2},\autoref{fig:PlaceboMap}, and \autoref{fig:PlaceboMapRiver}.

\autoref{tab:ri_placebo_results_1872} and \autoref{tab:ri_placebo_results}. First, it is important to note that randomly generated missions give significant results at the 0.1\% level in almost 50\% of the regressions, which indicates that the standard errors cannot be taken at face value. 
In 1872, the results show that the significance of the results should have been between 0.094 to 0.182, depending on how the placebo missions were generated. 
For 2010, the significance of the results was around 0.074 to 0.01, which are higher than the actual significance in the main regression; however still significant at a 10\% confidence level. 
Overall, I cannot fully reject spatial autocorrelation causing the significance of the results in 1872 for all three types of randomization.
However, for 2010, the results are still significant at a 10\% confidence level.

\section{Conclusion}

This paper shows the persistent effects on human capital through literacy associated with the Jesuits in the Amazon. 
I extend the analysis from \textcite{Valencia_Caicedo2018-gp} to the Amazon, a region that, similarly to Southern Latin America, had both a strong and successful missionary presence. 
The results indicate that being closer to a Jesuit mission is a strong predictor of literacy rate in 1872 and 2010\footnote{
  It is important to note that while the Jesuits seem to have generated positive human capital impacts on the regions nearby their missions, they served a colonial empire. This paper does not claim that Brazil's colonization was advantageous to the natives, especially since many of them succumbed to diseases and enslavement.}. 
While significant, the disparities between more distant places decreased across the 128 years between the samples. 

While we should not interpret the results directly as causal, I implement several methods to narrow the mechanism to the Jesuit presence:
First, all the specifications include a rich set of geographical variables that control for potential differences in geography and the region's suitability for settlement. 
Second, the Jesuits had no effect in the neighboring state of Maranhão, indicating that the mere Jesuit presence was insufficient to create long-term development. 
Third, the proximity to Carmelite or Franciscan missions cannot explain the results. 
Lastly, the results remain significant when using the distance to a Tupi-speaking area as an instrumental variable.

I analyze other possible channels that could explain the results.
For 1872, the results are not driven by demographic differences or higher school attendance.
Interestingly,  the impact of the Jesuits is more pronounced for women than men. 
Historically, the male population had to spend time away from the mission working either for settlers or the government making women the more traditional recipients of Jesuit education. 
For 2010, higher government spending, higher population density, or a higher number of schools in municipalities close to the missions cannot explain the results. 
Places near the Jesuit missions in 2010 are also more urbanized and have a higher GDP per capita, with differences only in the industry and services sector.

The results remain robust for other specifications, and there is still evidence of persistence during 1980 and 1991, but surprisingly not in 1970.
Using Oster bounds indicates that if controlling for additional unobservables, the coefficients estimated would become more negative. 
Finally, the randomization inference exercise provides evidence that the results for 2010 are not caused by spatial autocorrelation. 
However, I cannot reject that hypothesis for all the placebo-generating procedures for 1872.

The paper complements the literature that measures European missionaries' positive impacts on the human capital in the New World and Africa \parencite{Barsanetti2021-hp, Franco2021-vn, Waldinger2017-rz, Cage2016-kk, Wantchekon2015-ry, Nunn2014-tj, Nunn2010-ls, Gallego2010-bn, Dell2010-qt}. 
The results of this paper can help better inform how the development of present-day municipalities in the Amazon is correlated with historical settlement patterns.

Future work must be done to analyze the impact of missionaries in Brazil. While this paper and \textcite{Valencia_Caicedo2018-gp} analyze the areas where the Jesuits had a significant presence, the Jesuits and other missionary orders had a strong presence throughout all of South America. Another possible avenue of work would be to analyze the negative consequences caused by colonialism. While the current literature often considers a gain in human capital as positive, that also led to the loss of local identity, customs, and languages. 

\newpage

% \section{Bibliography}

\printbibliography

% \bibliography{citationsv3}

\FloatBarrier
\clearpage

\section*{Maps}

\begin{figure}[H]
	\begin{center}
  	 \makebox[\textwidth]			 
     {\includegraphics[width=0.9\paperwidth, angle=90,origin=c]{SerafimLeite.pdf}}
	\end{center}
	\caption{Location of Jesuit Missions along the Amazon as in the book by {\protect\textcite{Leite1943-dy}}}
	\label{fig:SerafimLeite}
\end{figure}

\begin{landscape}
\begin{figure}[t]
	\begin{center}
  	 \makebox[\textwidth]			 
     {\includegraphics[width=0.85\paperwidth]{PresenceofJesuitMissions2010.png}}
	\end{center}
	\caption{2010 Municipalities and the location of the Jesuit Missions}
	\label{fig:JesuitMissions2010Map}
\end{figure}
\end{landscape}

\begin{landscape}
\begin{figure}[t]
	\begin{center}
  	 \makebox[\textwidth]			 
     {\includegraphics[width=0.85\paperwidth]{PresenceofNonJesuitMissions2010.png}}
	\end{center}
	\caption{2010 Municipalities and the location of the Carmelite/Franciscan Religious Missions}
	\label{fig:NonJesuitMissions2010Map}
\end{figure}
\end{landscape}


\begin{landscape}
  \begin{figure}[t]
    \begin{center}
       \makebox[\textwidth]			 
       {\includegraphics[width=0.85\paperwidth]{Parishes1872.png}}
    \end{center}
    \caption{1872 Municipalities and the location of Census' Parishes}
    \label{fig:Parishes1872}
  \end{figure}
\end{landscape}

\begin{landscape}
  \begin{figure}[t]
    \begin{center}
       \makebox[\textwidth]			 
       {\includegraphics[width=0.85\paperwidth]
       {Tupi2010.png}}
    \end{center}
    \caption{Tupi-speaking areas, based from {\protect\textcite{Clement2015-rf}}. Red dots in the map indicate the location of Jesuit Missions; blue lines indicate the main rivers of the region}
    \label{fig:TupiSpeaking}
  \end{figure}
\end{landscape}

\begin{landscape}
  \begin{figure}[t]
    \begin{center}
       \makebox[\textwidth]			 
       {\includegraphics[width=0.85\paperwidth]{PlaceboMissionsCompletelyRandom.png}}
    \end{center}
    \caption{2010 Municipalities and randomly generated placebo missions}
    \label{fig:PlaceboMapv2}
  \end{figure}
\end{landscape}
 
\begin{landscape}
  \begin{figure}[t]
    \begin{center}
       \makebox[\textwidth]			 
       {\includegraphics[width=0.85\paperwidth]
       {PlaceboMissions.png}}
    \end{center}
    \caption{2010 Municipalities and randomly generated placebo missions, latitude, and longitude distribution}
    \label{fig:PlaceboMap}
  \end{figure}
\end{landscape}

\begin{landscape}
\begin{figure}[t]
	\begin{center}
  	 \makebox[\textwidth]			 
     {\includegraphics[width=0.85\paperwidth]{PlaceboMissionsRiverRandom.png}}
	\end{center}
	\caption{2010 Municipalities and randomly generated placebo missions, randomization across a 10km buffer of the rivers}
	\label{fig:PlaceboMapRiver}
\end{figure}
\end{landscape}

\FloatBarrier
\clearpage

\section*{Graphs}

\begin{figure}[h!]
	\begin{center}
  	 \makebox[\textwidth]			 
	 {\includegraphics[width=0.6\paperwidth]{LiteracyRateParishesJesuits1872.png}}
	\end{center}
	\caption{Unconditional trends for literacy and distance to the nearest Jesuit mission from parishes in 1872. The red dashed line indicates the best fit line.}
	\label{fig:Unconditional1872}
\end{figure}

\begin{figure}[h!]
	\begin{center}
  	 \makebox[\textwidth]			 
	 {\includegraphics[width=0.6\paperwidth]{LiteracyRateMunSeats2010.png}}
	\end{center}
	\caption{Unconditional trends for literacy and distance to the nearest Jesuit mission from municipality seats in 2010. The red dashed line indicates the best fit line.}
	\label{fig:Unconditional2010}
\end{figure}

\begin{figure}[h!]
	\begin{center}
  	 \makebox[\textwidth]			 
	 {\includegraphics[width=0.6\paperwidth]{JesuitDistances.png}}
	\end{center}
	\caption{Results for \protect\autoref{eqn:mainreg} using 2010 data considering only municipalities within a cutoff distance from a Jesuit mission. The red line indicates the original estimate from \protect\autoref{tab:main_results}. Error bars indicate the 95\% confidence interval with robust standard errors.}
	\label{fig:JesuitDistances}
\end{figure}

\begin{figure}[h!]
	\begin{center}
  	 \makebox[\textwidth]			 
	 {\includegraphics[width=0.6\paperwidth]{TreatedDistances.png}}
	\end{center}
	\caption{Results for \protect\autoref{eqn:treated} using 2010 data, considering \textit{Treated} as any municipality within a cutoff distance from a Jesuit mission. Error bars indicate the 95\% confidence interval with robust standard errors.}
	\label{fig:TreatedDistances}
\end{figure}

\begin{figure}[h!]
	\begin{center}
  	 \makebox[\textwidth]			 
	 {\includegraphics[width=0.6\paperwidth]{RiverDistances.png}}
	\end{center}
	\caption{Results for \protect\autoref{eqn:mainreg} using 2010 data considering only municipalities within a cutoff distance from a river. Error bars indicate the 95\% confidence interval with robust standard errors.}
	\label{fig:RiverDistances}
\end{figure}

\begin{figure}[h!]
	\begin{center}
  	 \makebox[\textwidth]			 
	 {\includegraphics[width=0.6\paperwidth]{CapitalDistances.png}}
	\end{center}
	\caption{Results for \protect\autoref{eqn:mainreg} using 2010 data excluding municipalities within a selected cutoff distance from a state capital. Error bars indicate the 95\% confidence interval with robust standard errors.}
	\label{fig:CapitalDistances}
\end{figure}	

%\begin{figure}[h!]
%	\begin{center}
%  	 \makebox[\textwidth]			 %{\includegraphics[width=0.6\paperwidth]
%  	 {DistanceMeasures1872.png}}
%	\end{center}
%	\caption{Relationship between euclidean distance from a parish to a Jesuit mission and the distance through the river network. The red dashed line indicates a 45-degree line}
%	\label{fig:RiverDist1872}
%\end{figure}	

%\begin{figure}[h!]
%	\begin{center}
%  	 \makebox[\textwidth]			 {\includegraphics[width=0.6\paperwidth]
%  	 {DistanceMeasures2010.png}}
%	\end{center}
%	\caption{Relationship between euclidean distance from a municipality seat to a Jesuit mission and the distance through the river network. The red dashed line indicates a 45-degree line}
%	\label{fig:RiverDist2010}
%\end{figure}

\FloatBarrier
\clearpage

\section*{Tables}

\input{~/OneDrive - University of Illinois - Urbana/Research/Projects/Third Year Paper/Tables_v2/summary_1872.tex}

\input{~/OneDrive - University of Illinois - Urbana/Research/Projects/Third Year Paper/Tables_v2/summary_2010.tex}

\subsubsection*{Main Results}

\input{~/OneDrive - University of Illinois - Urbana/Research/Projects/Third Year Paper/Tables_v2/main_results_1872.tex}

\input{~/OneDrive - University of Illinois - Urbana/Research/Projects/Third Year Paper/Tables_v2/main_results_2010.tex}

\FloatBarrier
\clearpage

\subsubsection*{Comparison with missions in Maranhao}

\input{~/OneDrive - University of Illinois - Urbana/Research/Projects/Third Year Paper/Tables_v2/main_results_2010_MA.tex}

\FloatBarrier
\clearpage

\subsubsection*{Comparison with other Missionary Orders}

\input{~/OneDrive - University of Illinois - Urbana/Research/Projects/Third Year Paper/Tables_v2/other_results_combined_1872.tex}

\input{~/OneDrive - University of Illinois - Urbana/Research/Projects/Third Year Paper/Tables_v2/other_results_2010.tex}

\FloatBarrier
\clearpage

\subsubsection*{Instrumental Variable}

\input{~/OneDrive - University of Illinois - Urbana/Research/Projects/Third Year Paper/Tables_v2/tupi_economic.tex}

\input{~/OneDrive - University of Illinois - Urbana/Research/Projects/Third Year Paper/Tables_v2/iv_1872.tex}

\input{~/OneDrive - University of Illinois - Urbana/Research/Projects/Third Year Paper/Tables_v2/iv_2010.tex}

\FloatBarrier
\clearpage

\subsubsection*{Heterogeneity by Gender}

\input{~/OneDrive - University of Illinois - Urbana/Research/Projects/Third Year Paper/Tables_v2/gender_results_1872.tex}

\input{~/OneDrive - University of Illinois - Urbana/Research/Projects/Third Year Paper/Tables_v2/gender_results_2010.tex}

\FloatBarrier
\clearpage

\subsubsection*{Alternative Explanations}

\input{~/OneDrive - University of Illinois - Urbana/Research/Projects/Third Year Paper/Tables_v2/demo_results_1872.tex}

% \input{~/OneDrive - University of Illinois - Urbana/Research/Projects/Third Year Paper/Tables_v2/labor_results_1872.tex}


\input{~/OneDrive - University of Illinois - Urbana/Research/Projects/Third Year Paper/Tables_v2/gdp_results_2010.tex}

\input{~/OneDrive - University of Illinois - Urbana/Research/Projects/Third Year Paper/Tables_v2/alternative_exp_2010.tex}

\subsubsection*{Other Distance Specifications:}

\input{~/OneDrive - University of Illinois - Urbana/Research/Projects/Third Year Paper/Tables_v2/river_distance_both.tex}

\clearpage

\subsubsection*{Other Decades:}

\input{~/OneDrive - University of Illinois - Urbana/Research/Projects/Third Year Paper/Tables_v2/all_years.tex}

\FloatBarrier
\clearpage

\subsubsection*{Randomization Inference}

\input{~/OneDrive - University of Illinois - Urbana/Research/Projects/Third Year Paper/Tables/placebo_results_1872.tex}

\input{~/OneDrive - University of Illinois - Urbana/Research/Projects/Third Year Paper/Tables/placebo_results_2010.tex}

\end{document}
\documentclass{article}

\usepackage[utf8]{inputenc}
\usepackage{subfig}
\usepackage{setspace}

\usepackage{amsmath}
\doublespacing

\usepackage{indentfirst}
\usepackage{pdflscape}
\usepackage[left=1in,right=1in,top=1in,bottom=1in]{geometry}

\DeclareUnicodeCharacter{0301}{\'{e}}

\usepackage[font=small,skip=0pt]{caption}
\usepackage[section]{placeins}
\usepackage{titlesec}
\titlelabel{\thetitle.\quad}
\usepackage{authblk}
\usepackage{csquotes}
\usepackage{booktabs}
\usepackage{siunitx}
\newcolumntype{d}{S[input-open-uncertainty=, input-close-uncertainty=, parse-numbers = false, table-align-text-pre=false, table-align-text-post=false]}
\usepackage{amssymb}
% \newcolumntype{d}{S[input-symbols = ()]}
\usepackage{float}
\usepackage{dcolumn}
\usepackage[bottom]{footmisc} 
\usepackage[textsize=tiny]{todonotes}
\usepackage{longtable}
\usepackage{array}
\usepackage{multirow}
\usepackage{wrapfig}
\usepackage{colortbl}
\usepackage{pdflscape}
\usepackage{tabu}
\usepackage{threeparttable}
\usepackage{threeparttablex}
\usepackage[normalem]{ulem}
\usepackage{makecell}
\usepackage{xcolor}
\usepackage{afterpage}
\usepackage{graphicx}
\usepackage{bbding}
\usepackage{xargs}
\usepackage{xpatch}
\usepackage[citecolor = blue]{hyperref}
\hypersetup{colorlinks = true, linkcolor = red, urlcolor = teal}

\graphicspath{{~/OneDrive - University of Illinois - Urbana/Research/Projects/Sesmarias Brazil/Figures/01. Maps/}}

\usepackage{bookmark}
\usepackage{todonotes}
\setlength{\marginparwidth}{2cm}

\usepackage[backend = biber, style=authoryear, sorting = nty, maxcitenames=1]{biblatex}

\addbibresource{citations_sesmarias.bib}

\DeclareFieldFormat{citehyperref}{%
  \DeclareFieldAlias{bibhyperref}{noformat}% Avoid nested links
  \bibhyperref{#1}}

\DeclareFieldFormat{textcitehyperref}{%
  \DeclareFieldAlias{bibhyperref}{noformat}% Avoid nested links
  \bibhyperref{%
    #1%
    \ifbool{cbx:parens}
      {\bibcloseparen\global\boolfalse{cbx:parens}}
      {}}}

\savebibmacro{cite}
\savebibmacro{textcite}

\renewbibmacro*{cite}{%
  \printtext[citehyperref]{%
    \restorebibmacro{cite}%
    \usebibmacro{cite}}}

\renewbibmacro*{textcite}{%
  \ifboolexpr{
    ( not test {\iffieldundef{prenote}} and
      test {\ifnumequal{\value{citecount}}{1}} )
    or
    ( not test {\iffieldundef{postnote}} and
      test {\ifnumequal{\value{citecount}}{\value{citetotal}}} )
  }
    {\DeclareFieldAlias{textcitehyperref}{noformat}}
    {}%
  \printtext[textcitehyperref]{%
    \restorebibmacro{textcite}%
    \usebibmacro{textcite}}}

\renewcommand*{\nameyeardelim}{\addcomma\space}

% Makes the authors name bold in the References section:

\def\sectionautorefname{Section}

\title{Land Grants in Colonial Brazil and Long-Term Effects on Development}

\author{Vinicius Okada da Silva\thanks{Contact information: University of Illinois at Urbana-Champaign. Department of Economics, 1407 W. Gregory Drive, David Kinley Hall: Room 126, Urbana, Illinois 61801. E-mail: vo10@illinois.edu}}

\affil{Department of Economics, University of Illinois at Urbana-Champaign}

\date{}

\begin{document}

\maketitle
\thispagestyle{empty} 

\vspace{-.1cm}
\begin{center}
  \textcolor{red}{Latest Update: \today}
  \\
  \href{https://viniokadasilva.github.io/Papers/JesuitsAmazon/JesuitsAmazon.pdf}{Click here for the Latest Version}
\end{center}
\vspace{.1cm}

\clearpage
\pagenumbering{arabic} 

\section{Introduction}

Brazil is a country that historically faces issues of both land and income inequality. Estimates from USAID in 2016

However, land inequality in Brazil is something that can be found in the past.

This paper tries to answer how much of Brazil's present-day inequality can be traced to colonial institutions. Specifically, this paper uses Portuguese land grants called \textit{sesmarias}, to identify the historical persistence of colonial activity in Brazil to present-day inequality.

\parencite{Dell2010-qt}
\parencite{Sokoloff2000-mb}



\textcite{Ratnoo2023-vw} [Paper about land tenure in India]


\section{Historical Background}

\textcite{Dean1971-iq} - ``Anyone who claimed to have the means and desire to make use of the land was given a grant, customarily one to three leagues in extent (16.7 to 50.1 square miles).''

\textcite{Simonsen2005-ps} - ``the ones that don't possess sesmarias or can't own land are disowned by the own society they live in"

\section{Data}

The main source of data comes from the \textit{Sesmarias of the Luso-Brazilian Empire Database}\footnote{
  Information on the content of the letters is available at \url{http://plataformasilb.cchla.ufrn.br/}. The georeferencing process was done in collaboration but as a separate project for this paper.}.
The database uses archival data from either state records or original manuscripts to obtain data on the concessions of sesmarias in Brazil. 
When available, information such as the year, the reason for the request, etc. are coded. 
The sesmarias are then georeferenced based on the geographical information present in the text, allowing us to trace them back to Brazilian municipalities.\footnote{A more in-depth description on how the sources of the letters and how the sesmarias were georeferenced is available in \autoref{app:appendix_data}}

Data for current land tenure in 2021 in Brazil is obtained from \textcite{Sparovek2019-dn}.\footnote{
  Available at \url{https://atlasagropecuario.imaflora.org/}}

Land usage from 1985-2010 is obtained from Mapbiomas.\textcite{Souza2020-kb}\footnote{
  Available at \url{https://brasil.mapbiomas.org/en/}}

Census data for 1872 is obtained from the Nucleus of Research in Economic and Geographic History from the Federal University of Minas Gerais.\footnote{
  Available at \url{http://www.nphed.cedeplar.ufmg.br/}}
The 1872 Imperial Census contains demographic data at the municipality and parish level and was the last census taken before the abolition of slavery in Brazil. \footnote{Distribution of the 1872 parishes alongside the municipality boundaries is available at \autoref{fig:parishes_1872}. For the sample used I have 469 municipalities and 968 parishes.}

Other census data is obtained from the IBGE (\textit{}).\footnote{Microcensus data downloaded through the R package \textit{censobr} from \textcite{Pereira2023-qv}}

\section{Descriptive}

Following \textcite{Lowes2021-ww} I show balance on geographical characteristics at the 10 x 10km grid level in [reference to table here].

\section{Identification Strategy}

\subsection{Coastal Ban on Livestock}

In 1701, the Portuguese Crown enacted a ban on cattle ranching from 80km of the coast (10 leagues) \parencites[p~.40]{Fausto2014-bh}[p~.198]{Simonsen2005-ps}[p~.460]{Bethell1984-of}.

\textcite{Bethell1984-of} ``Landholding in the sertao was truly extensive. Although there was legislation limiting the size of sesmarias to three square leagues, this restriction was simply disregarded. The sesmarias on which cattle ranches were established sometimes exceeded hundreds of thousands of acres"

The effect was the expansion of cattle ranches towards the west of Brazil, especially in the Northeastern states. 
As \textcite[p~.41]{Fausto2014-bh} indicates, the need for large lands to allow cattle to roam free led to the creation of large estates in the area, even bigger than those compared to the coast.\footnote{An example of this would be the d'Avila family which owned a large estate in the state of Bahia [...]}
That led to a ``a clear specialization between the two activities'' \parencite{Ribeiro2012-lb}.\footnote{An example of the effect can be seen in the Municipality of Ruy Barbosa and the state of Bahia and Caico in the state of Rio Grande do Norte. Both are described and being created by the cattle expansion that happened because of the 1701 Royal Decree. \parencite{UnknownUnknown-ro}}

\parencite[p~.]{Boxer1962-bj}

``Cattle farming was to supply dry beef, leather, and carrying animals to the sugar mills and, later, to the villas that emerged around mining, but was not to mix itself geographically with these other two important export activities from the colonial period, nor with the coffee estates that emerged during the nineteenth century,  when  Brazil was already independent from  Portugal.'' \parencite{Ribeiro2012-lb}.

``It was there that farms measuring thousands of hectares emerged, where cattle found favourable environmental conditions for the multiplication of herds.''\parencite{Ribeiro2012-lb}.


Given the nature of this ban, I exploit the cutoff of 80 km to use a regression discontinuity design to measure the effects between the two types of economies in the region. 

In the first-stage I check whether post-1701 we see an increasing number of land grants dedicated to livestock in municipalities farther than 80 km from the coast. 

Secondly, using the 1872 I analyze whether or not there were any effects of the coastal livestock ban on the demographics and economic activities at that time.




Historically livestock-raising areas were [...]

\begin{equation}
  Y_{i,m,s} = 
\end{equation}

Provision of Public goods is the cause for the effects on literacy in 1970 and onwards (?).

Other links:

\url{http://historialuso.an.gov.br/index.php?option=com_content&view=article&id=6191:escravos-de-ganho&catid=2073&Itemid=121}

\url{https://www.nexojornal.com.br/especial/2017/07/07/censo-de-1872-o-retrato-do-brasil-da-escravidao}

“Quando o senhor não tinha uma função para o escravo, ele deixava o escravo ao ganho”, explica o historiador Diego Bissigo. “Ele ia para cidade buscar emprego e o senhor ficava com o salário que o escravo recebesse. É uma forma de uso para o escravo. Assim, ou alugando para outro senhor também.”



\parencite[p.~142]{De_Oliveira_Andrade1980-xz}

\subsection{Treaty of Tordesillas}

\subsection{Instrumental Variable}

\section{Results}



\section{Robustness}
\subsection{Donut RDD}

\clearpage


\xpatchbibmacro{author}{\printnames{author}}{\textbf{\printnames{author}}}{}{}

\printbibliography

\clearpage

\section{Figures}

\begin{figure}
  \caption{Distribution of Land Grants pre- and post- 1701}
  \begin{center}
     \makebox[\textwidth]			 
     {\includegraphics[width=0.85\paperwidth]{/Users/vinicius/Library/CloudStorage/OneDrive-UniversityofIllinois-Urbana/Research/Projects/JMP/02. Figures/00.Maps/land_grant_distribution_1701.png}}
  \textit{Notes:} This figure considers whether or not any part of the municipality was within 80km of the coast.
  \end{center}
  \label{fig:SesmariasDistribution}
\end{figure}


\clearpage

\begin{figure}[h!]
  \caption{1970 Municipalities and Sharecropping}
  \begin{center}
     \makebox[\textwidth]			 
     {\includegraphics[width=0.85\paperwidth]{/Users/vinicius/Library/CloudStorage/OneDrive-UniversityofIllinois-Urbana/Research/Projects/JMP/02. Figures/00.Maps/coastal_rdd_grid_pasture.png}}
  \textit{Notes:} This figure considers whether or not any part of the municipality was within 80km of the coast.
  \end{center}
  \label{fig:pasture_grid}
\end{figure}

\begin{figure}
  \caption{1970 Municipalities and Sharecropping}
  \begin{center}
     \makebox[\textwidth]			 
     {\includegraphics[width=0.85\paperwidth]{/Users/vinicius/Library/CloudStorage/OneDrive-UniversityofIllinois-Urbana/Research/Projects/JMP/02. Figures/00.Maps//coastal_rdd_grid_natural.png}}
  \textit{Notes:} This figure considers whether or not any part of the municipality was within 80km of the coast.
  \end{center}
  \label{fig:natural_grid}
\end{figure}

\begin{figure}
  \caption{1970 Municipalities and Sharecropping}
  \begin{center}
     \makebox[\textwidth]			 
     {\includegraphics[width=0.85\paperwidth]{/Users/vinicius/Library/CloudStorage/OneDrive-UniversityofIllinois-Urbana/Research/Projects/JMP/02. Figures/00.Maps//coastal_rdd_grid_sugarcane.png}}
  \textit{Notes:} This figure considers whether or not any part of the municipality was within 80km of the coast.
  \end{center}
  \label{fig:sugarcane_grid}
\end{figure}

\clearpage

\begin{figure}[h!]
  \caption{1970 Municipalities and Sharecropping}
  \begin{center}
     \makebox[\textwidth]			 
     {\includegraphics[width=0.85\paperwidth]{/Users/vinicius/Library/CloudStorage/OneDrive-UniversityofIllinois-Urbana/Research/Projects/JMP/02. Figures/00.Maps/coastal_rdd_1970_population.png}}
  \textit{Notes:} This figure considers whether or not any part of the municipality was within 80km of the coast.
  \end{center}
  \label{fig:sharecropping_1970}
\end{figure}

\begin{figure}[h!]
  \caption{1970 Municipalities and Sharecropping}
  \begin{center}
     \makebox[\textwidth]			 
     {\includegraphics[width=0.85\paperwidth]{/Users/vinicius/Library/CloudStorage/OneDrive-UniversityofIllinois-Urbana/Research/Projects/JMP/02. Figures/00.Maps/coastal_rdd_1970_sharecropping.png}}
  \textit{Notes:} This figure considers whether or not any part of the municipality was within 80km of the coast.
  \end{center}
  \label{fig:sharecropping_1970}
\end{figure}

\begin{figure}
  \caption{1970 Municipalities and Land Ownership}
  \begin{center}
     \makebox[\textwidth]			 
     {\includegraphics[width=0.85\paperwidth]{/Users/vinicius/Library/CloudStorage/OneDrive-UniversityofIllinois-Urbana/Research/Projects/JMP/02. Figures/00.Maps/coastal_rdd_1970_rural_land.png}}
  \textit{Notes:} This figure considers whether or not any part of the municipality was within 80km of the coast.
  \end{center}
  \label{fig:land_ownership_1970}
\end{figure}

\begin{figure}
  \caption{1970 Municipalities and Sugarcane}
  \begin{center}
     \makebox[\textwidth]			 
     {\includegraphics[width=0.85\paperwidth]{/Users/vinicius/Library/CloudStorage/OneDrive-UniversityofIllinois-Urbana/Research/Projects/JMP/02. Figures/00.Maps/coastal_rdd_1970_sugarcane.png}}
  \textit{Notes:} This figure considers whether or not any part of the municipality was within 80km of the coast.
  \end{center}
  \label{fig:sugarcane_1970}
\end{figure}

\begin{figure}
  \caption{1872 Municipalities and Proximity to the Coast}
  \begin{center}
     \makebox[\textwidth]			 
     {\includegraphics[width=0.85\paperwidth]{/Users/vinicius/Library/CloudStorage/OneDrive-UniversityofIllinois-Urbana/Research/Projects/JMP/02. Figures/00.Maps/coastal_rdd_1970_cotton.png}}
  \textit{Notes:} This figure considers whether or not any part of the municipality was within 80km of the coast.
  \end{center}
  \label{fig:cotton_1970}
\end{figure}

\begin{figure}
  \caption{1872 Municipalities and Proximity to the Coast}
  \begin{center}
     \makebox[\textwidth]			 
     {\includegraphics[width=0.85\paperwidth]{/Users/vinicius/Library/CloudStorage/OneDrive-UniversityofIllinois-Urbana/Research/Projects/JMP/02. Figures/00.Maps/coastal_rdd_1970_rural_land.png}}
  \textit{Notes:} This figure considers whether or not any part of the municipality was within 80km of the coast.
  \end{center}
  \label{fig:sugarcane_1970}
\end{figure}

\begin{figure}
  \caption{1872 Municipalities and Proximity to the Coast}
  \begin{center}
     \makebox[\textwidth]			 
     {\includegraphics[width=0.85\paperwidth]{/Users/vinicius/Library/CloudStorage/OneDrive-UniversityofIllinois-Urbana/Research/Projects/JMP/02. Figures/00.Maps/coastal_rdd_1970_pecuaria.png}}
  \textit{Notes:} This figure considers whether or not any part of the municipality was within 80km of the coast.
  \end{center}
  \label{fig:sugarcane_1970}
\end{figure}
\clearpage

\section{Tables}
\subsection{Coastal RDD - 1872}
\begin{table}[!h]

\caption{Effects on Proportion of Slaves to Total Population (\%) \label{tab:rdd_free_slave}}
\centering
\begin{threeparttable}
\begin{tabular}[t]{lccccc}
\toprule
\multicolumn{1}{c}{} & \multicolumn{2}{c}{Optimal Bandwidth} & \multicolumn{1}{c}{[10, 150]} & \multicolumn{1}{c}{[20, 140]} & \multicolumn{1}{c}{[30, 130]} \\
\cmidrule(l{3pt}r{3pt}){2-3} \cmidrule(l{3pt}r{3pt}){4-4} \cmidrule(l{3pt}r{3pt}){5-5} \cmidrule(l{3pt}r{3pt}){6-6}
  & (1) & (2) & (3) & (4) & (5)\\
\midrule
Estimate & \num{5.168}** & \num{5.698}* & \num{6.397}** & \num{5.666} & \num{4.842}\\
 & (\num{2.270}) & (\num{3.414}) & (\num{3.202}) & (\num{3.550}) & (\num{3.961})\\

\midrule
Polynomial Order & 1 & 2 & 1 & 1 & 1\\
N & \num{239} & \num{239} & \num{123} & \num{105} & \num{91}\\
$R^2$ & \num{0.05} & \num{0.05} & \num{0.07} & \num{0.08} & \num{0.08}\\
\bottomrule
\multicolumn{6}{l}{\rule{0pt}{1em}* p $<$ 0.1, ** p $<$ 0.05, *** p $<$ 0.01}\\
\end{tabular}
\begin{tablenotes}
\item[a] Mean of the dependent variable is 14.2\%
\end{tablenotes}
\end{threeparttable}
\end{table}


\clearpage

\begin{table}[!h]

\caption{Effects on Proportion of Slaves Working in Farming (\%) \label{tab:rdd_enslaved_farming}}
\centering
\begin{threeparttable}
\begin{tabular}[t]{lccccc}
\toprule
\multicolumn{1}{c}{} & \multicolumn{2}{c}{Optimal Bandwidth} & \multicolumn{1}{c}{[10, 150]} & \multicolumn{1}{c}{[20, 140]} & \multicolumn{1}{c}{[30, 130]} \\
\cmidrule(l{3pt}r{3pt}){2-3} \cmidrule(l{3pt}r{3pt}){4-4} \cmidrule(l{3pt}r{3pt}){5-5} \cmidrule(l{3pt}r{3pt}){6-6}
  & (1) & (2) & (3) & (4) & (5)\\
\midrule
Estimate & \num{-8.993} & \num{-6.887} & \num{-8.619} & \num{-7.263} & \num{-5.337}\\
 & (\num{6.356}) & (\num{9.516}) & (\num{6.314}) & (\num{6.936}) & (\num{7.237})\\

\midrule
Polynomial Order & 1 & 2 & 1 & 1 & 1\\
N & \num{122} & \num{122} & \num{123} & \num{105} & \num{91}\\
$R^2$ & \num{0.05} & \num{0.06} & \num{0.05} & \num{0.07} & \num{0.08}\\
\bottomrule
\multicolumn{6}{l}{\rule{0pt}{1em}* p $<$ 0.1, ** p $<$ 0.05, *** p $<$ 0.01}\\
\end{tabular}
\begin{tablenotes}
\item[a] Mean of the dependent variable is 40.7\%
\end{tablenotes}
\end{threeparttable}
\end{table}


\begin{table}[!h]

\caption{Effects on Proportion of Male Slaves Working in Farming (\%) \label{tab:rdd_men_enslaved_farming}}
\centering
\begin{threeparttable}
\begin{tabular}[t]{lccccc}
\toprule
\multicolumn{1}{c}{} & \multicolumn{2}{c}{Optimal Bandwidth} & \multicolumn{1}{c}{[10, 150]} & \multicolumn{1}{c}{[20, 140]} & \multicolumn{1}{c}{[30, 130]} \\
\cmidrule(l{3pt}r{3pt}){2-3} \cmidrule(l{3pt}r{3pt}){4-4} \cmidrule(l{3pt}r{3pt}){5-5} \cmidrule(l{3pt}r{3pt}){6-6}
  & (1) & (2) & (3) & (4) & (5)\\
\midrule
Estimate & \num{-11.232}* & \num{-4.727} & \num{-8.586} & \num{-7.334} & \num{-5.358}\\
 & (\num{6.792}) & (\num{10.085}) & (\num{6.871}) & (\num{7.529}) & (\num{7.887})\\

\midrule
Polynomial Order & 1 & 2 & 1 & 1 & 1\\
N & \num{182} & \num{182} & \num{123} & \num{105} & \num{91}\\
$R^2$ & \num{0.02} & \num{0.03} & \num{0.03} & \num{0.04} & \num{0.05}\\
\bottomrule
\multicolumn{6}{l}{\rule{0pt}{1em}* p $<$ 0.1, ** p $<$ 0.05, *** p $<$ 0.01}\\
\end{tabular}
\begin{tablenotes}
\item[a] Mean of the dependent variable is 50.1\%
\end{tablenotes}
\end{threeparttable}
\end{table}


\begin{table}[!h]

\caption{Effects on Proportion of Female Slaves Working in Farming (\%) \label{tab:rdd_women_enslaved_farming}}
\centering
\begin{threeparttable}
\begin{tabular}[t]{lccccc}
\toprule
\multicolumn{1}{c}{} & \multicolumn{2}{c}{Optimal Bandwidth} & \multicolumn{1}{c}{[10, 150]} & \multicolumn{1}{c}{[20, 140]} & \multicolumn{1}{c}{[30, 130]} \\
\cmidrule(l{3pt}r{3pt}){2-3} \cmidrule(l{3pt}r{3pt}){4-4} \cmidrule(l{3pt}r{3pt}){5-5} \cmidrule(l{3pt}r{3pt}){6-6}
  & (1) & (2) & (3) & (4) & (5)\\
\midrule
Estimate & \num{-9.622} & \num{-12.641} & \num{-10.360} & \num{-9.083} & \num{-6.989}\\
 & (\num{6.936}) & (\num{10.367}) & (\num{6.575}) & (\num{7.161}) & (\num{7.495})\\

\midrule
Polynomial Order & 1 & 2 & 1 & 1 & 1\\
N & \num{114} & \num{114} & \num{123} & \num{105} & \num{91}\\
$R^2$ & \num{0.08} & \num{0.11} & \num{0.09} & \num{0.10} & \num{0.12}\\
\bottomrule
\multicolumn{6}{l}{\rule{0pt}{1em}* p $<$ 0.1, ** p $<$ 0.05, *** p $<$ 0.01}\\
\end{tabular}
\begin{tablenotes}
\item[a] Mean of the dependent variable is 30.1\%
\end{tablenotes}
\end{threeparttable}
\end{table}


\clearpage

\begin{table}[!h]

\caption{Effects on Proportion of Slaves in Domestic Work (\%) \label{tab:rdd_enslaved_domestic}}
\centering
\begin{threeparttable}
\begin{tabular}[t]{lccccc}
\toprule
\multicolumn{1}{c}{} & \multicolumn{2}{c}{Optimal Bandwidth} & \multicolumn{1}{c}{[10, 150]} & \multicolumn{1}{c}{[20, 140]} & \multicolumn{1}{c}{[30, 130]} \\
\cmidrule(l{3pt}r{3pt}){2-3} \cmidrule(l{3pt}r{3pt}){4-4} \cmidrule(l{3pt}r{3pt}){5-5} \cmidrule(l{3pt}r{3pt}){6-6}
  & (1) & (2) & (3) & (4) & (5)\\
\midrule
Estimate & \num{9.806}*** & \num{2.745} & \num{9.608}** & \num{6.669} & \num{5.629}\\
 & (\num{3.727}) & (\num{5.484}) & (\num{3.828}) & (\num{4.213}) & (\num{4.518})\\

\midrule
Polynomial Order & 1 & 2 & 1 & 1 & 1\\
N & \num{130} & \num{130} & \num{123} & \num{105} & \num{91}\\
$R^2$ & \num{0.10} & \num{0.12} & \num{0.10} & \num{0.14} & \num{0.14}\\
\bottomrule
\multicolumn{6}{l}{\rule{0pt}{1em}* p $<$ 0.1, ** p $<$ 0.05, *** p $<$ 0.01}\\
\end{tabular}
\begin{tablenotes}
\item[a] Mean of the dependent variable is 18.6\%
\end{tablenotes}
\end{threeparttable}
\end{table}


\begin{table}[!h]

\caption{Effects on Proportion of Male Slaves in Domestic Work (\%) \label{tab:rdd_enslaved_domestic_men}}
\centering
\begin{threeparttable}
\begin{tabular}[t]{lccccc}
\toprule
\multicolumn{1}{c}{} & \multicolumn{2}{c}{Optimal Bandwidth} & \multicolumn{1}{c}{[10, 150]} & \multicolumn{1}{c}{[20, 140]} & \multicolumn{1}{c}{[30, 130]} \\
\cmidrule(l{3pt}r{3pt}){2-3} \cmidrule(l{3pt}r{3pt}){4-4} \cmidrule(l{3pt}r{3pt}){5-5} \cmidrule(l{3pt}r{3pt}){6-6}
  & (1) & (2) & (3) & (4) & (5)\\
\midrule
Estimate & \num{5.855}* & \num{-0.333} & \num{5.723} & \num{3.001} & \num{2.371}\\
 & (\num{3.346}) & (\num{4.923}) & (\num{3.548}) & (\num{3.655}) & (\num{4.074})\\

\midrule
Polynomial Order & 1 & 2 & 1 & 1 & 1\\
N & \num{137} & \num{137} & \num{123} & \num{105} & \num{91}\\
$R^2$ & \num{0.09} & \num{0.11} & \num{0.10} & \num{0.14} & \num{0.13}\\
\bottomrule
\multicolumn{6}{l}{\rule{0pt}{1em}* p $<$ 0.1, ** p $<$ 0.05, *** p $<$ 0.01}\\
\end{tabular}
\begin{tablenotes}
\item[a] Mean of the dependent variable is 10.7\%
\end{tablenotes}
\end{threeparttable}
\end{table}


\begin{table}[!h]

\caption{Effects on Proportion of Female Slaves in Domestic Work (\%) \label{tab:rdd_enslaved_domestic_women}}
\centering
\begin{threeparttable}
\begin{tabular}[t]{lccccc}
\toprule
\multicolumn{1}{c}{} & \multicolumn{2}{c}{Optimal Bandwidth} & \multicolumn{1}{c}{[10, 150]} & \multicolumn{1}{c}{[20, 140]} & \multicolumn{1}{c}{[30, 130]} \\
\cmidrule(l{3pt}r{3pt}){2-3} \cmidrule(l{3pt}r{3pt}){4-4} \cmidrule(l{3pt}r{3pt}){5-5} \cmidrule(l{3pt}r{3pt}){6-6}
  & (1) & (2) & (3) & (4) & (5)\\
\midrule
Estimate & \num{15.722}*** & \num{11.082} & \num{14.964}*** & \num{11.980}* & \num{10.429}\\
 & (\num{5.155}) & (\num{7.625}) & (\num{5.548}) & (\num{6.271}) & (\num{6.510})\\

\midrule
Polynomial Order & 1 & 2 & 1 & 1 & 1\\
N & \num{184} & \num{184} & \num{123} & \num{105} & \num{91}\\
$R^2$ & \num{0.05} & \num{0.07} & \num{0.09} & \num{0.11} & \num{0.13}\\
\bottomrule
\multicolumn{6}{l}{\rule{0pt}{1em}* p $<$ 0.1, ** p $<$ 0.05, *** p $<$ 0.01}\\
\end{tabular}
\begin{tablenotes}
\item[a] Mean of the dependent variable is 27.3\%
\end{tablenotes}
\end{threeparttable}
\end{table}


\clearpage

\begin{table}[!h]

\caption{Effects on Proportion of Farmers (\%) \label{tab:rdd_farmers}}
\centering
\begin{threeparttable}
\begin{tabular}[t]{lccccc}
\toprule
\multicolumn{1}{c}{} & \multicolumn{2}{c}{Optimal Bandwidth} & \multicolumn{1}{c}{[10, 150]} & \multicolumn{1}{c}{[20, 140]} & \multicolumn{1}{c}{[30, 130]} \\
\cmidrule(l{3pt}r{3pt}){2-3} \cmidrule(l{3pt}r{3pt}){4-4} \cmidrule(l{3pt}r{3pt}){5-5} \cmidrule(l{3pt}r{3pt}){6-6}
  & (1) & (2) & (3) & (4) & (5)\\
\midrule
Estimate & \num{-4.491} & \num{-7.025} & \num{-4.159} & \num{-3.052} & \num{-3.602}\\
 & (\num{4.714}) & (\num{7.028}) & (\num{4.687}) & (\num{5.119}) & (\num{5.173})\\

\midrule
Polynomial Order & 1 & 2 & 1 & 1 & 1\\
N & \num{122} & \num{122} & \num{123} & \num{105} & \num{91}\\
$R^2$ & \num{0.03} & \num{0.05} & \num{0.03} & \num{0.05} & \num{0.05}\\
\bottomrule
\multicolumn{6}{l}{\rule{0pt}{1em}* p $<$ 0.1, ** p $<$ 0.05, *** p $<$ 0.01}\\
\end{tabular}
\begin{tablenotes}
\item[a] Mean of the dependent variable is 31.4\%
\end{tablenotes}
\end{threeparttable}
\end{table}


\begin{table}[!h]

\caption{Effects on Literacy Rate (\%) \label{tab:rdd_literacy_1872}}
\centering
\begin{threeparttable}
\begin{tabular}[t]{lccccc}
\toprule
\multicolumn{1}{c}{} & \multicolumn{2}{c}{Optimal Bandwidth} & \multicolumn{1}{c}{[10, 150]} & \multicolumn{1}{c}{[20, 140]} & \multicolumn{1}{c}{[30, 130]} \\
\cmidrule(l{3pt}r{3pt}){2-3} \cmidrule(l{3pt}r{3pt}){4-4} \cmidrule(l{3pt}r{3pt}){5-5} \cmidrule(l{3pt}r{3pt}){6-6}
  & (1) & (2) & (3) & (4) & (5)\\
\midrule
Estimate & \num{1.228} & \num{2.119} & \num{1.303} & \num{-2.724} & \num{-0.990}\\
 & (\num{4.063}) & (\num{5.994}) & (\num{4.868}) & (\num{5.330}) & (\num{5.699})\\

\midrule
Polynomial Order & 1 & 2 & 1 & 1 & 1\\
N & \num{205} & \num{205} & \num{123} & \num{105} & \num{91}\\
$R^2$ & \num{0.01} & \num{0.01} & \num{0.00} & \num{0.02} & \num{0.02}\\
\bottomrule
\multicolumn{6}{l}{\rule{0pt}{1em}* p $<$ 0.1, ** p $<$ 0.05, *** p $<$ 0.01}\\
\end{tabular}
\begin{tablenotes}
\item[a] Mean of the dependent variable is 26.3\%
\end{tablenotes}
\end{threeparttable}
\end{table}


\clearpage

\subsection{Coastal RDD - 1970}

\begin{table}[!h]

\caption{Effects on Proportion of People Sharecropping (\%) \label{tab:rdd_sharecropping}}
\centering
\begin{threeparttable}
\begin{tabular}[t]{lccccc}
\toprule
\multicolumn{1}{c}{} & \multicolumn{2}{c}{Optimal Bandwidth} & \multicolumn{1}{c}{[10, 150]} & \multicolumn{1}{c}{[20, 140]} & \multicolumn{1}{c}{[30, 130]} \\
\cmidrule(l{3pt}r{3pt}){2-3} \cmidrule(l{3pt}r{3pt}){4-4} \cmidrule(l{3pt}r{3pt}){5-5} \cmidrule(l{3pt}r{3pt}){6-6}
  & (1) & (2) & (3) & (4) & (5)\\
\midrule
Estimate & \num{1.430}*** & \num{1.385}** & \num{1.464}*** & \num{1.478}** & \num{1.430}**\\
 & (\num{0.450}) & (\num{0.691}) & (\num{0.551}) & (\num{0.601}) & (\num{0.670})\\

\midrule
Polynomial Order & 1 & 2 & 1 & 1 & 1\\
N & \num{1202} & \num{1202} & \num{848} & \num{714} & \num{608}\\
$R^2$ & \num{0.13} & \num{0.13} & \num{0.08} & \num{0.07} & \num{0.07}\\
\bottomrule
\multicolumn{6}{l}{\rule{0pt}{1em}* p $<$ 0.1, ** p $<$ 0.05, *** p $<$ 0.01}\\
\end{tabular}
\begin{tablenotes}
\item[a] Mean of the dependent variable is 3.33\%
\end{tablenotes}
\end{threeparttable}
\end{table}


\begin{table}[!h]

\caption{Effects on Proportion of People Working with Livestock (\%) \label{tab:rdd_livestock}}
\centering
\begin{threeparttable}
\begin{tabular}[t]{lccccc}
\toprule
\multicolumn{1}{c}{} & \multicolumn{2}{c}{Optimal Bandwidth} & \multicolumn{1}{c}{[10, 150]} & \multicolumn{1}{c}{[20, 140]} & \multicolumn{1}{c}{[30, 130]} \\
\cmidrule(l{3pt}r{3pt}){2-3} \cmidrule(l{3pt}r{3pt}){4-4} \cmidrule(l{3pt}r{3pt}){5-5} \cmidrule(l{3pt}r{3pt}){6-6}
  & (1) & (2) & (3) & (4) & (5)\\
\midrule
Estimate & \num{1.397} & \num{2.691} & \num{1.241} & \num{1.650} & \num{1.921}\\
 & (\num{1.493}) & (\num{2.337}) & (\num{1.125}) & (\num{1.250}) & (\num{1.394})\\

\midrule
Polynomial Order & 1 & 2 & 1 & 1 & 1\\
N & \num{557} & \num{557} & \num{848} & \num{714} & \num{608}\\
$R^2$ & \num{0.01} & \num{0.01} & \num{0.00} & \num{0.01} & \num{0.01}\\
\bottomrule
\multicolumn{6}{l}{\rule{0pt}{1em}* p $<$ 0.1, ** p $<$ 0.05, *** p $<$ 0.01}\\
\end{tabular}
\begin{tablenotes}
\item[a] Mean of the dependent variable is 18.9\%
\end{tablenotes}
\end{threeparttable}
\end{table}


\begin{table}[!h]

\caption{Effects on Proportion of People Working with Sugarcane (\%) \label{tab:rdd_sugarcane}}
\centering
\begin{threeparttable}
\begin{tabular}[t]{lccccc}
\toprule
\multicolumn{1}{c}{} & \multicolumn{2}{c}{Optimal Bandwidth} & \multicolumn{1}{c}{[10, 150]} & \multicolumn{1}{c}{[20, 140]} & \multicolumn{1}{c}{[30, 130]} \\
\cmidrule(l{3pt}r{3pt}){2-3} \cmidrule(l{3pt}r{3pt}){4-4} \cmidrule(l{3pt}r{3pt}){5-5} \cmidrule(l{3pt}r{3pt}){6-6}
  & (1) & (2) & (3) & (4) & (5)\\
\midrule
Estimate & \num{-0.011} & \num{0.277} & \num{0.095} & \num{-0.116} & \num{0.047}\\
 & (\num{0.572}) & (\num{0.883}) & (\num{0.569}) & (\num{0.551}) & (\num{0.574})\\

\midrule
Polynomial Order & 1 & 2 & 1 & 1 & 1\\
N & \num{612} & \num{612} & \num{848} & \num{714} & \num{608}\\
$R^2$ & \num{0.08} & \num{0.08} & \num{0.10} & \num{0.08} & \num{0.08}\\
\bottomrule
\multicolumn{6}{l}{\rule{0pt}{1em}* p $<$ 0.1, ** p $<$ 0.05, *** p $<$ 0.01}\\
\end{tabular}
\begin{tablenotes}
\item[a] Mean of the dependent variable is 1.4\%
\end{tablenotes}
\end{threeparttable}
\end{table}


\begin{table}[!h]

\caption{Effects on Literacy Rate (\%) \label{tab:rdd_literacy}}
\centering
\begin{threeparttable}
\begin{tabular}[t]{lccccc}
\toprule
\multicolumn{1}{c}{} & \multicolumn{2}{c}{Optimal Bandwidth} & \multicolumn{1}{c}{[10, 150]} & \multicolumn{1}{c}{[20, 140]} & \multicolumn{1}{c}{[30, 130]} \\
\cmidrule(l{3pt}r{3pt}){2-3} \cmidrule(l{3pt}r{3pt}){4-4} \cmidrule(l{3pt}r{3pt}){5-5} \cmidrule(l{3pt}r{3pt}){6-6}
  & (1) & (2) & (3) & (4) & (5)\\
\midrule
Estimate & \num{1.862} & \num{0.320} & \num{3.077} & \num{2.468} & \num{0.413}\\
 & (\num{2.499}) & (\num{3.852}) & (\num{2.155}) & (\num{2.399}) & (\num{2.656})\\

\midrule
Polynomial Order & 1 & 2 & 1 & 1 & 1\\
N & \num{670} & \num{670} & \num{848} & \num{714} & \num{608}\\
$R^2$ & \num{0.07} & \num{0.07} & \num{0.08} & \num{0.07} & \num{0.07}\\
\bottomrule
\multicolumn{6}{l}{\rule{0pt}{1em}* p $<$ 0.1, ** p $<$ 0.05, *** p $<$ 0.01}\\
\end{tabular}
\begin{tablenotes}
\item[a] Mean of the dependent variable is 43.1\%
\end{tablenotes}
\end{threeparttable}
\end{table}


\begin{table}[!h]

\caption{Effects on Proportion of People Working with Farming (\%) \label{tab:rdd_farming}}
\centering
\begin{threeparttable}
\begin{tabular}[t]{lccccc}
\toprule
\multicolumn{1}{c}{} & \multicolumn{2}{c}{Optimal Bandwidth} & \multicolumn{1}{c}{[10, 150]} & \multicolumn{1}{c}{[20, 140]} & \multicolumn{1}{c}{[30, 130]} \\
\cmidrule(l{3pt}r{3pt}){2-3} \cmidrule(l{3pt}r{3pt}){4-4} \cmidrule(l{3pt}r{3pt}){5-5} \cmidrule(l{3pt}r{3pt}){6-6}
  & (1) & (2) & (3) & (4) & (5)\\
\midrule
Estimate & \num{1.074} & \num{2.021} & \num{0.848} & \num{1.216} & \num{1.506}\\
 & (\num{1.472}) & (\num{2.306}) & (\num{1.107}) & (\num{1.229}) & (\num{1.374})\\

\midrule
Polynomial Order & 1 & 2 & 1 & 1 & 1\\
N & \num{557} & \num{557} & \num{848} & \num{714} & \num{608}\\
$R^2$ & \num{0.01} & \num{0.01} & \num{0.00} & \num{0.00} & \num{0.01}\\
\bottomrule
\multicolumn{6}{l}{\rule{0pt}{1em}* p $<$ 0.1, ** p $<$ 0.05, *** p $<$ 0.01}\\
\end{tabular}
\begin{tablenotes}
\item[a] Mean of the dependent variable is 19.5\%
\end{tablenotes}
\end{threeparttable}
\end{table}


\clearpage

\appendix

\setcounter{figure}{0}  
\setcounter{table}{0}  

\renewcommand{\thefigure}{A.\arabic{figure}}
\renewcommand{\thetable}{A.\arabic{table}}

\section{Figures}

\begin{figure}
  \caption{1872 Municipalities and Parish Locations}
  \begin{center}
     \makebox[\textwidth]			 
     {\includegraphics[width=0.85\paperwidth]{/Users/vinicius/Library/CloudStorage/OneDrive-UniversityofIllinois-Urbana/Research/Projects/JMP/02. Figures/00.Maps/parishes_1872.png}}
  \textit{Notes:} Geographical distribution of 1872 parishes alongside 1872 municipality boundaries.
  \end{center}
  \label{fig:parishes_1872}
\end{figure}

\section{Description of Letters and Georeferencing}
\label{app:appendix_data}

\end{document}
\documentclass{article}

\usepackage[utf8]{inputenc}
\usepackage{subfig}
\usepackage{setspace}
\spacing{1.15}

\usepackage{amsmath}

\usepackage{indentfirst}
\usepackage{pdflscape}
\usepackage[left=1in,right=1in,top=1in,bottom=1in]{geometry}

\DeclareUnicodeCharacter{0301}{\'{e}}

\usepackage[font=small,skip=0pt]{caption}
\usepackage[section]{placeins}
\usepackage{titlesec}
\titlelabel{\thetitle.\quad}
\usepackage{authblk}
\usepackage{csquotes}
\usepackage{booktabs}
\usepackage{siunitx}
\usepackage{amssymb}
\newcolumntype{d}{S[input-symbols = ()]}
\usepackage{float}
\usepackage{dcolumn}
\usepackage[bottom]{footmisc} 
\usepackage[textsize=tiny]{todonotes}
\usepackage{longtable}
\usepackage{array}
\usepackage{outlines}
\usepackage{multirow}
\usepackage{wrapfig}
\usepackage{float}
\usepackage{colortbl}
\usepackage{pdflscape}
\usepackage{tabu}
\usepackage{threeparttable}
\usepackage{threeparttablex}
\usepackage[normalem]{ulem}
\usepackage{makecell}
\usepackage{xcolor}
\usepackage{afterpage}
\usepackage{titlesec}
\titleformat{\section}{\normalsize\bfseries}{\thesection.}{1em}{}
\usepackage{graphicx}
\usepackage{bbding}
\usepackage{xargs}
\usepackage{xpatch}
\usepackage[citecolor = blue]{hyperref}
\hypersetup{colorlinks = true, linkcolor = red, urlcolor = teal}

\usepackage{bookmark}
\usepackage{todonotes}
\setlength{\marginparwidth}{2cm}

\DeclareUnicodeCharacter{0301}{\'{e}}
\DeclareUnicodeCharacter{2212}{-}
\DeclareUnicodeCharacter{0327}{\c}

\usepackage[backend = biber, style=authoryear, sorting = nty, maxcitenames=1]{biblatex}

\addbibresource{citations_sesmarias.bib}

\graphicspath{{~/OneDrive - University of Illinois - Urbana/Research/Projects/Sesmarias Brazil/Figures/Descriptive/}}

%\addbibresource[location = remote]{https://raw.githubusercontent.com/ViniOkadaSilva/Papers/master/Sesmarias/citations_sesmarias.bib}

\DeclareFieldFormat{citehyperref}{%
  \DeclareFieldAlias{bibhyperref}{noformat}% Avoid nested links
  \bibhyperref{#1}}

\DeclareFieldFormat{textcitehyperref}{%
  \DeclareFieldAlias{bibhyperref}{noformat}% Avoid nested links
  \bibhyperref{%
    #1%
    \ifbool{cbx:parens}
      {\bibcloseparen\global\boolfalse{cbx:parens}}
      {}}}

\savebibmacro{cite}
\savebibmacro{textcite}

\renewbibmacro*{cite}{%
  \printtext[citehyperref]{%
    \restorebibmacro{cite}%
    \usebibmacro{cite}}}

\renewbibmacro*{textcite}{%
  \ifboolexpr{
    ( not test {\iffieldundef{prenote}} and
      test {\ifnumequal{\value{citecount}}{1}} )
    or
    ( not test {\iffieldundef{postnote}} and
      test {\ifnumequal{\value{citecount}}{\value{citetotal}}} )
  }
    {\DeclareFieldAlias{textcitehyperref}{noformat}}
    {}%
  \printtext[textcitehyperref]{%
    \restorebibmacro{textcite}%
    \usebibmacro{textcite}}}

\renewcommand*{\nameyeardelim}{\addcomma\space}

\usepackage{setspace}
\usepackage{graphicx}

\newcommand{\tinytable}[1]{\textcolor{black}{\tiny \input{#1}}}

\begin{document}

\begin{center}
\large \textbf{Summer Research Proposal}
\\
\normalsize \textbf{Colonial Portuguese Land Grants in Brazil: Long-term Effects on Inequality and Economic Development}
\\
\smallskip
\small Vinicius Okada da Silva
\end{center}

\vspace{-10mm}

\section{Background}

Brazil has one of the highest levels of inequality of land distribution in the world, with ``[a]n estimated 1\% of the population own[ing] 45\% of all land" \parencite{Usaid2016-xs}. However, this is not a recent issue as even in the 1920 census, the mean land inequality per municipality was already high \parencite{Wigton-Jones2020-ex}. 
[Add something here]
Historically, the first method for land grants in Brazil was through the request of \textit{sesmarias}.

The \textit{sesmarias} have their origin from a medieval Portuguese law established in 1375 that allowed the grant of small plots of land to be used and developed \parencite[p.~16]{Diegues_Junior1959-ba}. 
In Brazil, the law was initially implemented with little difference based on an updated version of the law from 1446.
The first official \textit{sesmaria} was granted in 1530, less than thirty years after the beginning of the colonization of Brazil \parencite[p.~16]{Diegues_Junior1959-ba}. 
The law governing the \textit{sesmarias} would have two more changes during Brazil's colonial period, in 1511 and 1603. 
However, the core of the law remained the same, the land was to be granted in order to establish settlements in Portuguese South America and allow it to develop at a low cost for the King. 

In order to request a land grant the petitioner was required to prove that he had the financial means to develop and use the land requested.
As a result, it often favored the settlers that had the financial means to request the land, the Portuguese aristocracy \parencite{Lobb1976-mc}. 
The granting of \textit{sesmarias} ended in 1822, briefly before Brazil's independence.
Early work argued it led to the development of the ``economic aristocracy of the colonial society'' and the ``principal cause of the \textit{latifundio}'' in colonial Brazil \parencites[p.~36]{Lima2002-kd}[p.~48]{Da_Costa_Porto1979-dz}. 
The long-term effects of the \textit{sesmarias} system are also described as ``[t]oday the system of ownership and use of land is a continuation of the colonial system, with the \textit{sesmaria} becoming \textit{latifundia} property" \parencite[p.~18]{Andrade1980-md}.

\section{Research Question}

Given the historical importance of the \textit{sesmarias} land grants in Brazil in establishing the basis of land usage in Brazil, the goal of this project is to establish a novel georeferenced database of Portuguese colonial land grants in Brazil and use it to identify the historical causes of land inequality and economic development in Brazil. I propose that are four main channels in which the \textit{sesmarias} could have long-term effects 

\begin{outline}
  \1 Land inequality: only those with sufficient financial conditions were granted \textit{sesmarias}, and were often granted vast plots of land.
  \1 Income and political inequality: land was the easiest way to accumulate wealth in colonial Brazil. Limiting the number of people with access to land would lead to wealth accumulation by the few.\footnote{``If the land was concentrated by a few owners, the \textit{latifundio} is created and it limits the number of settlers and the possibility of them entering the social class of \textit{senhores de engenho} or farmers" \parencite[p.~40]{Bandecchi1963-uj}}
  \1 Demographic Differences: \textit{Sesmarias} often required African slaves, which could skew the demographics of a location.\footnote{``Under the auspices of King Philip I (1581-1598), the \textit{sesmaria} was widely applied in the northeast and central coast regions of Brazil where a system involving large properties and slave labor was considered the only way to make a profit in the new land, whether by means of cultivation or cattle ranching''\parencite{Lobb1976-mc}.}
  \1 Economic Development: often the lands granted were developed by the owners, leading to the early economic development of an area.
  %\2 Urban development $\Rightarrow$ .
  % \1 Political corruption: Dominance by aristocrats often hampered efforts for local reform and investment. 
\end{outline}

\section{Significance}

This paper would contribute to the understanding of how colonial land grants have long-term development consequences within countries, and the role that institutions played in the long-term development of Brazil.  

Papers such as \textcite{Dell2010-qt} and \textcite{Lowes2020-pr} find long-term negative effects of extractive institutions measured by the colonial land assigned to mining in Peru and land concessions to private companies in Congo. 
\textcite{Dell2019-np}, on the contrary, finds positive economic effects of land assigned to sugar production in Indonesia. 
\textcite{Banerjee2005-ki} finds how areas under British colonial regime in India fare better off than others left to local elites.
Given the prominence of \textit{sesmarias} in Brazil as an extractive institution, especially in the Northeast with sugar plantations which often required a vast amount of slave labor, it wo

This paper also contributes to the understanding of the economic development of Brazil by trying to explain what are the historical causes of the diverging paths in development in each region. 
\textcite{Musacchio2014-pq} analyzes the role of local colonial institutions by in Brazil and how they led to diverging educational paths. 
\textcite{Naritomi2012-or} compares the negative effects on economic development from areas in Brazil that historically had more sugarcane plantations and gold mines. 
\textcite{Rocha2017-yq} finds positive effects of immigration in the state of Sao Paulo. 
\textcite{Wigton-Jones2020-ex} finds long-term effects on present-day inequality based on land inequality present at the 1920 census. 
\textcite{Laudares2022-vy} finds that municipalities to the right of the Treaty of Tordesillas line are 

\section{Data Collection}

Most of the current work in the project would require work in cooperation with researchers in Brazil. 
Part of the project requires the transcription of the original manuscripts into text documents.
Once the text is available, then some information such as the date of the request, to whom the petition was asked, the justification for the use of the land, geographical markers, total area, the conditions to which the land was to be granted, and characteristics of the person requesting the land are extracted.\footnote{The current database, called the \textit{Sesmarias of the Luso-Portuguese Empire}, is available at \url{http://plataformasilb.cchla.ufrn.br/}.} 
% Currently, around 15-20\% of all the letters have been transcribed and had their information extracted and uploaded to the website.

Given the geographical information extracted, part of the project is to georeference the original location of the \textit{sesmaria}. This is possible by using geographical markers (such as proximity to a river or a city) to get an estimate of the original location of the land grant.
Once each \textit{sesmaria} is georeferenced and has the necessary information about it I would then be able to connect each point to Brazilian censuses from 1872-2010 and other sources of historical data.

\newpage

\xpatchbibmacro{author}{\printnames{author}}{\textbf{\printnames{author}}}{}{}

\printbibliography

\appendix

\end{document}
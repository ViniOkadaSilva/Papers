\documentclass[aspectratio=1610]{beamer}
%\linespread{1.5}\selectfont

\usepackage{booktabs}
\usepackage{siunitx}
\usepackage{dcolumn}
\usepackage{float}
\usepackage{placeins}
\usepackage{lscape} 
\usepackage{tikz}
\usepackage[export]{adjustbox}
\usepackage{ragged2e}
\justifying
\usepackage{outlines}
\usepackage{amsmath}
\usepackage{booktabs}
\usepackage{float}
\usepackage{dcolumn}
\usepackage{longtable}
\usepackage{array}
\usepackage{multirow}
\usepackage{wrapfig}
\usepackage{float}
\usepackage{colortbl}
\usepackage{pdflscape}
\usepackage{tabu}
\usepackage{threeparttable}
\usepackage{caption}\captionsetup{labelformat=empty}
%\captionsetup{font=footnotesize}
\usepackage{subcaption}
\usepackage{threeparttable}
\usepackage[normalem]{ulem}
\usepackage{makecell}
\usepackage{xcolor}
\usepackage{hyperref}
\hypersetup{
    colorlinks = true, 
    linkcolor = red, 
    urlcolor = teal, 
    citecolor = blue}

%\usepackage{caption}
%\captionsetup{labelformat=empty}
\usepackage{appendixnumberbeamer}
\renewcommand{\raggedright}{\leftskip=0pt \rightskip=20pt plus 0cm}

\DeclareUnicodeCharacter{0301}{\'{e}}
\DeclareUnicodeCharacter{2212}{-}
\DeclareUnicodeCharacter{0327}{\c}

\usepackage[backend = biber, style=authoryear, sorting = nty, maxcitenames=1]{biblatex}

\addbibresource{citations_sesmarias.bib}

\graphicspath{{~/OneDrive - University of Illinois - Urbana/Research/Projects/Sesmarias Brazil/Figures/Descriptive/}}

%\addbibresource[location = remote]{https://raw.githubusercontent.com/ViniOkadaSilva/Papers/master/Sesmarias/citations_sesmarias.bib}

\DeclareFieldFormat{citehyperref}{%
  \DeclareFieldAlias{bibhyperref}{noformat}% Avoid nested links
  \bibhyperref{#1}}

\DeclareFieldFormat{textcitehyperref}{%
  \DeclareFieldAlias{bibhyperref}{noformat}% Avoid nested links
  \bibhyperref{%
    #1%
    \ifbool{cbx:parens}
      {\bibcloseparen\global\boolfalse{cbx:parens}}
      {}}}

\savebibmacro{cite}
\savebibmacro{textcite}

\renewbibmacro*{cite}{%
  \printtext[citehyperref]{%
    \restorebibmacro{cite}%
    \usebibmacro{cite}}}

\renewbibmacro*{textcite}{%
  \ifboolexpr{
    ( not test {\iffieldundef{prenote}} and
      test {\ifnumequal{\value{citecount}}{1}} )
    or
    ( not test {\iffieldundef{postnote}} and
      test {\ifnumequal{\value{citecount}}{\value{citetotal}}} )
  }
    {\DeclareFieldAlias{textcitehyperref}{noformat}}
    {}%
  \printtext[textcitehyperref]{%
    \restorebibmacro{textcite}%
    \usebibmacro{textcite}}}

\renewcommand*{\nameyeardelim}{\addcomma\space}

\usepackage{setspace}
\usepackage{graphicx}

\newcommand{\tinytable}[1]{\textcolor{black}{\tiny \input{#1}}}

\graphicspath{{~/OneDrive - University of Illinois - Urbana/Research/Writing/git/Sesmarias/Pictures/}}

\beamertemplatenavigationsymbolsempty

%Information to be included in the title page:
\title{Portuguese Colonial Land Grants in Brazil: Long-term Effects on Inequality and Economic Development}
\author{Vinicius Okada da Silva [5th Year]}
\date{}

\setbeamertemplate{footline}[frame number]

\begin{document}


\begin{frame}[plain, noframenumbering]
	\titlepage
    \begin{center}
        Committee: \textbf{Richard Akresh (Chair), Marieke Kleemans, Felipe Valencia Caicedo (UBC), Mary Arends-Kuenning (ACE)}
    \end{center}
\end{frame}

\begin{frame}{Feedback}
    
\end{frame}

\begin{frame}{Motivation}
    \begin{outline}
        \1 Inequality in access to land has been .
            \vspace{2mm}
            \2 ``\textcolor{red}{\textbf{Brazil has one of the highest levels of inequality of land distribution in the world}} [...] \textcolor{red}{\textbf{An estimated 1\% of the population owns 45\% of all land in Brazil}}.'' \parencite{Usaid2016-xs}
        \1 ``The agrarian problem is one of the most serious [Brazil] has, because of the great concentration of land ownership and the low level of utilization by the large and medium property owners" \parencite[p.~1]{De_Oliveira_Andrade1980-xz}
    \end{outline}
\end{frame}

\begin{frame}{Research Question}
    \begin{outline}
        \1 How much of economic development and inequality can be traced to colonial institutions?
            \vspace{2mm}
            \2 Goal of this research would analyze the effects of colonial Portuguese land grants (\textit{sesmarias}) on long-term development and inequality in Brazil.
            \vspace{2mm}
        % Cite here the papers that have used these identifications.
        \pause 
        \1 Proposed Identification:
            \2 Exploit a 1701 law that banned livestock grazing within 80km of the coast of Brazil.
    \end{outline}
\end{frame}

\begin{frame}{Possible Channels}
    \begin{outline}
        \1 What are the long-term economic effects of colonial Portuguese land grants in Brazil?
        \vspace{2mm}
        \pause 
            \2 Land Tenure $\Rightarrow$ Different sizes of land grants could lead to different utilization of land, and how owners would exploit it.
            \pause 
            \vspace{2mm}
            \2 Demographic Differences $\Rightarrow$ Land grants often required African slaves, which could skew the demographics of a location.
            \pause 
    \end{outline}
\end{frame}

\begin{frame}{Diagram with History of Land Concessions in Brazil}
    
\end{frame}

\begin{frame}{Diagram with Proposed Channels}
    
\end{frame}

\begin{frame}{Contribution}
    \begin{outline}
        \1 Role of colonization, institutions, and land tenure in present outcomes:
            \vspace{2mm}
            \2 Institutional and Natural Endowments: \cite{Acemoglu2001-dz} (AER), \cite{Sokoloff2000-mb} (JEP). 
            \vspace{2mm}
            \2 Americas: \cite{Naritomi2012-or} (JEH), 
            \cite{Musacchio2014-pq} (JEH),
            \cite{Wigton-Jones2020-ex} (JEG),
            \cite{Laudares2022-vy} (WP),
            \cite{Sellars2018-yp} (JDE),
            \cite{Smith2023-ip} (WP)
            \vspace{2mm}
            \2 India and Africa: 
            \cites{Banerjee2005-ki} (AER)
    \end{outline}
\end{frame}

\begin{frame}{Data}
    \begin{outline}
        \1 Land Grant Locations:
            \2 Information on the land grants from the \href{http://plataformasilb.cchla.ufrn.br/}{Sesmarias of the Luso-Brazilian Empire Database}
        \vspace{2mm}
        \1 Establish that they had an effect in the past:
            \2 1872 Brazilian Census [\textbf{Novel Data for 1872 at a Fine Geographical Level}]
        \vspace{2mm}
        \1 Present-Day Effects on Land Tenure
            \2 1995 Brazilian Agricultural Census
    \end{outline}
\end{frame}

\begin{frame}{1872 Parish Level Information}{[New Data]}
    \begin{figure}[h!]
        \begin{center}
           \makebox[\textwidth]			 
           {\includegraphics[width=0.85\paperwidth]{/Users/vinicius/Library/CloudStorage/OneDrive-UniversityofIllinois-Urbana/Research/Projects/JMP/02. Figures/00.Maps/parishes_1872.png}}
        \end{center}
        \label{fig:parishes_1872}
      \end{figure}
\end{frame}

% Be more clear on agriculture
% Interact with land grants 
% Put somewhere the Y variables I'm taking a look at 
% Make very clear on what I'm using
% Show first the OLS - main things that are going on 
% Coast_Dist
% What the story: be more clear 
% I can show the RDD
% Make clear 
% Run for separate regions
% Show in Table form 
% Maybe check if the law had a bite in a different 

\begin{frame}{Identification Strategy}{Coastal Ban on Livestock}
    \begin{outline}
        \1 In 1701, the Portuguese Crown enacted a ban on cattle ranching from 80km of the coast (10 leagues) \parencites[p~.40]{Fausto2014-bh}[p~.198]{Simonsen2005-ps}[p~.460]{Bethell1984-of}.
        \1 ``Landholding in the [interior] was truly extensive [...] The sesmarias on which cattle ranches were established sometimes exceeded hundreds of thousands of acres" \parencite{Bethell1984-of} 
        \1 ``Extensive cattle raising, with open grazing, did not require much attention or labor. For that reason, the number of slaves in the region was small'' \parencite[p.~113]{De_Oliveira_Andrade1980-xz}
        \1 Regression Discontinuity design exploiting this 80km cutoff.
    \end{outline}
    \vspace{2mm}
    
    \begin{equation}
    Y_{m,s} = \beta \cdot CoastDist_{m,s} + f(D_{m,s})+ \mu_s + X_{m,s} + \epsilon_{m,s}
    \end{equation}
\end{frame}

\begin{frame}{Distribution of Land Grants pre- and post- 1701}
    \begin{figure}
        \begin{center}
           \makebox[\textwidth]			 
           {\includegraphics[width=0.85\paperwidth]{/Users/vinicius/Library/CloudStorage/OneDrive-UniversityofIllinois-Urbana/Research/Projects/JMP/02. Figures/00.Maps/land_grant_distribution_1701.png}}
        \end{center}
        \label{fig:SesmariasDistribution}
      \end{figure}      
\end{frame}

\begin{frame}{1872 Results}{Agriculture}
    \begin{figure}
        \begin{center}
           \makebox[\textwidth]			 
           {\includegraphics[width=0.85\paperwidth]{/Users/vinicius/Library/CloudStorage/OneDrive-UniversityofIllinois-Urbana/Research/Projects/JMP/02. Figures/00.Maps/1872_agriculture_RDD.png}}
        \end{center}
      \end{figure}
\end{frame}

\begin{frame}{1872 Results}{Slavery}
    \begin{figure}
        \begin{center}
           \makebox[\textwidth]			 
           {\includegraphics[width=0.85\paperwidth]{/Users/vinicius/Library/CloudStorage/OneDrive-UniversityofIllinois-Urbana/Research/Projects/JMP/02. Figures/00.Maps/1872_slaves_RDD.png}}
        \end{center}
      \end{figure}
\end{frame}

\begin{frame}{1872 Results}{Slavery}
    \begin{figure}
        \begin{center}
           \makebox[\textwidth]			 
           {\includegraphics[width=0.85\paperwidth]{/Users/vinicius/Library/CloudStorage/OneDrive-UniversityofIllinois-Urbana/Research/Projects/JMP/02. Figures/00.Maps/1872_Census_Enslaved_People.png}}
        \end{center}
    \end{figure}
\end{frame}

\begin{frame}{1872 RD vs. OLS}{Demographics}
    \small
    \input{~/OneDrive - University of Illinois - Urbana/Research/Projects/JMP/03. Tables/1872_OLS_Demographics.tex}
\end{frame}

\begin{frame}{1985 LandSat RD vs. OLS}{Land Usage}
    
\end{frame}

\begin{frame}{1872 RD vs. OLS}{Descriptive OLS - Labor}
    \footnotesize
    \input{~/OneDrive - University of Illinois - Urbana/Research/Projects/JMP/03. Tables/1872_OLS_Labor.tex}
\end{frame}

\begin{frame}{1995 Agricultural Census Results}{Occupied Land}
    \begin{figure}
        \begin{center}
           \makebox[\textwidth]			 
           {\includegraphics[width=0.85\paperwidth]{/Users/vinicius/Library/CloudStorage/OneDrive-UniversityofIllinois-Urbana/Research/Projects/JMP/02. Figures/00.Maps/1995_Ag_Census_Occupied_RDD.png}}
        \end{center}
    \end{figure}
\end{frame}

\begin{frame}{1995 Agricultural Census Results}{Occupied Land}
    \begin{figure}
        \begin{center}
           \makebox[\textwidth]			 
           {\includegraphics[width=0.85\paperwidth]{/Users/vinicius/Library/CloudStorage/OneDrive-UniversityofIllinois-Urbana/Research/Projects/JMP/02. Figures/00.Maps/ag_census_1995_occupied_map.png}}
        \end{center}
        \label{fig:occupied_area_1995}
    \end{figure}
\end{frame}

\begin{frame}{1995 Agricultural Census Results}{Productive Land Not Used}
    \begin{figure}
        \begin{center}
           \makebox[\textwidth]			 
           {\includegraphics[width=0.85\paperwidth]{/Users/vinicius/Library/CloudStorage/OneDrive-UniversityofIllinois-Urbana/Research/Projects/JMP/02. Figures/00.Maps/1995_Ag_Census_Productive_Land_Not_Used_RDD.png}}
        \end{center}
    \end{figure}
\end{frame}

\begin{frame}{1995 Agricultural Census Results}{Productive Land Not Used}
    \begin{figure}
        \begin{center}
           \makebox[\textwidth]			 
           {\includegraphics[width=0.85\paperwidth]{/Users/vinicius/Library/CloudStorage/OneDrive-UniversityofIllinois-Urbana/Research/Projects/JMP/02. Figures/00.Maps/ag_census_1995_productive_land_not_used_map.png}}
        \end{center}
      \end{figure}
\end{frame}

\begin{frame}{1995 Agricultural Census}{Leased Land}
    \begin{figure}[h!]
        \begin{center}
           \makebox[\textwidth]			 
           {\includegraphics[width=0.85\paperwidth]{/Users/vinicius/Library/CloudStorage/OneDrive-UniversityofIllinois-Urbana/Research/Projects/JMP/02. Figures/00.Maps/ag_census_1995_leased_map.png}}
        \end{center}
      \end{figure}
\end{frame}

\begin{frame}{1995 RD vs. OLS}{Agricultural Land Usage}
    \small
    \input{~/OneDrive - University of Illinois - Urbana/Research/Projects/JMP/03. Tables/1995_Ag_Census_Results.tex}
\end{frame}

\begin{frame}{1970 Census}{People Working in Sugarcane Production}
    \begin{figure}[h!]
        \begin{center}
           \makebox[\textwidth]			 
           {\includegraphics[width=0.85\paperwidth]{/Users/vinicius/Library/CloudStorage/OneDrive-UniversityofIllinois-Urbana/Research/Projects/JMP/02. Figures/00.Maps/coastal_rdd_1970_sugarcane.png}}
        \end{center}
      \end{figure}
\end{frame}

\begin{frame}{Conclusion}
    
\end{frame}

\begin{frame}[allowframebreaks, t, noframenumbering, plain]{References}
    \printbibliography
\end{frame}

\appendix

\begin{frame}{History/Background}{Request Process}
    \begin{outline}
        \1 Petitioner submits a letter for an unoccupied land detailing their qualifications (captain, governor, etc.)
        \vspace{1mm}
        \pause 
        \1 Governor reads it, and if accepted returns back a letter with the requirements for the petitioner to satisfy.
        \vspace{1mm}
        \pause 
        \1 Five years to develop the land
        \vspace{1mm}
        \pause 
        \1 If successful, upon an inspection, land was transferred to the \textit{sesmeiro}.
        \vspace{1mm}
        \pause 
        \1 Able to sell, pass down as inheritance, etc. 
    \end{outline}
\end{frame}

\begin{frame}{Selection}\hypertarget{selection}{}
    \begin{outline}
        \1 Agglomeration: \hyperlink{agglomeration}{\beamerbutton{Effects on Neighboring Grids}}
    \end{outline}
\end{frame}

\begin{frame}{Identification}{Exploring the Content of the Letters}
    \begin{outline}
        \1 Focus on the letters and their contents. 
        \vspace{2mm}
        \1 Make the unit of observation a state by year. 
        \vspace{2mm}
        \1 \textbf{Example Research Question:} How would a change in state governorship affect the contents of the letter:
        \vspace{1mm}
            \2 \textbf{Channel:} New governor, not enough information on how strict he would be enforcing the land grants $\Rightarrow$ the letters are longer and more specific.
        \vspace{1mm}
    \end{outline}
\end{frame}

\begin{frame}{Other Relevant (?) Information to Add}
    \begin{outline}
        \1 \textit{Sesmarias} caused economic uncertainty in colonial times as often poor people would settle, develop land, and then lose the right of the land because a richer person would claim it \parencite[p.~142]{Da_Costa_Porto1979-dz}.
    \end{outline}
\end{frame}

\begin{frame}{Manueline Ordinances 1511-1512}
    ``Na petição por uma carta de sesmaria, o requerente devia justificar seu pedido, e quando recebesse a carta de concessão havia uma serie de obrigações entre as quais estava a necessidade do cultivo''
\end{frame}

\begin{frame}{\hypertarget{1872_slavery} 1872 Census - Slavery Distribution
    \hyperlink{maps}{\beamerbutton{Back}}} 
    
    \includegraphics[width=0.5\textwidth]
    {~/OneDrive - University of Illinois - Urbana/Research/Projects/Sesmarias Brazil/Figures/01. Maps/slave_proportion.png}
    \includegraphics[width=0.5\textwidth]
    {~/OneDrive - University of Illinois - Urbana/Research/Projects/Sesmarias Brazil/Figures/01. Maps/total_slaves.png}
\end{frame}

\begin{frame}{\hypertarget{1872_sugarcane} 1872 Census - Potential Sugarcane Output
    \hyperlink{IV}{\beamerbutton{Back}}} 
    \centering
    \includegraphics[width=0.95\textwidth]
    {~/OneDrive - University of Illinois - Urbana/Research/Projects/Sesmarias Brazil/Figures/01. Maps/sugarcane_production.png}
\end{frame}

\begin{frame}{\hypertarget{2010_sugarcane} 2010 Census - Potential Sugarcane Output
    \hyperlink{IV}{\beamerbutton{Back}}} 
    \centering
    \includegraphics[width=0.95\textwidth]
    {~/OneDrive - University of Illinois - Urbana/Research/Projects/Sesmarias Brazil/Figures/01. Maps/sugarcane_production_2010.png}
\end{frame}

\begin{frame}{\hypertarget{1872_gender} 1872 Census - Gender Distribution
    \hyperlink{maps}{\beamerbutton{Back}}} 
    %\hypertarget{1872_slavery}
    \includegraphics[width=0.5\textwidth]
    {~/OneDrive - University of Illinois - Urbana/Research/Projects/Sesmarias Brazil/Figures/01. Maps/gender_ratio.png}
    \includegraphics[width=0.5\textwidth]
    {~/OneDrive - University of Illinois - Urbana/Research/Projects/Sesmarias Brazil/Figures/01. Maps/gender_ratio_slaves.png}
\end{frame}

\begin{frame}{Basic Descriptive Statistics}
    {Year Dist. \hyperlink{maps}{\beamerbutton{Back}}}
    \centering
    \includegraphics[width=0.85\textwidth,height=0.85\textheight,keepaspectratio]
    {~/OneDrive - University of Illinois - Urbana/Research/Projects/Sesmarias Brazil/Figures/Descriptive/year_dist.png}
\end{frame}

\begin{frame}{Basic Descriptive Statistics (1 hec = 2.5 Football Fields)}
    {Size Dist. \hyperlink{maps}{\beamerbutton{Back}}} 
    \hypertarget{year_dist}
    \centering
    \includegraphics[width=0.85\textwidth,height=0.85\textheight,keepaspectratio]
    {~/OneDrive - University of Illinois - Urbana/Research/Projects/Sesmarias Brazil/Figures/Descriptive/size_dist.png}
\end{frame}

\begin{frame}{Basic Descriptive Statistics (1 hec = 2.5 Football Fields)}
    {Size Dist. \hyperlink{data}{\beamerbutton{Back}}}
    \hypertarget{year_dist_1697}
    \centering
    \includegraphics[width=0.85\textwidth,height=0.85\textheight,keepaspectratio]
    {~/OneDrive - University of Illinois - Urbana/Research/Projects/Sesmarias Brazil/Figures/Descriptive/size_dist_1697.png}
\end{frame}

\begin{frame}{Smallest Land Grant}
    \begin{outline}
        \1 The smallest land grant we have in the dataset is from 1603, in Rio de Janeiro (RJ0118). The petitioner asked for some land to build a house in the city of São Sebastião, which explains why in hectares it is so small.
    \end{outline}
\end{frame}

\begin{frame}{Basic Descriptive Statistics}{No Obs. in Pernambuco}
    \centering
    \includegraphics[width=0.85\textwidth,height=0.85\textheight,keepaspectratio]
    {~/OneDrive - University of Illinois - Urbana/Research/Projects/Sesmarias Brazil/Figures/Descriptive/year_dist_PE.png}
\end{frame}

\begin{frame}{
    \hypertarget{sugar}
    Georeferenced Land Grants
    \hyperlink{maps}{\beamerbutton{Back}}}
    {Sugarcane vs. Ranching}
    \centering
    \includegraphics[width=0.85\textwidth,height=0.85\textheight,keepaspectratio]
    {~/OneDrive - University of Illinois - Urbana/Research/Projects/Sesmarias Brazil/Figures/01. Maps/sugar_pasture.png}
\end{frame}

\begin{frame}{
    \hypertarget{discovery}
    Georeferenced Land Grants
    \hyperlink{maps}{\beamerbutton{Back}}}
    {Alleged Discovery of the Land}
    \centering
    \includegraphics[width=0.85\textwidth,height=0.85\textheight,keepaspectratio]
    {~/OneDrive - University of Illinois - Urbana/Research/Projects/Sesmarias Brazil/Figures/01. Maps/discovery.png}
\end{frame}

\begin{frame}{
    \hypertarget{discovery_year}
    Georeferenced Land Grants 
    \hyperlink{maps}{\beamerbutton{Back}}}
    {Alleged Discovery of the Land}
    \centering
    \includegraphics[width=0.85\textwidth,height=0.85\textheight,keepaspectratio]
    {~/OneDrive - University of Illinois - Urbana/Research/Projects/Sesmarias Brazil/Figures/01. Maps/discovery_year.png}
\end{frame}

\begin{frame}{
    \hypertarget{no_land}
    Georeferenced Land Grants 
    \hyperlink{maps}{\beamerbutton{Back}}}
    {Alleged Discovery of the Land}
    \centering
    \includegraphics[width=0.85\textwidth,height=0.85\textheight,keepaspectratio]
    {~/OneDrive - University of Illinois - Urbana/Research/Projects/Sesmarias Brazil/Figures/01. Maps/no_land.png}
\end{frame}

\begin{frame}{
    \hypertarget{year}
    Georeferenced Land Grants
    \hyperlink{maps}{\beamerbutton{Back}}}
    {Year of the Land Grant}
    \centering
    \includegraphics[width=0.85\textwidth,height=0.85\textheight,keepaspectratio]
    {~/OneDrive - University of Illinois - Urbana/Research/Projects/Sesmarias Brazil/Figures/01. Maps/year_distribution.png}
\end{frame}

\begin{frame}{
    \hypertarget{no_land}
    Georeferenced Land Grants 
    \hyperlink{maps}{\beamerbutton{Back}}}
    {Alleged Discovery of the Land}
    \centering
    \includegraphics[width=0.85\textwidth,height=0.85\textheight,keepaspectratio]
    {~/OneDrive - University of Illinois - Urbana/Research/Projects/Sesmarias Brazil/Figures/01. Maps/no_land.png}
\end{frame}

\begin{frame}{
    \hypertarget{no_land}
    Georeferenced Land Grants 
    \hyperlink{maps}{\beamerbutton{Back}}}
    {Alleged Discovery of the Land}
    \centering
    \includegraphics[width=0.85\textwidth,height=0.85\textheight,keepaspectratio]
    {~/OneDrive - University of Illinois - Urbana/Research/Projects/Sesmarias Brazil/Figures/01. Maps/pasture_1985_2010.png}
\end{frame}

\begin{frame}{
    \hypertarget{no_land}
    Georeferenced Land Grants 
    \hyperlink{maps}{\beamerbutton{Back}}}
    {Alleged Discovery of the Land}
    \centering
    \includegraphics[width=0.85\textwidth,height=0.85\textheight,keepaspectratio]
    {~/OneDrive - University of Illinois - Urbana/Research/Projects/Sesmarias Brazil/Figures/01. Maps/sugarcane_1985_2010.png}
\end{frame}

\begin{frame}{
    \hypertarget{no_land}
    Georeferenced Land Grants 
    \hyperlink{maps}{\beamerbutton{Back}}}
    {Alleged Discovery of the Land}
    \centering
    \includegraphics[width=0.85\textwidth,height=0.85\textheight,keepaspectratio]
    {~/OneDrive - University of Illinois - Urbana/Research/Projects/Sesmarias Brazil/Figures/01. Maps/agriculture_pasture_1985_2010.png}
\end{frame}

\begin{frame}{
    \hypertarget{no_land}
    Georeferenced Land Grants 
    \hyperlink{maps}{\beamerbutton{Back}}}
    {Alleged Discovery of the Land}
    \centering
    \includegraphics[width=0.85\textwidth,height=0.85\textheight,keepaspectratio]
    {~/OneDrive - University of Illinois - Urbana/Research/Projects/Sesmarias Brazil/Figures/01. Maps/forest_1985_2010.png}
\end{frame}

\begin{frame}{
    \hypertarget{pasture_grid}
    Georeferenced Land Grants 
    \hyperlink{grid_maps}{\beamerbutton{Back}}}
    {Pasture request + Land Usage in 2000}
    \centering
    \includegraphics[width=0.85\textwidth,height=0.85\textheight,keepaspectratio]
    {~/OneDrive - University of Illinois - Urbana/Research/Projects/Sesmarias Brazil/Figures/01. Maps/pasture_2000_grid.png}
\end{frame}

\begin{frame}{
    \hypertarget{psugarcane_grid}
    Georeferenced Land Grants 
    \hyperlink{maps}{\beamerbutton{Back}}}
    {Sugarcane request + Potential Sugarcane Production}
    \centering
    \includegraphics[width=0.85\textwidth,height=0.85\textheight,keepaspectratio]
    {~/OneDrive - University of Illinois - Urbana/Research/Projects/Sesmarias Brazil/Figures/01. Maps/sugarcane_potential_2000_grid.png}
\end{frame}

\begin{frame}{
    \hypertarget{sugarcane_grid}
    Georeferenced Land Grants 
    \hyperlink{maps}{\beamerbutton{Back}}}
    {Sugarcane request + Area used for Sugarcane (2000)}
    \centering
    \includegraphics[width=0.85\textwidth,height=0.85\textheight,keepaspectratio]
    {~/OneDrive - University of Illinois - Urbana/Research/Projects/Sesmarias Brazil/Figures/01. Maps/sugarcane_usage_2000_grid.png}
\end{frame}

\begin{frame}{
    \hypertarget{nightlight_grid}
    Georeferenced Land Grants 
    \hyperlink{maps}{\beamerbutton{Back}}}
    {All requests + Nightlight data (2000)}
    \centering
    \includegraphics[width=0.85\textwidth,height=0.85\textheight,keepaspectratio]
    {~/OneDrive - University of Illinois - Urbana/Research/Projects/Sesmarias Brazil/Figures/01. Maps/nightlight_2000_grid.png}
\end{frame}


\begin{frame}{
    \hypertarget{sugar_table}
    First-Stage Estimates - 1872 
    \hyperlink{IV}{\beamerbutton{Back}}}
    \centering
    \input{~/OneDrive - University of Illinois - Urbana/Research/Projects/Sesmarias Brazil/Tables/sugar_first_stage.tex}
\end{frame}

\begin{frame}{
    \hypertarget{sugar_table_2010}
    First-Stage Estimates - 2010
    \hyperlink{IV}{\beamerbutton{Back}}}
    \centering
    \input{~/OneDrive - University of Illinois - Urbana/Research/Projects/Sesmarias Brazil/Tables/sugar_first_stage_2010.tex}
\end{frame}

\begin{frame}{
    \hypertarget{sugar_table_2010}
    First-Stage Estimates - Grid
    \hyperlink{IV}{\beamerbutton{Back}}}
    \centering
    \input{~/OneDrive - University of Illinois - Urbana/Research/Projects/Sesmarias Brazil/Tables/sugar_first_stage_grid.tex}
\end{frame}

\begin{frame}{
    \hypertarget{econ_activities}
    OLS - Economic Activities
    \hyperlink{maps}{\beamerbutton{Back}}}
    \centering
    \input{~/OneDrive - University of Illinois - Urbana/Research/Projects/Sesmarias Brazil/Tables/economic_activity_grid.tex}
\end{frame}

\begin{frame}{
    \hypertarget{econ_dev}
    OLS - Economic Development
    \hyperlink{maps}{\beamerbutton{Back}}}
    \centering
    \input{~/OneDrive - University of Illinois - Urbana/Research/Projects/Sesmarias Brazil/Tables/economic_development_grid.tex}
\end{frame}

\begin{frame}{
    \hypertarget{land_usage_year}
    OLS - Land Usage
    \hyperlink{maps}{\beamerbutton{Back}}}
    \centering
    \input{~/OneDrive - University of Illinois - Urbana/Research/Projects/Sesmarias Brazil/Tables/land_usage_year_grid.tex}
\end{frame}

\begin{frame}{
    \hypertarget{agglomeration}
    OLS - Agglomeration
    \hyperlink{selection}{\beamerbutton{Back}}}
    \centering
    \input{~/OneDrive - University of Illinois - Urbana/Research/Projects/Sesmarias Brazil/Tables/agglomeration_grid.tex}
\end{frame}

\begin{frame}
    {Balance Tables \hyperlink{data}{\beamerbutton{Back}}}
    \hypertarget{grid_balance}
    {Grid Level}
    \begin{center}
    \input{~/OneDrive - University of Illinois - Urbana/Research/Projects/Sesmarias Brazil/Tables/summary_sesmarias_grid.tex}
    \end{center}
\end{frame}

\begin{frame}[shrink=25]{
    \hypertarget{year_2}
    Balance Tables 
    \hyperlink{data}{\beamerbutton{Back}}}
    {1872 Municipality Level}
    \begin{center}
    \input{~/OneDrive - University of Illinois - Urbana/Research/Projects/Sesmarias Brazil/Tables/summary_1872.tex}
    \end{center}
\end{frame}

\begin{frame}{
    \hypertarget{land_grant_level}
    Balance Tables 
    \hyperlink{data}{\beamerbutton{Back}}}
    {\textcolor{white}{Land Grant Level}}
    \centering
    \input{~/OneDrive - University of Illinois - Urbana/Research/Projects/Sesmarias Brazil/Tables/summary_sesmarias.tex}
\end{frame}

\end{document}
\documentclass[aspectratio=1610]{beamer}
%\linespread{1.5}\selectfont

\usepackage{graphicx}
\usepackage{booktabs}
\usepackage{siunitx}
\usepackage{dcolumn}
\usepackage{float}
\usepackage{placeins}
\usepackage{lscape} 
\usepackage{tikz}
\usepackage[export]{adjustbox}
\usepackage{ragged2e}
\usepackage{centernot}
\justifying
\usepackage{outlines}
\usepackage{amsmath}
\usepackage{booktabs}
\usepackage{float}
\usepackage{dcolumn}
\usepackage{longtable}
\usepackage{array}
\usepackage{multirow}
\usepackage{wrapfig}
\usepackage{float}
\usepackage{colortbl}
\usepackage{pdflscape}
\usepackage{tabu}
\usepackage{threeparttable}
\usepackage{caption}\captionsetup{labelformat=empty}
%\captionsetup{font=footnotesize}
\usepackage{subcaption}
\usepackage{threeparttable}
\usepackage[normalem]{ulem}
\usepackage{makecell}
\usepackage{xcolor}
\usepackage{hyperref}
\hypersetup{
    colorlinks = true, 
    linkcolor = red, 
    urlcolor = teal, 
    citecolor = blue}

%\usepackage{caption}
%\captionsetup{labelformat=empty}
\usepackage{appendixnumberbeamer}
\renewcommand{\raggedright}{\leftskip=0pt \rightskip=20pt plus 0cm}

\DeclareUnicodeCharacter{0301}{\'{e}}
\DeclareUnicodeCharacter{2212}{-}
\DeclareUnicodeCharacter{0327}{\c}

\usepackage[backend = biber, style=authoryear, sorting = nty, maxcitenames=1]{biblatex}

\addbibresource{citations_sesmarias.bib}

\graphicspath{{~/OneDrive - University of Illinois - Urbana/Research/Projects/Sesmarias Brazil/Figures/Descriptive/}}

%\addbibresource[location = remote]{https://raw.githubusercontent.com/ViniOkadaSilva/Papers/master/Sesmarias/citations_sesmarias.bib}

\DeclareFieldFormat{citehyperref}{%
  \DeclareFieldAlias{bibhyperref}{noformat}% Avoid nested links
  \bibhyperref{#1}}

\DeclareFieldFormat{textcitehyperref}{%
  \DeclareFieldAlias{bibhyperref}{noformat}% Avoid nested links
  \bibhyperref{%
    #1%
    \ifbool{cbx:parens}
      {\bibcloseparen\global\boolfalse{cbx:parens}}
      {}}}

\savebibmacro{cite}
\savebibmacro{textcite}

\renewbibmacro*{cite}{%
  \printtext[citehyperref]{%
    \restorebibmacro{cite}%
    \usebibmacro{cite}}}

\renewbibmacro*{textcite}{%
  \ifboolexpr{
    ( not test {\iffieldundef{prenote}} and
      test {\ifnumequal{\value{citecount}}{1}} )
    or
    ( not test {\iffieldundef{postnote}} and
      test {\ifnumequal{\value{citecount}}{\value{citetotal}}} )
  }
    {\DeclareFieldAlias{textcitehyperref}{noformat}}
    {}%
  \printtext[textcitehyperref]{%
    \restorebibmacro{textcite}%
    \usebibmacro{textcite}}}

\renewcommand*{\nameyeardelim}{\addcomma\space}

\usepackage{setspace}
\usepackage{graphicx}

\newcommand{\tinytable}[1]{\textcolor{black}{\tiny \input{#1}}}

\graphicspath{{~/OneDrive - University of Illinois - Urbana/Research/Writing/git/Sesmarias/Pictures/}}

\beamertemplatenavigationsymbolsempty

%Information to be included in the title page:
\title{Portuguese Colonial Land Grants in Brazil: Long-term Effects on Inequality and Economic Development}
\author{Vinicius Okada da Silva [5th Year]}
\date{\today}

\setbeamertemplate{footline}[frame number]

\begin{document}

\setbeamertemplate{itemize items}[circle]

\begin{frame}[plain, noframenumbering]
	\titlepage
    \begin{center}
        Committee: \textbf{Richard Akresh (Chair), Marieke Kleemans, Felipe Valencia Caicedo (UBC), Mary Arends-Kuenning (ACE)}
    \end{center}
\end{frame}

\begin{frame}{Feedback}
    \begin{enumerate}
        \item \textbf{Data}: Best way to connect the land grants to the identification.
        \vspace{2mm}
        \item \textbf{Identification}: Any other ways to combine what I am presenting today 
        \vspace{2mm}
        \item \textbf{Results + Motivation}: Plenty of null results (reduced form) - how does that fit into the story/paper?
    \end{enumerate}
\end{frame}

\begin{frame}{Background and Motivation}
    \begin{outline}
        \1 Inequality in access to land is a key issue in Brazil.
            \vspace{2mm}
            \2 ``\textcolor{red}{\textbf{Brazil has one of the highest levels of inequality of land distribution in the world}} [...] \textcolor{red}{\textbf{An estimated 1\% of the population owns 45\% of all land in Brazil}}.'' \parencite{Usaid2016-xs}
        \vspace{2mm}
        \pause 
        \1 ``The agrarian problem is one of the most serious problems [Brazil] has, because of the great concentration of land ownership and the low level of utilization by the large and medium property owners" \parencite[p.~1]{De_Oliveira_Andrade1980-xz}
    \end{outline}
\end{frame}

\begin{frame}{Research Question}
    \begin{outline}
        \1 How much of economic development and inequality can be traced to historical land grants in Brazil?
        \vspace{2mm}
            \2 Today I will focus on possible first channels.
        \vspace{2mm}
        \pause 
        \1 Identification:
        \vspace{2mm}
            \2 Exploit a 1701 Royal Decree that banned livestock grazing within 80km of the coast of Brazil.
        \vspace{2mm}
            \2 Created a separation between where the land grants for livestock could be assigned.
    \end{outline}
\end{frame}

\begin{frame}{Contribution}
    \begin{outline}
        \1 Understanding the historical effects of land distribution and usage in Brazil.
        \vspace{2mm}
            \2 Americas: 
            \cite{Wigton-Jones2020-ex} (JEG),
            \cite{Sellars2018-yp} (JDE),
            \cite{Smith2023-ip} (WP)
            \vspace{2mm}
            \2 India and Africa: 
            \cites{Banerjee2005-ki} (AER)
        \vspace{2mm}
        \1 Understand the persistent effects of colonial Brazil's economic structure on the present.
        \vspace{2mm}
        \2 Institutional and Natural Endowments: 
        \cite{Acemoglu2001-dz} (AER), 
        \cite{Sokoloff2000-mb} (JEP).
        \vspace{2mm}
        \2 
        \cite{Naritomi2012-or} (JEH), 
        \cite{Musacchio2014-pq} (JEH),
        \cite{Laudares2022-vy} (WP). 
    \end{outline}
\end{frame}

\begin{frame}{Outline}
    \begin{enumerate}
        \item Background
        \vspace{2mm}
        \item Data
        \vspace{2mm}
        \item Identification Strategy
        \vspace{2mm}
        \item Channels
        \vspace{2mm}
        \item Results
        \vspace{2mm}
        \item Discussion
    \end{enumerate}
\end{frame}

\begin{frame}{Background}
    \begin{outline}
        \1 Goal was to encourage Portuguese settlement of Brazil.
        \vspace{2mm}
        \1 One of the few ways to have access to land in colonial Brazil and given to people who could afford to develop the land \parencites{Smith1944-oi}{Dean1971-iq}.
        \vspace{-1mm}
        \1 People without direct access to it were often marginalized \parencite{Simonsen2005-ps}.
        \vspace{2mm}
        \1 Lasted until 1822.
        \vspace{2mm}
        \1 Historical and anecdotal evidence of the land grants having permanent effects in Brazilian economic structure:
        \vspace{2mm}
            \2 Early studies argued it led to the development of the ``\textcolor{red}{\textbf{economic aristocracy of the colonial society}}'' and the ``\textcolor{red}{\textbf{principal cause of the [large estates]}}'' in Brazil \parencites[p.~36]{Lima2002-kd}[p.~48]{Da_Costa_Porto1979-dz}.
    \end{outline}    
\end{frame}


%\begin{frame}{Channels}
%    \begin{outline}
%        \1 \textbf{Labor Outcomes} $\Rightarrow$ Land distribution could lead to different specialization (sugarcane vs. livestock).
%        \pause 
%        \vspace{2mm}
%        \1 \textbf{Land Tenure} $\Rightarrow$ Different sizes of land grants could lead to different utilization of land, and how owners would exploit it.
%        \pause 
%        \vspace{2mm}
%        \1 \textbf{Demographic Differences} $\Rightarrow$ Land grants often required African slaves, which could skew the demographics of a location.
%    \end{outline}
%\end{frame}

\begin{frame}{\hypertarget{data}Data}
    \begin{outline}
        \1 Land Grant Locations:
        \vspace{2mm}
            \2 Information on the land grants from the \href{http://plataformasilb.cchla.ufrn.br/}{Sesmarias of the Luso-Brazilian Empire Database} [\textbf{Novel Data}]
        \vspace{2mm}
            \2 Added the state of Bahia, currently in progress for March to finish Sao Paulo and Minas Gerais. \hyperlink{current_status}{\beamerbutton{Map}}
        \vspace{2mm}
        \1 Check whether they had an effect in the past:
        \vspace{2mm}
            \2 1872 Brazilian Census [\textbf{Novel Data at a Finer Geographical Level}] \hyperlink{parishes}{\beamerbutton{Parishes}}
        \vspace{2mm}
        \1 Present-Day Effects on Land Usage (10 x 10km Grid)
        \vspace{2mm}
            \2 1985 LandSat data from MapBiomas
        \vspace{2mm}
        \1 Present-Day Effects on Land Tenure (1995 Municipalities)
        \vspace{2mm}
            \2 1995 Brazilian Agricultural Census
    \end{outline}
\end{frame}

% Be more clear on agriculture
% Interact with land grants 
% Put somewhere the Y variables I'm taking a look at 
% Make very clear on what I'm using
% Show first the OLS - main things that are going on 
% Coast_Dist
% What the story: be more clear 
% I can show the RDD
% Make clear 
% Run for separate regions
% Show in Table form 
% Maybe check if the law had a bite in a different 

\begin{frame}{Identification Strategy}{Coastal Ban on Livestock}
    \begin{outline}
        \1 In 1701, the Portuguese Crown enacted a ban on cattle ranching within 80km of the coast \parencites[p~.40]{Fausto2014-bh}[p~.198]{Simonsen2005-ps}[p~.460]{Bethell1984-of}.
        \vspace{2mm} 
            \2 Goal was to prevent grazing on coastal farms.
        \vspace{2mm}
        \pause 
        \1 ``Landholding in the [interior] was truly extensive [...] The [land grants] on which cattle ranches were established sometimes exceeded hundreds of thousands of acres" \parencite{Bethell1984-of} 
        \vspace{2mm}
        \1 ``Extensive cattle raising, with open grazing, did not require much attention or labor. For that reason, the number of slaves in the region was small'' \parencite[p.~113]{De_Oliveira_Andrade1980-xz}
    \end{outline}
\end{frame}

\begin{frame}{Channels}
% Please add the following required packages to your document preamble:
% \usepackage{graphicx}
\begin{table}[]
    \centering
    \caption{}
    \label{tab:my-table}
    \resizebox{\textwidth}{!}{%
    \begin{tabular}{lll}
                                               & \multicolumn{1}{c}{\textbf{Within 80 km}}                                                                          & \multicolumn{1}{c}{\textbf{Past 80 km}}                                                                              \\ \cline{2-3} 
    \multicolumn{1}{l|}{\textbf{Land Grants}}  & \multicolumn{1}{l|}{No Livestock}                                                                                  & \multicolumn{1}{l|}{Either Crops or Livestock}                                                                       \\ \cline{2-3} 
    \multicolumn{1}{l|}{\textbf{Labor}}        & \multicolumn{1}{l|}{\begin{tabular}[c]{@{}l@{}}More Human Capital Intensive\\ Focused on Agriculture\end{tabular}} & \multicolumn{1}{l|}{\begin{tabular}[c]{@{}l@{}}Less Human Capital Intensive\\ Focused on Livestock\end{tabular}}     \\ \cline{2-3} 
    \multicolumn{1}{l|}{\textbf{Demographics}} & \multicolumn{1}{l|}{More Slavery}                                                                                  & \multicolumn{1}{l|}{Less Slavery}                                                                                    \\ \cline{2-3} 
    \multicolumn{1}{l|}{\textbf{Land Usage}}   & \multicolumn{1}{l|}{\begin{tabular}[c]{@{}l@{}}Smaller\\ More Productive\end{tabular}}                             & \multicolumn{1}{l|}{\begin{tabular}[c]{@{}l@{}}Larger Lands\\ Poorer Utilization\\ Lower Land Security\end{tabular}} \\ \cline{2-3} 
    \end{tabular}%
    }
\end{table}
\end{frame}

\begin{frame}{Identification Strategy}{Coastal Ban on Livestock}
    \begin{outline}
        \1 Regression Discontinuity design exploiting this 80km cutoff.
    \end{outline}
    \vspace{2mm}
    \begin{equation}
        Y_{m,s} = \beta \cdot CoastDist_{m,s} + f(D_{m,s})+ \mu_s + \epsilon_{m,s}
    \end{equation}

    \begin{outline}
        \1 Descriptive OLS to distinguish between the border effects vs. more general effects.
        \vspace{2mm}
            \2 Also shows evidence of selection and bias.
    \end{outline}
    
    \begin{equation}
        Y_{m,s} = \gamma \cdot Past80km_{m,s} + \mu_s + \epsilon_{m,s}
    \end{equation}
    \vspace{-1mm}
    \begin{outline}
        \1 Identify effects on demographics and labor (1872), land usage (1985 and 1995). 
    \end{outline}
\end{frame}

\begin{frame}{Distribution of Land Grants pre- and post- 1701}
    \begin{figure}
        \begin{center}
           \makebox[\textwidth]			 
           {\includegraphics[width=0.85\paperwidth]{/Users/vinicius/Library/CloudStorage/OneDrive-UniversityofIllinois-Urbana/Research/Projects/JMP/02. Figures/00.Maps/land_grant_distribution_1701.png}}
        \end{center}
        \label{fig:SesmariasDistribution}
      \end{figure}      
\end{frame}

\begin{frame}{Distribution of Cattle Land Grants pre- and post- 1701}
    \begin{figure}
        \begin{center}
           \makebox[\textwidth]			 
           {\includegraphics[width=0.85\paperwidth]{/Users/vinicius/Library/CloudStorage/OneDrive-UniversityofIllinois-Urbana/Research/Projects/JMP/02. Figures/00.Maps/land_grant_cattle_distribution_1701.png}}
        \end{center}
      \end{figure}      
\end{frame}

\begin{frame}{\hypertarget{1872_demographics}1872 Census - RD vs. OLS}{Demographics}
    \scriptsize
    \input{~/OneDrive - University of Illinois - Urbana/Research/Projects/JMP/03. Tables/1872_OLS_Demographics.tex}
    \hyperlink{1872_slavery}{\beamerbutton{Enslaved}}
    \hyperlink{1872_white}{\beamerbutton{White}}
    \hyperlink{1872_free_black}{\beamerbutton{Free Black}}
\end{frame}

\begin{frame}{\hypertarget{1872_labor}1872 Census RD vs. OLS}
    {Descriptive OLS - Labor}
    \scriptsize
    \input{~/OneDrive - University of Illinois - Urbana/Research/Projects/JMP/03. Tables/1872_OLS_Labor.tex}
    \hyperlink{1872_agriculture}{\beamerbutton{Crops}}
    \hyperlink{1872_ranching}{\beamerbutton{Ranching}}
    \hyperlink{1872_industry}{\beamerbutton{Industry}}
\end{frame}


\begin{frame}{\hypertarget{1985_landsat}1985 LandSat RD vs. OLS}{Land Usage}
    \scriptsize
    \input{~/OneDrive - University of Illinois - Urbana/Research/Projects/JMP/03. Tables/1985_Grid_Land_Usage_Results_NE.tex}
    \hyperlink{1985_nat}{\beamerbutton{Natural}}
    \hyperlink{1985_pasture}{\beamerbutton{Pasture}}
    \hyperlink{1985_sugarcane}{\beamerbutton{Sugarcane}}
\end{frame}

\begin{frame}{\hypertarget{1995_Ag_Census}1995 Agricultural Census RD vs. OLS}{Agricultural Land Usage}
    \footnotesize
    \input{~/OneDrive - University of Illinois - Urbana/Research/Projects/JMP/03. Tables/1995_Ag_Census_Results_NE.tex}
    \hyperlink{1995_livestock}{\beamerbutton{Livestock}}
    \hyperlink{1995_occupied}{\beamerbutton{Occupied}}
    \hyperlink{1995_over2000ha}{\beamerbutton{Over 2000ha}}
\end{frame}

\begin{frame}{Discussion}
    \begin{enumerate}
        \item Overall, not much happening at the border (RD).
        \vspace{2mm}
        \item Null results $\centernot\implies$ No effect of the land grants.
        \vspace{2mm}
        \item Other ways to disentangle/combine the land grants + this 80 km cutoff?
    \end{enumerate}
\end{frame}

\begin{frame}{Next Steps}
    \begin{outline}
        \1 Incorporate the feedback today.
        \vspace{2mm}
        \1 Incorporate 1970-2010 census microdata for future methods.
        \vspace{2mm}
        \1 Wrap up the data for Minas and Sao Paulo + start Ceara.
    \end{outline}
\end{frame}

\begin{frame}[allowframebreaks, t, noframenumbering, plain]{References}
    \printbibliography
\end{frame}

\appendix

\begin{frame}{History/Background}{Request Process}
    \begin{outline}
        \1 Petitioner submits a letter for an unoccupied land detailing their qualifications (captain, governor, etc.)
        \vspace{1mm}
        \1 Governor reads it, and if accepted returns back a letter with the requirements for the petitioner to satisfy.
        \vspace{1mm}
        \1 Five years to develop the land
        \vspace{1mm}
        \1 If successful, upon an inspection, land was transferred to the \textit{sesmeiro}.
        \vspace{1mm}
        \1 Able to sell, pass down as inheritance, etc. 
    \end{outline}
\end{frame}

\begin{frame}{\hypertarget{current_status}Current Land Grant Data Status \hyperlink{data}{\beamerbutton{Back}}}
    \begin{outline}
        \1 Timeline is to have both Minas and Sao Paulo by March.
    \end{outline}
        \begin{figure}[h!]
            \begin{center}
               \makebox[\textwidth]			 
               {\includegraphics[width=0.85\paperwidth]{/Users/vinicius/Library/CloudStorage/OneDrive-UniversityofIllinois-Urbana/Research/Projects/JMP/02. Figures/00.Maps/current_status_sesmarias.png}}
            \end{center}
          \end{figure}
    \end{frame}
    
    \begin{frame}{\hypertarget{parishes}1872 Parish Level Information }{[New Data] \hyperlink{data}{\beamerbutton{Back}}}
        \begin{figure}[h!]
            \begin{center}
               \makebox[\textwidth]			 
               {\includegraphics[width=0.85\paperwidth]{/Users/vinicius/Library/CloudStorage/OneDrive-UniversityofIllinois-Urbana/Research/Projects/JMP/02. Figures/00.Maps/parishes_1872.png}}
            \end{center}
            \label{fig:parishes_1872}
          \end{figure}
    \end{frame}

\begin{frame}{1872 Results}{Slavery}
    \begin{figure}
        \begin{center}
           \makebox[\textwidth]			 
           {\includegraphics[width=0.85\paperwidth]{/Users/vinicius/Library/CloudStorage/OneDrive-UniversityofIllinois-Urbana/Research/Projects/JMP/02. Figures/00.Maps/1872_Census_Enslaved_People.png}}
        \end{center}
    \end{figure}
\end{frame}

\begin{frame}{\hypertarget{1872_slavery}1872 Results}{Slavery \hyperlink{1872_demographics}{\beamerbutton{Back}}}
    \begin{figure}
        \begin{center}
           \makebox[\textwidth]			 
           {\includegraphics[width=0.85\paperwidth]{/Users/vinicius/Library/CloudStorage/OneDrive-UniversityofIllinois-Urbana/Research/Projects/JMP/02. Figures/00.Maps/1872_slaves_RDD.png}}
        \end{center}
      \end{figure}
\end{frame}

\begin{frame}{\hypertarget{1872_white}1872 Results}{White Population \hyperlink{1872_demographics}{\beamerbutton{Back}}} 
    \begin{figure}
        \begin{center}
           \makebox[\textwidth]			 
           {\includegraphics[width=0.85\paperwidth]{/Users/vinicius/Library/CloudStorage/OneDrive-UniversityofIllinois-Urbana/Research/Projects/JMP/02. Figures/00.Maps/1872_white_RDD.png}}
        \end{center}
      \end{figure}
\end{frame}

\begin{frame}{\hypertarget{1872_free_black}1872 Results}{Free Black Population \hyperlink{1872_demographics}{\beamerbutton{Back}}}
    \begin{figure}
        \begin{center}
           \makebox[\textwidth]			 
           {\includegraphics[width=0.85\paperwidth]{/Users/vinicius/Library/CloudStorage/OneDrive-UniversityofIllinois-Urbana/Research/Projects/JMP/02. Figures/00.Maps/1872_free_black_RDD.png}}
        \end{center}
      \end{figure}
\end{frame}

\begin{frame}{\hypertarget{1872_agriculture}1872 Results}{Agriculture \hyperlink{1872_labor}{\beamerbutton{Back}}}
    \begin{figure}
        \begin{center}
           \makebox[\textwidth]			 
           {\includegraphics[width=0.85\paperwidth]{/Users/vinicius/Library/CloudStorage/OneDrive-UniversityofIllinois-Urbana/Research/Projects/JMP/02. Figures/00.Maps/1872_agriculture_RDD.png}}
        \end{center}
      \end{figure}
\end{frame}

\begin{frame}{\hypertarget{1872_ranching}1872 Results}{Ranching \hyperlink{1872_labor}{\beamerbutton{Back}}}
    \begin{figure}
        \begin{center}
           \makebox[\textwidth]			 
           {\includegraphics[width=0.85\paperwidth]{/Users/vinicius/Library/CloudStorage/OneDrive-UniversityofIllinois-Urbana/Research/Projects/JMP/02. Figures/00.Maps/1872_ranching_RDD.png}}
        \end{center}
      \end{figure}
\end{frame}

\begin{frame}{\hypertarget{1872_industry}1872 Results}{Industry \hyperlink{1872_labor}{\beamerbutton{Back}}}
    \begin{figure}
        \begin{center}
           \makebox[\textwidth]			 
           {\includegraphics[width=0.85\paperwidth]{/Users/vinicius/Library/CloudStorage/OneDrive-UniversityofIllinois-Urbana/Research/Projects/JMP/02. Figures/00.Maps/1872_industry_RDD.png}}
        \end{center}
      \end{figure}
\end{frame}

\begin{frame}{1872 Results}{Other Professions}
    \begin{figure}
        \begin{center}
           \makebox[\textwidth]			 
           {\includegraphics[width=0.85\paperwidth]{/Users/vinicius/Library/CloudStorage/OneDrive-UniversityofIllinois-Urbana/Research/Projects/JMP/02. Figures/00.Maps/1872_other_prof_RDD.png}}
        \end{center}
      \end{figure}
\end{frame}

\begin{frame}{1995 Agricultural Census Results}{Productive Land Not Used}
    \begin{figure}
        \begin{center}
           \makebox[\textwidth]			 
           {\includegraphics[width=0.85\paperwidth]{/Users/vinicius/Library/CloudStorage/OneDrive-UniversityofIllinois-Urbana/Research/Projects/JMP/02. Figures/00.Maps/ag_census_1995_productive_land_not_used_map.png}}
        \end{center}
      \end{figure}
\end{frame}

\begin{frame}{1995 Agricultural Census Results}{Occupied Land}
    \begin{figure}
        \begin{center}
           \makebox[\textwidth]			 
           {\includegraphics[width=0.85\paperwidth]{/Users/vinicius/Library/CloudStorage/OneDrive-UniversityofIllinois-Urbana/Research/Projects/JMP/02. Figures/00.Maps/ag_census_1995_occupied_map.png}}
        \end{center}
        \label{fig:occupied_area_1995}
    \end{figure}
\end{frame}

\begin{frame}{\hypertarget{1995_livestock}1995 Agricultural Census}{Livestock \hyperlink{1995_Ag_Census}{\beamerbutton{Back}}}
    \begin{figure}
        \begin{center}
           \makebox[\textwidth]			 
           {\includegraphics[width=0.85\paperwidth]{/Users/vinicius/Library/CloudStorage/OneDrive-UniversityofIllinois-Urbana/Research/Projects/JMP/02. Figures/00.Maps/1995_Ag_Census_Livestock_RDD.png}}
        \end{center}
      \end{figure}
\end{frame}

\begin{frame}{\hypertarget{1995_occupied}1995 Agricultural Census}{Occupied Land \hyperlink{1995_Ag_Census}{\beamerbutton{Back}}}
    \begin{figure}
        \begin{center}
           \makebox[\textwidth]			 
           {\includegraphics[width=0.85\paperwidth]{/Users/vinicius/Library/CloudStorage/OneDrive-UniversityofIllinois-Urbana/Research/Projects/JMP/02. Figures/00.Maps/1995_Ag_Census_Occupied_RDD.png}}
        \end{center}
      \end{figure}
\end{frame}

\begin{frame}{\hypertarget{1995_2000ha}1995 Agricultural Census}{Land Over 2000ha \hyperlink{1995_Ag_Census}{\beamerbutton{Back}}}
    \begin{figure}
        \begin{center}
           \makebox[\textwidth]			 
           {\includegraphics[width=0.85\paperwidth]{/Users/vinicius/Library/CloudStorage/OneDrive-UniversityofIllinois-Urbana/Research/Projects/JMP/02. Figures/00.Maps/1995_Ag_Census_2000ha_RDD.png}}
        \end{center}
      \end{figure}
\end{frame}

\begin{frame}{\hypertarget{1985_nat}1985 LandSat RD}{Natural Vegetation  \hyperlink{1985_landsat}{\beamerbutton{Back}}}
    \begin{figure}
        \begin{center}
           \makebox[\textwidth]			 
           {\includegraphics[width=0.85\paperwidth]{/Users/vinicius/Library/CloudStorage/OneDrive-UniversityofIllinois-Urbana/Research/Projects/JMP/02. Figures/00.Maps/coastal_rdd_grid_graph_natural_vegetation.png}}
        \end{center}
    \end{figure}
\end{frame}

\begin{frame}{\hypertarget{1985_sugarcane}1985 LandSat RD}{Area dedicated to sugarcane \hyperlink{1985_landsat}{\beamerbutton{Back}}}
    \begin{figure}
        \begin{center}
           \makebox[\textwidth]			 
           {\includegraphics[width=0.85\paperwidth]{/Users/vinicius/Library/CloudStorage/OneDrive-UniversityofIllinois-Urbana/Research/Projects/JMP/02. Figures/00.Maps/coastal_rdd_grid_graph_actual_sugarcane.png}}
        \end{center}
    \end{figure}
\end{frame}

\begin{frame}{\hypertarget{1985_pasture}1985 LandSat RD}{Planted Pasture \hyperlink{1985_landsat}{\beamerbutton{Back}}} 
    \begin{figure}
        \begin{center}
           \makebox[\textwidth]			 
           {\includegraphics[width=0.85\paperwidth]{/Users/vinicius/Library/CloudStorage/OneDrive-UniversityofIllinois-Urbana/Research/Projects/JMP/02. Figures/00.Maps/coastal_rdd_grid_graph_pasture.png}}
        \end{center}
    \end{figure}
\end{frame}

\begin{frame}{1995 Agricultural Census - 1970 Census}{Leased Land and Sugarcane Workers}
    \hspace*{-2cm}
    \includegraphics[width=0.65\paperwidth]{/Users/vinicius/Library/CloudStorage/OneDrive-UniversityofIllinois-Urbana/Research/Projects/JMP/02. Figures/00.Maps/coastal_rdd_1970_sugarcane.png}
    \hspace{-20mm}
    \includegraphics[width=0.65\paperwidth]{/Users/vinicius/Library/CloudStorage/OneDrive-UniversityofIllinois-Urbana/Research/Projects/JMP/02. Figures/00.Maps/ag_census_1995_leased_map.png}
\end{frame}

\begin{frame}{1995 Agricultural Census}{Over 2000 ha}
    \begin{figure}[h!]
        \begin{center}
           \makebox[\textwidth]			 
           {\includegraphics[width=0.85\paperwidth]{/Users/vinicius/Library/CloudStorage/OneDrive-UniversityofIllinois-Urbana/Research/Projects/JMP/02. Figures/00.Maps/ag_census_1995_latifundia_map.png}}
        \end{center}
      \end{figure}
\end{frame}

\begin{frame}{\hypertarget{1872_slave_labor}1872 RD vs. OLS}{Descriptive OLS - Labor Enslaved}
    \scriptsize
    \input{~/OneDrive - University of Illinois - Urbana/Research/Projects/JMP/03. Tables/1872_OLS_Labor_Enslaved.tex}
\end{frame}

\begin{frame}{\hypertarget{1872_free_labor}1872 RD vs. OLS}{Descriptive OLS - Labor Free}
    \scriptsize
    \input{~/OneDrive - University of Illinois - Urbana/Research/Projects/JMP/03. Tables/1872_OLS_Labor_Free.tex}
\end{frame}

\begin{frame}{1985 LandSat RD vs. OLS}{Geography}
    \scriptsize
    \input{~/OneDrive - University of Illinois - Urbana/Research/Projects/JMP/03. Tables/1985_Grid_Geography_Results_NE.tex}
\end{frame}


\end{document}
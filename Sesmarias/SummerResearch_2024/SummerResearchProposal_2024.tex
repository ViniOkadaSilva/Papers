\documentclass[11pt]{article}

\usepackage[utf8]{inputenc}
\usepackage{subfig}

\usepackage{setspace}
\spacing{1.213}

\usepackage{amsmath}

\usepackage{indentfirst}
\usepackage{pdflscape}
\usepackage[left=1in,right=1in,top=1in,bottom=1in]{geometry}

\DeclareUnicodeCharacter{0301}{\'{e}}

\usepackage[font=small,skip=0pt]{caption}
\usepackage[section]{placeins}
\usepackage{titlesec}
\titlelabel{\thetitle.\quad}
\usepackage{authblk}
\usepackage{csquotes}
\usepackage{booktabs}
\usepackage{siunitx}
\usepackage{amssymb}
\newcolumntype{d}{S[input-symbols = ()]}
\usepackage{float}
\usepackage{dcolumn}
\usepackage[bottom]{footmisc} 
\usepackage[textsize=tiny]{todonotes}
\usepackage{longtable}
\usepackage{array}
\usepackage{outlines}
\usepackage{multirow}
\usepackage{wrapfig}
\usepackage{float}
\usepackage{colortbl}
\usepackage{pdflscape}
\usepackage{tabu}
\usepackage{threeparttable}
\usepackage{threeparttablex}
\usepackage[normalem]{ulem}
\usepackage{makecell}
\usepackage{xcolor}
\usepackage{afterpage}
\usepackage{titlesec}
\titleformat{\section}{\normalsize\bfseries}{\thesection.}{1em}{}
\usepackage{graphicx}
\usepackage{bbding}
\usepackage{xargs}
\usepackage{xpatch}
\usepackage[citecolor = blue]{hyperref}
\hypersetup{colorlinks = true, linkcolor = red, urlcolor = teal}

\usepackage{bookmark}
\usepackage{todonotes}
\setlength{\marginparwidth}{2cm}

\DeclareUnicodeCharacter{0301}{\'{e}}
\DeclareUnicodeCharacter{2212}{-}
\DeclareUnicodeCharacter{0327}{\c}

\usepackage[backend = biber, style=authoryear, sorting = nty, maxcitenames=1]{biblatex}

\addbibresource{citations_sesmarias.bib}

\graphicspath{{~/OneDrive - University of Illinois - Urbana/Research/Projects/Sesmarias Brazil/Figures/Descriptive/}}

%\addbibresource[location = remote]{https://raw.githubusercontent.com/ViniOkadaSilva/Papers/master/Sesmarias/citations_sesmarias.bib}

\DeclareFieldFormat{citehyperref}{%
  \DeclareFieldAlias{bibhyperref}{noformat}% Avoid nested links
  \bibhyperref{#1}}

\DeclareFieldFormat{textcitehyperref}{%
  \DeclareFieldAlias{bibhyperref}{noformat}% Avoid nested links
  \bibhyperref{%
    #1%
    \ifbool{cbx:parens}
      {\bibcloseparen\global\boolfalse{cbx:parens}}
      {}}}

\savebibmacro{cite}
\savebibmacro{textcite}

\renewbibmacro*{cite}{%
  \printtext[citehyperref]{%
    \restorebibmacro{cite}%
    \usebibmacro{cite}}}

\renewbibmacro*{textcite}{%
  \ifboolexpr{
    ( not test {\iffieldundef{prenote}} and
      test {\ifnumequal{\value{citecount}}{1}} )
    or
    ( not test {\iffieldundef{postnote}} and
      test {\ifnumequal{\value{citecount}}{\value{citetotal}}} )
  }
    {\DeclareFieldAlias{textcitehyperref}{noformat}}
    {}%
  \printtext[textcitehyperref]{%
    \restorebibmacro{textcite}%
    \usebibmacro{textcite}}}

\renewcommand*{\nameyeardelim}{\addcomma\space}

\usepackage{setspace}
\usepackage{graphicx}

\newcommand{\tinytable}[1]{\textcolor{black}{\tiny \input{#1}}}

\begin{document}

\begin{center}
\large \textbf{Summer Research Proposal}
\\
\vspace{3mm}
\large{\textbf{Colonial Portuguese Land Grants in Brazil: Long-term Effects on Development}}
\\
\vspace{3mm}
\smallskip
\normalsize{Vinicius Okada da Silva}
\end{center}

\vspace{-5mm}

\section{Background}

Brazil has one of the highest levels of land inequality in the world, with ``an estimated 1\% of the population own[ing] 45\% of all land" \parencite{Usaid2016-xs}. 
This issue is compounded by the fact that large agricultural lands in Brazil are often unproductive, with the Brazilian agrarian reform agency indicating that in 2010 ``72\% of all land occupied by large holdings was considered unproductive'' \parencite{Carlson2019-mk}.
The combination of land concentration and low levels of utilization are spread through the economy as it depresses rural wages, keeping those people away from the consumer markets \parencite[p.~1]{De_Oliveira_Andrade1980-xz}.
% However, this is not a recent issue as even in the 1920 census, average land inequality per municipality was already high \parencite{Wigton-Jones2020-ex}. 
However, land inequality is something that has persisted in Brazil ever since its colonization.
The goal of this project is to study the historical colonial causes of land inequality in Brazil, by exploiting geographical variation in the request for land grants, called \textit{sesmarias}.
% Even during the colonial period of Brazil, disputes over land existed and the conflict be

These land grants were the first legal method for obtaining land in colonial Brazil \parencite{Smith1944-oi}.
The grants began early during Brazil's colonization, with the first one being granted in 1530 \parencite[p.~16]{Diegues_Junior1959-ba}. 
The application process would require petitioners to write a letter to the governor providing the reason for the request and proof of the financial abilities to use the land correctly \parencite{Smith1944-oi}.
Given the financial means and the need to populate Brazil, often anyone who claimed a land grant was given a grant with an extent between 16.7 to 50.1 square miles \parencite{Dean1971-iq}.
As a result, access to land was already restricted to the wealthier population in colonial times resulting in the economic marginalization of the rural poor \parencites{Lobb1976-mc}[p.~23-24]{Panini1990-rj}.
The granting of \textit{sesmarias} ended in 1822, briefly before Brazil's independence.

While the colonial land grant system is not present in modern Brazil, historians argue that its effects can be seen in modern Brazil.
The wealth requirement alongside the vast tracts that were often given away are argued to be one of causes of the ``economic aristocracy of the colonial society'' and the ``principal cause of the [large estates]'' in colonial Brazil \parencites[p.~36]{Lima2002-kd}[p.~48]{Da_Costa_Porto1979-dz}. 
\textcite[p.~18]{Andrade1980-md} further argues that even in the 1980s ``the system of ownership and use of land is a continuation of the colonial system, with the [land grants] becoming [the large estates]".

\section{Research Question}

Given the historical importance of the land grants in colonial Brazil in establishing the basis of land usage in Brazil and its long duration of 300 years, this project aims to establish a novel georeferenced database of Portuguese colonial land grants in Brazil and use it to identify the historical impact of the land grants in economic development in Brazil. First, I would see how the effects of the land grants have persisted through the centuries, and second, I would analyze through what mechanisms they were able to persist. 

\section{Significance}

Given the prominence of land grants in colonial Brazil and the variation on why the land was granted, this project would contribute to the understanding of the role of colonial land assignment in long-term development.  
First, there are no empirical papers studying the direct causes of colonial land distribution in Brazil. 
Previous literature has found negative long-term effects of colonial land usage in Africa and South America \parencites{Dell2010-qt}{Lowes2021-ww}. 
However, there exists evidence that not all land regimes led to negative effects and instead led to economic development, with examples in India and Indonesia \parencites{Banerjee2005-ki}{Dell2019-np}{Ratnoo2023-vw}.  
Other studies have analyzed the effect of land grants in the United states \parencites{Akee2014-uw}{Allen2019-kh}{Smith2023-ip}

This paper also contributes to the understanding of the historical economic development of Brazil by trying to explain the diverging paths in development in each region. 
The land regime and size in each region, as measured by the land grants could have differential impacts on development.
\textcite{Wigton-Jones2020-ex} studies the effects of 1920 agricultural census land inequality and how it still has persisted to the present.
The literature has analyzed how different economic cycles and how immigration have led to differential educational outcomes in Brazil \parencites{Musacchio2014-pq}{Rocha2017-yq}.
Related literature has also analyzed the effect of the Spanish-Portuguese borders in South America, the role of sugarcane, and gold mining in Brazil \parencites{Laudares2022-vy}{Naritomi2012-or}.

\section{Data}

The primary data source is the location of the historical land grants in Brazil. Data has already been completed for five states in the Northeast: Bahia, Rio Grande do Norte, Pernambuco, Paraiba, and Alagoas.
Current data collection is still ongoing for two states in the Southeast: São Paulo and Minas Gerais with both expected to be done by the middle of March. 
I further have information on the year of the land grant, alongside the economic activity to which it was requested. 
A geographical description of the location of all the land grants data that been collected is available in Figure 01.\footnote{There's the possibility of adding another state in the Northeast, Ceara, by mid-April since its land grants have been organized in a table format.}

\begin{figure}[h!]
  \caption{Distribution of Land Grants across Brazil}
  \begin{center}
     \makebox[\textwidth]			 
     {\includegraphics[width=0.65\paperwidth]{/Users/vinicius/Library/CloudStorage/OneDrive-UniversityofIllinois-Urbana/Research/Projects/JMP/02. Figures/00.Maps/current_status_sesmarias.png}}
  \textit{Notes:} Geographical distribution of the land grants
  \end{center}
\end{figure}

Outcome data comes from a variety of sources.
Census data for 1872 is obtained from the Nucleus of Research in Economic and Geographic History from the Federal University of Minas Gerais.\footnote{Further work was done to georeference this dataset at a finer geographical level. Instead of having only georeferenced information at the municipality level, }
Both census and agricultural census data are obtained from the IBGE.\footnote{Microcensus is available through the IBGE but the data downloaded through the R package \textit{censobr} from \textcite{Pereira2023-qv}}
Land usage from 1985-2010 from LandSat satellite imagery is obtained from Mapbiomas \parencite{Souza2020-kb}.

\section{Identification}

To study the effects of the land grants I use a propensity score matching procedure to select control municipalities that are similar in geographical characteristics to those that received at least one land grant. In the first step, I estimate the following:

\begin{equation}
  LandGrant_m = X_{m,s} + \mu_s + \epsilon_{m,s}
\end{equation}

The set of variables used to match are: latitude, longitude, mean elevation, mean slope, soil quality for food crops (proxied by the measures of \textcite{Galor2016-ba}), potential sugarcane output from the FAO, and the distance to the coast. These variables are selected because they are proxies for agricultural output, geographical location, market access, and the main export of Brazil during the colonial times which was sugarcane. For each treated municipality I select one untreated municipality to be its control, which generates the matched sample.

From the matched sample I then estimate the following equation:

\begin{equation}
  Y_{m,s} = LandGrant_m + X_{m,s} + \mu_s + \epsilon_{m,s}
\end{equation}

The assumption for the matched sample is that conditional on the set of controls, the municipalities that received a land grant are as good as random since the control municipalities had similar geographical characteristics. 

\section{Preliminary Results}

In the table below, I show results from the 1995 Brazilian Agricultural Census. The sample includes only the five northeastern states where I have complete information on the land grants. In the odd columns, I show a simple OLS with controls, while in the even columns the results from the matched sample. I consider three types of treatment: if the municipality received any land grants, land grants before 1700, and after 1700. This breakdown is due to two rules, one in 1698 that limited the size of the grants, and another in 1701 that banned livestock grazing within 80km of the coast. The outcomes are the percentage of agricultural land that is used for livestock, the percentage of farms over 2000 hectares which measures land concentration, the percentage of occupied land which measures land uncertainty, and the percentage of agricultural land that is leased which measures land tenure. 

In Panel A, municipalities with a land grant are more likely to have a higher share of agricultural land being assigned to livestock. These municipalities do have a larger share of large farms indicating concentration of land. However, there are no effects on land insecurity or land leasing. 

When checking only for land grants before 1700 in Panel B there are no effects on livestock usage, mostly because they were coastal. However, these municipalities still see an increase in large estates measured by the proportion of farms over 2000ha indicating persistence of land concentration. There is also a decrease in land insecurity, as measured by the percentage of occupied agricultural land. The system of land tenure seems to be more towards land leasing indicating more cooperation between farmers and workers which matches the system established on sugarcane farms after the abolition of slavery.  

Given the size limit and livestock ban, the post-1700 grants would be expected to have differential effects. The results are shown in Panel C. First, we see that, unlike the early grants, there are increases in the percentage of the agricultural area used for livestock matching the historical expansion of cattle towards the interior. There are still similar effects on the proportion of large landholdings. However, there are no effects on land insecurity or land leasing. 

\section{Future Steps}

Over the summer I plan on further integrating the land grant data with the rest of the available datasets on why these effects have persisted up to 1995. 
That includes analyzing the effects on the 1872 census, and the 1920 agricultural census.
I would further explore the patterns on which the grants themselves were distributed. 
Given the results, I would also further explore what are the mechanisms that would have caused the effects to persist. 
I would also explore the heterogeneity of the effects by region. 
The contrast between the colonization of the Northeast and the Southeast the grants themselves would likely have differential impacts by region. 
I further plan to incorporate better the information on the proposed economic activity of a land grant. 


\input{~/OneDrive - University of Illinois - Urbana/Research/Projects/JMP/03. Tables/1995_Ag_Census_Matching.tex}



\clearpage

\xpatchbibmacro{author}{\printnames{author}}{\textbf{\printnames{author}}}{}{}

\printbibliography

\appendix

\end{document}
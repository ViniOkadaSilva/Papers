\documentclass{article}

\usepackage[utf8]{inputenc}
\usepackage{subfig}
\usepackage{setspace}

\usepackage{amsmath}
\doublespacing

\usepackage{indentfirst}
\usepackage{pdflscape}
\usepackage[left=1in,right=1in,top=1in,bottom=1in]{geometry}

\DeclareUnicodeCharacter{0301}{\'{e}}

\usepackage[font=small,skip=0pt]{caption}
\usepackage[section]{placeins}
\usepackage{titlesec}
\titlelabel{\thetitle.\quad}
\usepackage{authblk}
\usepackage{csquotes}
\usepackage{booktabs}
\usepackage{siunitx}
\newcolumntype{d}{S[ input-open-uncertainty=, input-close-uncertainty=, parse-numbers = false, table-align-text-pre=false, table-align-text-post=false ]}
\usepackage{amssymb}
% \newcolumntype{d}{S[input-symbols = ()]}
\usepackage{float}
\usepackage{dcolumn}
\usepackage[bottom]{footmisc} 
\usepackage[textsize=tiny]{todonotes}
\usepackage{longtable}
\usepackage{array}
\usepackage{multirow}
\usepackage{wrapfig}
\usepackage{colortbl}
\usepackage{pdflscape}
\usepackage{tabu}
\usepackage{threeparttable}
\usepackage{threeparttablex}
\usepackage[normalem]{ulem}
\usepackage{makecell}
\usepackage{xcolor}
\usepackage{afterpage}
\usepackage{graphicx}
\usepackage{bbding}
\usepackage{xargs}
\usepackage{xpatch}
\usepackage[citecolor = blue]{hyperref}
\hypersetup{colorlinks = true, linkcolor = red, urlcolor = teal}

\usepackage{bookmark}
\usepackage{todonotes}
\setlength{\marginparwidth}{2cm}

\usepackage[backend = biber, style=authoryear, sorting = nty, maxcitenames=1]{biblatex}

\addbibresource{citations_sesmarias.bib}

\DeclareFieldFormat{citehyperref}{%
  \DeclareFieldAlias{bibhyperref}{noformat}% Avoid nested links
  \bibhyperref{#1}}

\DeclareFieldFormat{textcitehyperref}{%
  \DeclareFieldAlias{bibhyperref}{noformat}% Avoid nested links
  \bibhyperref{%
    #1%
    \ifbool{cbx:parens}
      {\bibcloseparen\global\boolfalse{cbx:parens}}
      {}}}

\savebibmacro{cite}
\savebibmacro{textcite}

\renewbibmacro*{cite}{%
  \printtext[citehyperref]{%
    \restorebibmacro{cite}%
    \usebibmacro{cite}}}

\renewbibmacro*{textcite}{%
  \ifboolexpr{
    ( not test {\iffieldundef{prenote}} and
      test {\ifnumequal{\value{citecount}}{1}} )
    or
    ( not test {\iffieldundef{postnote}} and
      test {\ifnumequal{\value{citecount}}{\value{citetotal}}} )
  }
    {\DeclareFieldAlias{textcitehyperref}{noformat}}
    {}%
  \printtext[textcitehyperref]{%
    \restorebibmacro{textcite}%
    \usebibmacro{textcite}}}

\renewcommand*{\nameyeardelim}{\addcomma\space}

% Makes the authors name bold in the References section:

\def\sectionautorefname{Section}

\title{Land Grants in Colonial Brazil and Long-Term Effects on Development}

\author{Vinicius Okada da Silva\thanks{Contact information: University of Illinois at Urbana-Champaign. Department of Economics, 1407 W. Gregory Drive, David Kinley Hall: Room 126, Urbana, Illinois 61801. E-mail: vo10@illinois.edu}}

\affil{Department of Economics, University of Illinois at Urbana-Champaign}

\date{}

\begin{document}

\maketitle

\clearpage
\pagenumbering{arabic} 

\section{Introduction}

Brazil is a country that historically faces issues of both land and income inequality. Estimates from USAID in 2016

However, land inequality in Brazil is something that can be found in the past.

This paper tries to answer how much of Brazil's present-day inequality can be traced to colonial institutions. Specifically, this paper uses Portuguese land grants called \textit{sesmarias}, to identify the historical persistence of colonial activity in Brazil to present-day inequality.

\section{Historical Background}

\textcite{Dean}

\section{Data}

The main source of data comes from the \textit{Sesmarias of the Luso-Brazilian Empire Database}\footnote{
  Information on the content of the letters is available at \url{http://plataformasilb.cchla.ufrn.br/}. The georeferencing process was done as a separate project for this paper.}.
The database uses archival data from either state records or original manuscripts to obtain data on the concessions of sesmarias in Brazil. 
When available, information such as the year, the reason for the request, etc. are coded. 
The sesmarias are then georeferenced based on the information present in the text, allowing us to trace them back to Brazilian municipalities.

Census data for 1872 is obtained from the Nucleus of Research in Economic and Geographic History from the Federal University of Minas Gerais.\footnote{
  Available at \url{http://www.nphed.cedeplar.ufmg.br/}}
The 1872 Imperial Census contains demographic data at the municipality level and was the last census taken before the abolition of slavery in Brazil.

Other census data 

\section{Identification Strategy}

\printbibliography

\clearpage

\section{Figures}

\clearpage

\section{Tables}

\end{document}